\chapter{运动的描述}
\section{运动学基本概念}
\subsection{质点}

世界上的一切物体都有一定的大小和形状,在不同的问题中它们所起的作用是不同的.物理的任务就是在众多的物理现象中抽象出该物理现象的规律.我们从最简单的运动学展开对自然规律的探讨,物体的\CJKunderwave{位置} 随 \CJKunderwave{时间}
\footnote{关于时间的理解暂时可以按日常生活中的理解,在后面我们会给出时间的严格讨论.}
的变化,叫做\CJKunderwave{机械运动}.这是最简单的运动,但是描述起来却不像想像的那么简单,因为需要确定物体的位置,一个物体有大小,那么我们如何标志物体的位置?如果物体的形状还会发生变化,那么这个问题将变的更加复杂!所以要想研究物体的运动规律首先面对的是解决标志物体位置这个基本的问题.

在研究物体的机械运动时,为简化问题可以抓住主要问题忽略次要问题做一定的简化.如果在所研究的问题中,物体的\CJKunderwave{大小和形状对该问题影响不大}时,则可以忽略物体的大小,将物体看成一个\CJKunderwave{有质量的几何点},叫做{\bf 质点}.

质点是一种\CJKunderwave{理想物理模型},它实际上不存在,由于问题的复杂性往往采用一定的近似使它简化.比如,点电荷也是一种理想模型,还有理想变压器等.我们举一例来说明关于质点的常考的题目类型,如下:

\begin{xuanze}[example]
   1.下列关于质点的说法中,正确的是
   A.质点是一个理想化的模型,实际上并不存在,所以引入这个概念没有多大意义
   B.体积很小的物体更容易看做质点
   C.凡轻小的物体,皆可看做质点
   D.当物体的形状和大小对所研究的问题属于无关或者次要因素时,即可把物体看成质点

   a.D

   e.建立理想模型是物理中的重要的研究方法,对于复杂问题的研究有重大意义,A错误;一个物体能否看做质点不应看其大小,关键是看其大小对于研究的问题的影响能否忽略,体积很小的物体有时可以看成质点,有时不能看成质点,B错误;一个物体能否看成质点不以轻重而论,C错误;物体能否看成质点取决于其大小和形状对所研究的问题是否属于无关或次要因素,若是就可以看成质点,D正确.

\end{xuanze}
\subsection{参考系}
要描述一个物体的运动,首先要选定某个其它的物体做参考,观察物体相对于这个``其它物体''的位置是否随时间变化,以及怎样变化.这种 \CJKunderwave{用来做参考的物体} 称为参考系.

对一个物体的运动情况的描述,取决于所选择的参考系,选取的参考系不同,对于同一个物体运动的描述一般也不相同.

参考系具有相对性.它的具体含意为:对于一个物体的运动, \CJKunderwave{总能够找到一个参考系,使该物体对于此参考系是静止的} ,也就是静止具有相对性.对于多个物体,一般它们的运动不相同, \CJKunderwave{找不到一个参考系,使所有的物体对于该参考系都静止} ,也就是运动具有绝对性.

\begin{xuanze}[example]
   1.关于参考系,下列说法正确的是
   A.参考系必须是静止不动的物体
   B.参考系必须是静止不动或正在做直线运动的物体
   C.研究物体的运动,可选择不同的参考系,但是选择不同的参考系观察的结果是一样的
   D.研究物体的运动,可选择不同的参考系,但选择不同的参考系研究同一物体的运动而言,一般会出现不同的结果

   a.D

   e.参考系的选取是任意的,A,B错误;选择不同的参考系,对同一物体运动的描述一般是不同的,C错误,D正确.

\end{xuanze}

\subsection{坐标系}
上一节中,参考系可以确定物体是静还是动的问题.但是不能确定动多么快的问题,也就是定性的,所以要准确的描述物体的位置及位置变化需要建立坐标系,这个坐标系包括:\CJKunderwave{原点,正方向和单位长度.}

研究物体的直线运动时,一般建立直线坐标系,研究物体的曲线运动(轨迹是曲线的运动)时建立平面直角坐标系.另外还有极坐标系,自然坐标系等.感兴趣的同学可以参考一下相关的数学资料.\CJKunderwave{所有坐标系中的一个点和物体的位置一一对应}.

\begin{jisuan}[example]
   1.一质点在x轴上运动,各个时刻的位置坐标如
   \begin{tabular}{|*{7}{c|}}
      \hline
      t/s & 0 & 1 & 2 & 3 & 4 & 5\\
      \hline
      x/m & 0 & 5 & -4 & -1 & -7 & 1\\
      \hline
   \end{tabular}
   所示:
   \qitem 请画出x轴,在上面标出质点在各个时刻的位置.
   \qitem 哪个时刻离开坐标原点最远?有多远?

   a.见解析

   e.(1)各时刻质点的位置坐标如
      \begin{tikzpicture}[scale=0.4]
	 \draw [->] (-8,0)--(7,0);
	 \foreach \x in {-7,-6,-5,-4,-3,-2,-1,0,1,2,3,4,5}
	 \draw (\x,0pt)--(\x,3pt) node [anchor=north] {\x};
	 \draw (8,0) node [anchor=north east] {$x/m$};
	 \draw [<-] (-7,4pt)--(-7,24pt) node [anchor=south]{ 4s 末};
	 \draw [<-] (-4,4pt)--(-4,24pt) node [anchor=south]{ 2s 末};
	 \draw [<-] (-1,4pt)--(-1,24pt) node [anchor=south]{ 3s 末};
	 \draw [<-] (1,4pt)--(1,24pt) node [anchor=south]{ 5s 末};
	 \draw [<-] (5,4pt)--(5,24pt) node [anchor=south]{ 1s 末};
	 \draw [<-] (0,4pt)--(0,44pt) node [anchor=south]{0时刻};
      \end{tikzpicture}
   所示.

   ee.(2)由图可知第4s 末质点离开坐标原点最远,有7m.


\end{jisuan}

