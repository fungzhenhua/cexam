\documentclass[a4paper,fontset = windowsnew]{ctexbook}
\usepackage{xifthen}
\usepackage{calc}
\usepackage{graphicx}
\usepackage{tikz}
\usepackage{cexam}

\begin{document}
\chapter{行数统计程序}

\parindent=0pt

%\int_use:N \prevgraf
2019年8月29日下午重新改写下一代测行程序依据 prevgraf, 更加准确
\makeatletter
\ExplSyntaxOn
% 设置段落行数
\int_new:N \get_prevgraf_int
\int_new:N \get_prevgraf_i_int
\cs_new:Nn \sep_hd_old: {} %保存排过的版
\dim_new:N \sep_hd_old_ht  %文本高度
%  \cs_new:Npn \cexam_get_txtht:nno #1#2#3
% 四个参量依次为:1行数(返回),2高度(返回),3宽度(指定),4文本(指定)
  \cs_new:Npn \get_pg_ht:nnnn #1#2#3#4
  {
    \int_set:Nn \cexam_equ_int {\int_use:N\c@equation}
    \hbox_set:Nn \cexam_txtht_box
    {\parbox{#3}{#4\par\int_gset:Nn #1{\int_use:N \prevgraf}\quad}}
    \int_set:Nn \c@equation {\int_use:N \cexam_equ_int}
    \dim_set:Nn {#2}{\box_dp:N \cexam_txtht_box}
    \dim_add:Nn {#2}{\box_ht:N \cexam_txtht_box}
  }
% -->ok
% 定义hd_old: 累加程序
  \cs_new:Npn \sep_hd_old_add:n #1\scan_stop:
  {
   \exp_args:NNo \cs_set:Nn \sep_hd_old: {\sep_hd_old: #1}
  }
%%考虑文本比图高小的情况
  \cs_new:Npn \cexam_get_rec_i_new:nnnnnn #1#2#3#4#5#6
  {
% 置空old
    \cs_set:Nn \sep_hd_old: {} 
% 获得排版宽度
    \dim_set:Nn \cexam_pswd_dim {\linewidth}
    \dim_sub:Nn \cexam_pswd_dim {#3}
    \dim_sub:Nn \cexam_pswd_dim {#4}
    \dim_sub:Nn \cexam_pswd_dim {#5}
    \get_pg_ht:nnnn 
    {#1}
    {\sep_hd_old_ht}
    {\cexam_pswd_dim}
    {#6}
    \dim_compare:nNnTF
    {\sep_hd_old_ht} < {#2}
    {\relax}
    {
      \cexam_get_rec_new:nnnnnn
      {#1}{#2}{#3}{#4}{#5}{#6}
    }
  }
%% 重新改写开辟一个矩形区域的程序
% 六个参量1计数器,2矩形高,3矩形宽,4左缩进,5右缩进,6文本(含数学文本)
  \cs_new:Npn \cexam_get_rec_new:nnnnnn #1#2#3#4#5#6
  { 
% 获得排版宽度
    \dim_set:Nn \cexam_pswd_dim {\linewidth}
    \dim_sub:Nn \cexam_pswd_dim {#3}
    \dim_sub:Nn \cexam_pswd_dim {#4}
    \dim_sub:Nn \cexam_pswd_dim {#5}
% 对hd_old 测量高度 同时获得旧文本行数.
    \get_pg_ht:nnnn 
    {#1}
    {\sep_hd_old_ht}
    {\cexam_pswd_dim}
    {\sep_hd_old:}
%   分离头,干,尾
    \exp_args:No \cexam_sep:n #6 \scan_stop:
%  头部并入old
%    \exp_args:NNo \cs_set:Nn \sep_hd_old: {\sep_hd_old:\sep_hd_data:}
    \exp_args:No \sep_hd_old_add:n \sep_hd_data:\scan_stop:
%  取得old 的新高度
    \get_pg_ht:nnnn 
    {#1}
    {\sep_hd_old_ht}
    {\cexam_pswd_dim}
    {\sep_hd_old:}
% 对比旧高与图高
    \dim_compare:nNnTF 
    {\sep_hd_old_ht} > {#2}
    {
      \bool_if:NTF \g__cexam_sep_by_bool
      {\relax}
      {
	\dim_sub:Nn \sep_hd_old_ht {#2}
	\dim_while_do:nNnn 
	{\sep_hd_old_ht} > {0pt}
	{
	  \int_sub:Nn #1 {1}
	  \dim_sub:Nn \sep_hd_old_ht {\baselineskip}
	}
% 追加2的原因在于,第一取得总高度比实际高度要高,所以测量会多出一行;
% 第二,目标是计算剩下的行数,来减去,要求最后小于0,所以再追加一行保证大于0
	\dim_compare:nNnTF 
	{\dim_abs:n{\sep_hd_old_ht}} < {5pt}
	{\int_add:Nn #1{1}}
	{\int_add:Nn #1{2}}
      }
    }
    {
% 干部并入old
%      \exp_args:NNo \cs_set:Nn \sep_hd_old: {\sep_hd_old:\sep_by_data:}
      \exp_args:No \sep_hd_old_add:n \sep_by_data:\scan_stop:
% 取得old 的新高度
      \get_pg_ht:nnnn 
      {#1}
      {\sep_hd_old_ht}
      {\cexam_pswd_dim}
      {\sep_hd_old:}
% 对比旧高与图高
      \dim_compare:nNnTF 
      {\sep_hd_old_ht} > {#2}
      {\relax} %此处待补充数学结尾时的排版设定
      { 
% 整体高度不足,则进入循环
	\cexam_get_rec_new:nnnnnn 
	{#1}{#2}{#3}{#4}{#5}{\sep_tl_data:}
      }
    }
  }
%% 原排版程序改写成新程序
  \cs_new:Npn \cexam_type_i_new:nnnnnnn #1#2#3#4#5#6#7
  {
    \cexam_fmt_pic:nn {#1}{#2}
%    \cexam_get_rec_new:nnnnnn
    \cexam_get_rec_i_new:nnnnnn
    {\cexam_picmath_int}
    {\cexam_picht_dim}{\cexam_picwd_dim}
    {#3}{#4}{#7}
    \str_if_in:nnTF {#1}{l}
    {
      \dim_set:Nn \cexam_pslin_dim {#3}
      \dim_set:Nn \cexam_psrin_dim {#4}
      \dim_set:Nn \cexam_pswd_dim {\linewidth}
      \dim_add:Nn \cexam_pslin_dim {\cexam_picwd_dim}
      \cexam_sha_mk_ii:nnnnnnn
      {\cexam_picmath_int}
      {\cexam_pslin_dim}{\cexam_pswd_dim}{\cexam_psrin_dim}
      {#5}{\linewidth}{#6}
      \dim_add:Nn \cexam_picwd_dim{#3}
    }
    {
      \dim_set:Nn \cexam_pslin_dim {#3}
      \dim_set:Nn \cexam_psrin_dim {#4}
      \dim_set:Nn \cexam_pswd_dim {\linewidth}
      \dim_add:Nn \cexam_psrin_dim {\cexam_picwd_dim}
      \cexam_sha_mk_ii:nnnnnnn
      {\cexam_picmath_int}
      {\cexam_pslin_dim}{\cexam_pswd_dim}{\cexam_psrin_dim}
      {#5}{\linewidth}{#6}
    }
    \tex_parshape:D \cexam_shad:
    \box_use:N \cexam_picture_box
    #7
  }
%%%%%%%%%测试行数%%%%%%%%%%
\par
**********************************************
\par
 \dim_new:N \my_ht_dim

 \cexam_type_i_new:nnnnnnn
 {l}
 {
   \begin{tikzpicture}
     \draw (0,0) rectangle (3,6);
   \end{tikzpicture}
 }
 {0pt}{0pt}
 {0pt}{0pt}
 {
    这是行数测试文本
    这是行数测试文本
    这是行数测试文本
    \begin{equation}
      E=MC^2
    \end{equation}
    这是行数测试文本
    \begin{equation}
      E=MC^2
    \end{equation}
    这是行数测试文本
    这是行数测试文本
    \begin{equation}
      E=MC^2
    \end{equation}
    这是行数测试文本
    这是行数测试文本
    这是行数测试文本
    这是行数测试文本
    这是行数测试文本
    这是行数测试文本
    这是行数测试文本
    这是行数测试文本
    这是行数测试文本
    这是行数测试文本
    这是行数测试文本
    这是行数测试文本
    这是行数测试文本
    这是行数测试文本
    这是行数测试文本
    这是行数测试文本
    这是行数测试文本
    这是行数测试文本
    这是行数测试文本
    这是行数测试文本
    这是行数测试文本
    这是行数测试文本
    这是行数测试文本
    这是行数测试文本
    这是行数测试文本
    这是行数测试文本
    这是行数测试文本
    这是行数测试文本
    这是行数测试文本
    这是行数测试文本
    这是行数测试文本
    这是行数测试文本
    这是行数测试文本
    这是行数测试文本
    这是行数测试文本
    这是行数测试文本
    这是行数测试文本
    这是行数测试文本
    这是行数测试文本
    这是行数测试文本
    这是行数测试文是行数测试文本
  }

    \end{document}
