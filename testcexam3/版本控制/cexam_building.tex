\documentclass[a4paper,fontset = adobe]{ctexbook}
\usepackage{xifthen}
\usepackage{calc}
\usepackage{graphicx}
\usepackage{tikz}
\usepackage{cexam}

\begin{document}
\chapter{行数统计程序}
\makeatletter

行数统计程序的latex3改写

\ExplSyntaxOn
%%boolean
\bool_new:N \g__cexam_math_on_bool
\bool_new:N \g__cexam_math_tail_bool
%%
\cs_new:Npn \cexam_separate_i:n  #1$$#2$$#3\scan_stop:
{
  \cs_set:Nn \hang_math_head: {#1}
  \cs_set:Nn \hang_math_body: {$$#2$$}
  \cs_set:Nn \hang_math_tail: {#3}
}

\cs_new:Npn \cexam_separate_ii:n #1\[#2\]#3\scan_stop:
{
  \cs_set:Nn \hang_math_head: {#1}
  \cs_set:Nn \hang_math_body: {\[#2\]}
  \cs_set:Nn \hang_math_tail: {#3}
}

\cs_new:Npn \cexam_separate_iii:n #1\begin#2\end#3#4\scan_stop:
{
  \cs_set:Nn \hang_math_head: {#1}
  \cs_set:Nn \hang_math_body: {\begin#2\end{#3}}
  \cs_set:Nn \hang_math_tail: {#4}
}
\cs_new:Npn \cexam_separate_done:n #1\scan_stop:
{
  \str_if_in:nnTF {#1}{$$}%$$
  {\cexam_separate_i:n #1\scan_stop:}
  {\str_if_in:nnTF {#1}{\[}%\]
      {\cexam_separate_ii:n #1\scan_stop:}
      {
	\str_if_in:nnTF {#1}{\begin}
	{\cexam_separate_iii:n #1\scan_stop:}
	{}
      }

    }
  }
  \cs_new:Npn \cexam_separate_isin:nn #1#2
  {
    \str_if_in:nnTF {*#1}{*#2}
    {
      \bool_set_true:N \g__cexam_math_on_bool
      \str_if_in:nnTF {#1*}{#2*}
      {
	\cs_set:Nn \hang_math_head: {}
	\cs_set:Nn \hang_math_body: {}
	\cs_set:Nn \hang_math_tail: {}
	\bool_set_false:N \g__cexam_math_tail_bool
      }
      {
	\cexam_separate_done:n *#1\scan_stop:
	\cs_set:Nn \hang_math_head: {}
	\bool_set_true:N \g__cexam_math_tail_bool
      }
    }
    {
      \str_if_in:nnTF {#1*}{#2*}
      {
	\bool_set_true:N \g__cexam_math_on_bool
	\cexam_separate_done:n #1*\scan_stop:
	\cs_set:Nn \hang_math_tail: {}
	\bool_set_false:N \g__cexam_math_tail_bool
      }
      {
	\str_if_in:nnTF {#1}{#2}
	{
	  \bool_set_true:N \g__cexam_math_on_bool
	  \cexam_separate_done:n #1\scan_stop:
	  \bool_set_true:N \g__cexam_math_tail_bool
	}{}
      }
    }
  }
  \cs_new:Npn \cexam_separate:n #1 \scan_stop:
  {
    \str_if_in:nnTF {#1}{$$}%$$
    {
      \cexam_separate_isin:nn {#1}{$$}%$$
    }
    {
      \str_if_in:nnTF {#1}{\[}%\]
	{
	  \cexam_separate_isin:nn {#1}{\[}%\]
	  }
	  {
	    \str_if_in:nnTF {#1}{\begin}%\end
	    {
	      \cexam_separate_isin:nn {#1}{\begin}%\end
	    }
	    {
	      \cs_set:Nn \hang_math_head: {#1}
	      \cs_set:Nn \hang_math_body: {}
	      \cs_set:Nn \hang_math_tail: {}
	      \bool_set_false:N \g__cexam_math_tail_bool
	      \bool_set_false:N \g__cexam_math_on_bool
	    }
	  }
	}
      }

% 以下为补充命令
      \cs_new:Npn \cexam_separate:o #1\scan_stop:
      {
	\cexam_separate:n #1\scan_stop:
      }

      \bool_if:NTF \g__cexam_math_on_bool 
      {The~ math~ modle~ is~ open}
      {The~ text~ model~ is~ open}


      \par
      ******测行程序******
      \par

%%2019年7月30日,学习之计划重新规划这个cexam.sty的latex3的重写程序.
%%由于在进行行数测定时需要一步一步的追加需要处理的资料,所以在此处设计为将一直处理的到最后的内容放入靠前的位置,同时后面的位置追加资料后不改变前面等求量的符号,所以更加的明确.
%20190730begin
%四个参数依次为:1计数器增量,2计数器,3行减量,4总减行高

      \cs_new:Npn \cexam_get:nNnN #1#2#3#4
      {
	\dim_while_do:nNnn {#4}>{0pt}
	{
	  \dim_sub:Nn {#4}{#3}
	  \int_add:Nn {#2}{#1}
	}
      }
%%-->ok

%用来记录文本高度的盒子
      \box_new:N \cexam_TxtHt_box
%%三个参数依次为:1高(所要求的量),2宽,3文本
% 此程序产生文本的总体高度
      \cs_new:Npn \cexam_get_TxtHt:nno #1#2#3
      {
	\hbox_set:Nn \cexam_TxtHt_box 
	{\parbox{#2}{#3}}
	\dim_set:Nn {#1}{\box_dp:N \cexam_TxtHt_box}
	\dim_add:Nn {#1}{\box_ht:N \cexam_TxtHt_box}
      }
%%-->ok
      \dim_new:N \rectangle_tempht_dim
%六个参数依次为:1计数器增量,2计数器,3行减量,4总减行高,5矩形宽,6文本(含数学文本)
      \cs_new:Npn \cexam_get_rectangle_i:nNnNnn #1#2#3#4#5#6
      {
	\cexam_get_TxtHt:nno {\rectangle_tempht_dim}{\linewidth - #5}{#6}
	\dim_compare:nTF {\rectangle_tempht_dim < #4}
	{
	  \dim_sub:Nn #4 {\rectangle_tempht_dim}
	  \cexam_get:nNnN {#1}{#2}{#3}{\rectangle_tempht_dim}
	}
	{\cexam_get:nNnN {#1}{#2}{#3}{#4}}
      }
%%行数计算中使用的长度
      \int_new:N \cexam_equation
      \dim_new:N \cexam_mathhead_ht %高
      \dim_new:N \cexam_mathbody_ht %高
      \dim_new:N \cexam_math_minus  %数学减高
      \dim_new:N \cexam_text_minus  %文本减高
      \dim_set:Nn \cexam_text_minus{\baselineskip}
      \dim_set:Nn \cexam_math_minus{2\baselineskip}
%四个参数依次为:1计数器,2矩形高,3矩形宽,4文本(含数学文本)
      \cs_new:Npn \cexam_get_rectangle:nnnn #1#2#3#4
      {
	\bool_if:NTF \g__cexam_math_tail_bool
	{\exp_args:No \cexam_separate:n #4 \scan_stop:}
	{\cexam_separate:n #4 \scan_stop:}
	\cexam_get_rectangle_i:nNnNnn {1}{#1}{\cexam_text_minus}{#2}{#3}{\hang_math_head:}
%%以上先分离一次,对head计算行数,如果总减高不为零,则对body排版
	\dim_compare:nTF {#2 > 0pt}
	{
	  \bool_if:NTF \g__cexam_math_on_bool
	  {	
	    \cexam_get_rectangle_i:nNnNnn {3}{#1}{\cexam_math_minus}{#2}{#3}{\hang_math_body:}
	  }{\relax}
	}{\relax}
%%以上对body排版,如仍然不能令#4=0pt,同继续对tail排版
	\dim_compare:nTF {#2 > 0pt}
	{
	  \bool_if:NTF \g__cexam_math_tail_bool
	  {
	    \cexam_get_rectangle:nnnn {#1}{#2}{#3}{\hang_math_tail:}
	  }{\relax}
	}{\relax}
      }
%%-->ok_testing
%%开关生成形状
      \dim_new:N \cexam_psrin_dim %左缩进量
      \dim_new:N \cexam_pslin_dim %左缩进量
      \dim_new:N \cexam_pswd_dim %行长度
      \int_new:N \cexam_psnum_int %行数
      \cs_new:Nn \cexam_shaped: {}%形状存储
      \cs_new:Npn \cexam_shape_i:nn #1#2
      {
	\cs_set:Nn \cexam_shaped: {~#1~#2}
      }
%为了保证生成正确的形状,此处定义不可展开的空格
      \cs_new_protected:Nn \cexam_shape_cape:{~}
%四个参数为:1.计数器 2.左缩进 3.行长 4.右缩进 
      \cs_new:Npn \cexam_shape_make:nnnn #1#2#3#4
      {
	\int_add:Nn {#1}{1}
	\exp_args:NNx\cs_set:Nn \cexam_shaped:{\cexam_shape_cape: \int_use:N #1}
	\dim_set:Nn \cexam_pslin_dim {#2}
	\dim_set:Nn \cexam_psrin_dim {#4}
	\dim_set:Nn \cexam_pswd_dim {#3}
	\dim_sub:Nn \cexam_pswd_dim {#2}
	\dim_sub:Nn \cexam_pswd_dim {#4}
	\int_while_do:nNnn {#1} > {1}
	{
	  \int_sub:Nn {#1}{1}
	  \exp_args:NNx\cs_set:Nn \cexam_shaped:{\cexam_shaped:\cexam_shape_cape: \dim_use:N \cexam_pslin_dim}
	  \exp_args:NNx\cs_set:Nn \cexam_shaped:{\cexam_shaped:\cexam_shape_cape: \dim_use:N \cexam_pswd_dim}
	}
	  \exp_args:NNx\cs_set:Nn \cexam_shaped:{\cexam_shaped:\cexam_shape_cape: 0pt}
	  \exp_args:NNx\cs_set:Nn \cexam_shaped:{\cexam_shaped:\cexam_shape_cape: \dim_use:N \linewidth}
      }

%%-->testing
      \box_new:N \cexam_picture_box
      \box_new:N \cexam_vpicture_box
      \dim_new:N \cexam_picht_dim
      \dim_new:N \cexam_picwd_dim
      \int_new:N \cexam_picmath_int
%%1图片,2文本
      \cs_new:Npn \cexam_publish_mp:nn #1#2
      {
          \hbox_set:Nn \cexam_picture_box {#1}
	  \dim_set:Nn {\cexam_picwd_dim}{\box_wd:N \cexam_picture_box}
	  \dim_set:Nn {\cexam_picht_dim}{\box_ht:N \cexam_picture_box}
	  \dim_add:Nn {\cexam_picht_dim}{\box_dp:N \cexam_picture_box}
	  \cexam_get_rectangle:nnnn {\cexam_picmath_int}{\cexam_picht_dim}{\cexam_picwd_dim}{#2}
	  \cexam_shape_make:nnnn {\cexam_picmath_int}{0pt}{\linewidth}{\cexam_picwd_dim}
	  \hbox_set:Nn \cexam_picture_box {\raisebox{-\totalheight-\ccwd}[0pt][0pt]{\parbox[b]{\linewidth}{\hfill #1}}}
	  \box_set_wd:Nn \cexam_picture_box {0pt}
	  \tex_parshape:D \cexam_shaped:
	  \box_use:N \cexam_picture_box #2
      }

%%%%%%%%%%%%%%%%%%%%%%%%%%%%%%%%%%%%%%%%%%%%%%%%%%%%%%%%%%%%%%%%%%%%%%%%%

      \dim_new:N \g__exam_test_dim
      \int_new:N \g__exam_test_int
      \dim_set:Nn \g__exam_test_dim {50pt}

      \cexam_publish_mp:nn {
      \begin{tikzpicture}
	\draw (0,0) rectangle (2,2);
      \end{tikzpicture}
    }{
	这是第三版的图文试题排版测试
	这是第三版的图文试题排版测试
	\begin{equation}
	  E=MC^2
	\end{equation}
	这是第三版的图文试题排版测试
	这是第三版的图文试题排版测试
	这是第三版的图文试题排版测试
	这是第三版的图文试题排版测试
	这是第三版的图文试题排版测试
	这是第三版的图文试题排版测试
	这是第三版的图文试题排版测试
	这是第三版的图文试题排版测试
	这是第三版的图文试题排版测试
	这是第三版的图文试题排版测试
	这是第三版的图文试题排版测试
	这是第三版的图文试题排版测试
	这是第三版的图文试题排版测试
	这是第三版的图文试题排版测试
	这是第三版的图文试题排版测试
	这是第三版的图文试题排版测试
      }

\ExplSyntaxOff


\ExplFileName

\ExplFileDate

\ExplFileVersion

\ExplFileDescription

     


    \end{document}
