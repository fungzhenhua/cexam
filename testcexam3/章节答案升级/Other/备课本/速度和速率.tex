\section{速度和速率}
\subsection{机械运动}
物体的位置随时间的变化,我们称之为机械运动.简言之,就是研究物体的位移随时间的变化.
\subsection{速度}
\subsubsection{定义}
质点在$\Delta t$ 时间内发生的位移$ \Delta x $ 与 时间的比值,定义为速度.用公式表达如下
\begin{equation}
  v=\cfrac{\Delta x}{\Delta t}\label{eq:velocity}
\end{equation}
\subsubsection{矢标性}
 从定义上来看,位移$\Delta x$ 是一个矢量,时间$\Delta t$ 是一个标量,矢量除以标量还是一个矢量.所以速度是一个矢量.速度的方向我们称之为质点运动的方向,在一些题目中没有明确提出,也这样认为.
 \subsubsection{单位}
 每一个物理量都有一个特定的意义,所以也必然有一个单位.从定义看来,位移的单位是``米'',时间的单位是``秒'',所以速度单位是:米/秒,符号为$m/s$
\subsubsection{速度的物理意义}
速度是表征物体运动快慢的物理量.这里大家注意,如果问某个物理量的物理意义,我们统一的说法是:XX是表征XXXX的物理量.
\subsection{瞬时速度和平均速度}
\subsubsection{瞬时速度}
瞬时速度描述物理某一瞬时的运动的快慢,所以要求时间$\Delta t \rightarrow 0$,瞬时速度的定义为:
\begin{equation}
 v=\lim_{\Delta t \to 0} \cfrac{\Delta x}{\Delta t}
  \label{eq:instantaneous velocity}
\end{equation}

\subsubsection{平均速度}
平均速度描述质点运动的平均快慢程度,所以是在一段时间内发生的,不要求时间趋于零.为了区别于瞬时速度,在速度$v$ 的上方加一个横杠$\overline{v}$,读作 ``维拔'',其定义为:
\begin{equation}
\overline{v}=\cfrac{\Delta x}{\Delta t}
  \label{eq:average velocity}
\end{equation}

\subsection{速率}
\subsubsection{定义}
速率是指路程比上时间,表示单位时间内发生的路程.定义如下
\begin{equation}
v=\cfrac{\Delta s}{\Delta t}
  \label{eq:speed}
\end{equation}

\subsubsection{矢标性}
从速率的定义来看,路程是标量,时间是标量,所以速率也是一个标量.

\subsubsection{单位}
从速率定义来看,速率的单位和速度相同,都是$m/s$



\subsubsection{瞬时速率}
类比速度,则速率也分为瞬时速率和平均速率,瞬时速率定义如下:
\begin{equation}
v=\lim_{\Delta t \to 0} \cfrac{\Delta s}{\Delta t}
  \label{eq:instantaneous speed}
\end{equation}

\subsubsection{平均速率}
类比速度,则速率也分为瞬时速率和平均速率,平均速率定义如下:
\begin{equation}
\overline{v}=\cfrac{\Delta s}{\Delta t}
  \label{eq:average speed}
\end{equation}

\subsubsection{注意事项}
在初中时同学们运算都采用的速率,但是当时没有区分速度和速率,因为当时研究的是单向直线运动,这种情况下速度的大小和速率是相等的,所以没有出现错误.然而一旦情况复杂,则必须用速度来描述.

今后如没有特殊声明,则一律使用速度来描述运动,不再使用速率.

\subsection{例题分析}
\begin{selection}
  1.关于速度的定义$v=\frac{\Delta x}{\Delta t}$ ,以下叙述正确的是
  A.物体做匀速直线运动时,速度$v$ 与运动的位移 $\Delta x$ 成正比,与运动时间$\Delta t$ 成反比
  B.速度$v$ 的大小与运动的位移$\Delta x$ 和时间$\Delta t$ 都无关
  C.此速度定义适用于任何运动
  D.速度是表示物体运动快慢及方向的物理量

  a.BCD

  e.$v=\frac{\Delta v}{\Delta t}$ 是速度的定义式,所以适用于一切情况,C对;此定义法为比值定义法,不能说此物理量与分子成正比或者与分母成反比,所以A错,B对;速度的大小表示运动的快慢,方向表示物体运动的方向,所以D对.

  2.物体沿直线做单向直线运动,途经直线上的A,B,C 三点,经过这三点时的速度分别为$v_A$ , $v_B$ , $v_C$ ,则下列说法正确的是
  A.$v_A$, $v_B$ , $v_C$ 越大,则由A到C所用的时间越短
  B.经过 A,B,C 三点时的瞬时速率就是$v_A$ , $v_B$ , $v_C$
  C.由A到C这一阶段的平均速度为$\overline{v}=\frac{v_A+v_B+v_C}{3}$
  D.由A到C这一阶段的平均速度越大,则由A到C所用的时间越短

  a.D

  e.A到C所用时间取决于A到C的平均速度,与初,末态的瞬时速度无关,A错误,D正确.瞬时速度的大小叫瞬时速率,B错误.平均速度一般不等于速度的平均值,C错误.

\end{selection}

\begin{calculate}
3.一辆汽车由A出发做直线运动,前$5s$ 向东行驶了$30m$ 到达 B 点,又行驶了$5s$ 前进了$60m$ 到达 C点,在C点停了$4s$ 后又向西行驶,经历了$6s$  运动$12m$ 到达 A点西侧的D点.求
[1]全过程的平均速度;
[2]全过程的平均速率.

a.(1)平均速度大小为$1.5m/s$,方向水平向西 (2) 平均速率为$10.5m/s$

e.(1)设向东为正方向
$$x=30m+60m+(-120m)=-30m$$
全程用时
$$t=5s+5s+4s+6s=20s$$
所以平均速度为
$$\overline{v}=\cfrac{x}{t}=\cfrac{-30m}{20s}=-1.5m/s$$
负号表示速度方向向西
\newline
(2)全程的路程为
$$s=30m+60m+120m=210m$$
所以平均速率为
$$\overline{v}=\cfrac{s}{t}=\cfrac{210m}{20s}=10.5m/s$$


\end{calculate}
