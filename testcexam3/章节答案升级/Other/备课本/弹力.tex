\section{弹力}
\subsection{弹性形变}
物体在力的作用下发生的形状或体积改变叫做形变.物体在受力后会发生形变,若撤去外力作用后,该物体能够恢复原状,则这种形变叫弹性形变.

\subsection{塑性形变}
物体受外力作用而使物体各部分相对位置的改变,当外力撤消后,物体不能恢复原状则这种形变称为塑性形变.

物体在受到足够大的外力作用时,会发生永久性的形状改变,这种在外力撤去后物体形状的改变可保存下来的性能叫做范性.现在更多的时候称作塑性.

\subsection{弹力的定义}
发生弹性形变的物体,由于要恢复原状,要对跟它接触的物体产生力的作用,这种作用叫做弹力.换种说法,也就是在弹性限度内,物体对使物体发生形变的施力物体产生的力叫做弹力.

\subsection{产生条件}

弹力是接触力,只能存在于物体相互接触处,但是相互接触的物体之间,并不一定有弹力的作用,因为接触不代表有形变.

产生弹力的条件:一,接触;二,有弹性形变.

注意:物体发生的弹性形变,不一定都能看出来,这称之为微小形变.任何物体中只要发生形变,就一定会对与它接触的物体产生弹力.一旦超出了弹性形变的限度就会完全失去弹力,这种超过了其弹性承受范围的形变称为``范性形变''.(即超过了弹性限度,塑性物体除外)

\subsection{弹力的方向}
受力物体所受弹力的方向与施力物体弹性形变的方向相反.具体可以有下列情况
\subsubsection{轻绳}
轻绳上的弹力方向沿绳指向绳收缩的方向.
\subsubsection{压力和支持力}
压力、支持力的方向总跟接触面垂直,面与面接触,点与面接触的情况,弹力都垂直于面;点与点的接触要找到公切面,弹力垂直于这个公切面指向被支持物.
\subsubsection{杆}
杆上的弹力可以指向任意方向, 在一个具体问题中,可以由它所受外力和运动状态决定.
\subsubsection{铰链}
一个杆如果一端使用一个轴固定住,但是可以自由转动,这个结构叫做铰链.这种情况杆上的力一定是沿着杆向里或者沿着杆向外.因为如果不沿杆,则这个杆会发生转动,所以可以判断一定是沿杆方向的.

\subsection{弹力的大小}
弹力的大小跟形变的大小的关系.在弹性限度内,形变越大,弹力也越大;形变消失,弹力就随着消失.对于拉伸形变(或压缩形变)来说,伸长(或缩短)的长度越大,产生的弹力就越大.对于弯曲形变来说,弯曲的越厉害,产生的弹力就越大.对于扭转形变来说扭转的越厉害,产生的弹力就越大.

\subsection{弹力的本质}
 弹力的本质是分子间的作用力.当物体被拉伸或者压缩时,分子间的距离便会发生变化,使分子间的相对位置拉开或者靠拢,这样,分子间的引力与斥力就不会平衡,出现相吸或者相斥的倾向,而这些分子间的吸引或者排斥的总效果,就是宏观上观察到的弹力.如果外力太大,分子间的距离被拉开得太多,分子就会滑进另一个稳定的位置,即使外力除去后,也不能回复到原位,就会保留永久的变形.这就是弹力的本质.

 \subsection{弹力的区别}
 弹力是按照力的性质命名的.而压力,支持力,拉力则是由力的效果命名的.这是两个完全不同的概念.因此,弹力和压力,支持力,拉力之间没有明确的关系.弹力不一定是压力,支持力,拉力.

 例如,套在同一光滑竖直杆的两个环形磁铁,其相同的磁极相对,两个磁铁均处于静止状态.对其中一个磁铁进行受力分析,磁铁受本身的竖直向下的重力作用和竖直向上的排斥力作用,二力为一对平衡力.此时,向上的排斥力便作为支持力.此支持力就不是弹力 .另外,由牛顿第三定律可以得到,大小等于向上的排斥力,方向向下的磁力也作用于下面的磁铁上.此时,这个向下的磁力就是上面的磁铁给它的向下的压力.这个压力也不是弹力.

 又如,在两根光滑平行直导轨间,分布有竖直方向且等间距分布的方向不同的匀强磁场,导轨上有一个宽度与磁场相同的金属框.当磁场匀速运动时,线框就会受到安培力作用而运动起来.此时,安培力就是线框运动的合外力,也就是拉力,但此拉力也不是弹力.

 注意:在这里出现的牛顿第三定律和安培力,在以后会讲到,此处主要为了说明弹力和压力,支持力,拉力之间的区别.基于以上讨论,不能笼统的说,弹力就是压力,支持力,拉力,要具体问题具体分析.
 
 \subsection{胡克定律}
 在\CJKunderwave{弹性限度内},弹簧弹力的大小$F$ 跟弹簧伸长(或缩短)的长度$x$成\CJKunderwave{正比}.即
 \begin{equation}
   F=kx
   \label{eq:hook law a}
 \end{equation}

k:弹簧的劲度系数,反映弹簧本身的属性,由弹簧自身的长度,粗细,材料等因素决定,与弹力$F$ 的大小和形变量$x$ 无关.

x:弹簧的形变量.记弹簧原长为$l_0$ ,发生形变后的长度为$l$,则弹簧的形变量为$x=|l-l_0|$

注意:此处的定义按人教版教材给出,只用式\eqref{eq:hook law a} 来计算弹力的大小,而方向则根据具体情况单独判断.

含方向的胡克定律:设弹簧伸长的方向为正,则$x=l-l_0$ ,于是可得
\begin{equation}
  F=-kx
  \label{eq:hook law b}
\end{equation}

在式\eqref{eq:hook law b} 中,如果弹簧伸长,则$l>l_0$ ,弹力与形变的方向相反,于是$F<0$ ,如果弹簧压缩$l<l_0$ ,则 $F>0$.

\subsection{例题分析}
 \begin{selection}
  1.下列关于弹力的几种说法,正确的是
  A.只要两物体接触就一定产生弹力
  B.静止在水平地面上的物体所受重力就是它对水平面的压力
  C.静止在水平面上的物体受到向上的弹力是因为地面发生了形变
  D.只要物体发生形变就一定有弹力产生

  a.C

  e.两物体接触并发生弹性形变才会产生弹力,A,D错误.静止在水平面上的物体所受重力的施力物体是地球,而压力的施力物体是该物体,受力物体是水平面,两力不同,B错误,C正确.

  2.两个光滑的木板固定为如<:
  \begin{tikzpicture}
    \draw (0,0)--(3,0);
    \draw (1.866,0.5) node {\tiny A} circle [radius=0.5cm];
    \draw [rotate=30] (0,0)--(3,0);
    \draw [pattern=north west lines] (0,-0.2)--(0,0)--(3,0)--(3,-0.2);
    \draw [rotate=30,pattern=north west lines] (0,0.2)--(0,0)--(3,0)--(3,0.2);
  \end{tikzpicture}
  :>的结构,其中下面的木板水平,小球$A$ 放于其中且与两木板都接触,则
  A.两个木板对$A$ 都有弹力作用
  B.只有下面的木板对$A$ 有弹力作用
  C.将图中结构整体逆时针旋转一小角度后,$A$ 球受两木板的弹力作用
  D.将图中结构整体逆时针旋转一小角度后,球$A$ 仅受左边木板的弹力作用

  a.BC

  e.接触但是如果没有形变则不会有弹力.一般的微小形变没法直接看出来,则采用假设法.假设撤去上面的木板,则$A$ 仍会静止不动,但撤去下面的木板,$A$ 会掉下来,故$B$ 正确,同理可得C正确.

  3.在半球形光滑碗内,斜搁一根筷子,如<:
  \begin{tikzpicture}
    \draw (-1,0) arc (180:360:1); 
    \draw (-1,0)--(1,0)node [anchor=north west]{\tiny B};
    \filldraw (0,0) node [anchor=south]{\tiny $O$} circle [radius=1pt];
    \filldraw [black,rotate around={30:(-0.5,-0.866)}] (-0.5,-0.866) rectangle (1.5,-0.766) ;
    \draw (-0.5,-0.866) node [anchor=north east]{\tiny A}--(-0.5,-1.2);
    \draw (0.5,-0.866) --(0.5,-1.2);
    \draw [pattern=north west lines] (-1.5,-1.4)--(-1.5,-1.2)--(1.5,-1.2)--(1.5,-1.4);
  \end{tikzpicture}
  :>所示,筷子与碗的接触点分别为$A$ ,$B$,则碗对筷子$A$ , $B$ 两点处的作用力方向分别为
  A.均竖直向上
  B.均指向球心$O$
  C.$A$ 点处指向球心$O$ ,$B$ 点处竖直向上
  D.$A$点处指向球心$O$,$B$ 处垂直于筷子斜向上

  a.D

  e.碗对筷子$A$ ,$B$ 两点处的作用力属于弹力,而接触处的弹力总是垂直于接触面,因而寻找接触面便成为确定弹力方向的关键.在$A$ 点处,当筷子滑动时,筷子与碗的接触点在碗的内表面(半球面)上滑动,所以在$A$ 点处的接触面是球面在该点的切面,此处弹力与切面垂直,即指向球心$O$.在B点处,当筷子滑动时,筷子与碗的接触点在筷子的下表面上滑动,所以在$B$点处的接触面与筷子平行,此处的弹力垂直于筷子斜向上.

 \end{selection}
 \begin{calculate}
   4.一根轻质弹簧,当它受到$10N$ 的拉力时长度为$12cm$, 当它受到$25N$ 的拉力时长度为$15cm$.问:(注意:弹簧始终在弹性限度内)
   [1]弹簧不受力时的自然长度为多长?
   [2]该弹簧的劲度系数为多大?

   a. $0.1m $ \qquad $500N/m$

   e.设弹簧的劲度系数为$k$,原长$l_0$ ,由\eqref{eq:hook law a}式得
   $$F_1=k(l_1-l_0)$$
   $$F_2=k(l_2-l_0)$$
   (1)以第一式除以第二式,消去$k$得
   $$\cfrac{F_1}{F_2}=\cfrac{l_1-l_0}{l_2-l_0}$$
   解得
   $$l_0=\cfrac{F_1l_2-F_2l_1}{F_1-F_2}=0.1m$$
   (2)以上述第一式减去第二式,消去$l_0$得
   $$k=\cfrac{F_1-F_2}{l_1-l_2}=500N/m$$



 \end{calculate}
