\section{力学的单位制}
\subsection{基本量和基本单位}
在物理学中,选定几个物理量的单位,就能够利用\CJKunderwave{物理量之间的关系}推导出其它物理量的单位,这些被选定的物理量叫做\CJKunderwave{基本量},它们的单位叫做基本单位.

\subsection{导出单位}
由基本量根据\CJKunderwave{物理关系}推导出来的其他物理量的单位,叫做\CJKunderwave{导出单位}.

\subsection{单位制}
由\CJKunderwave{基本单位}和\CJKunderwave{导出单位}组成单位制.

  
\subsection{国际单位}
{\parindent=0pt\hfill
\begin{tabular}{|*{8}{c|}}
  \hline
  基本量 & 长度(l)& 质量 (m) & 时间 (t) & 电流(I) & { \small 热力学温度}(
  T) & {\small 发光强度}(I)& {\small 物质的量}(n)\\
  \hline
  {\small 基本单位} & 米($m$) & 千克 ($kg$) & 秒 ($s$) & 安培 ($A$) & 开尔文 ($K$) & 坎德拉($cd$)&摩尔 ($mol$)\\
  \hline
\end{tabular}
\hfil}

\subsection{例题分析}

\begin{selection}
  1.关于国际单位制,下列说法正确的是
  A.在力学单位制中,若采用$cm$ ,$g$, $s$ 作为基本单位,力的单位是牛($N$)
  B.牛是国际单位制中的一个基本单位
  C.牛是国际单位制中的一个导出单位
  D.$\mbox{千克}\cdot \mbox{米}/\mbox{秒}^2$, $\mbox{焦}/\mbox{米}$ 都属于力的国际单位

  a.CD

  e.力的单位是由牛顿第二定律导出的,当各量均取国际单位时,由$F=ma$ 可知力的单位是质量的单位和加速度的单位的乘积,即 $N=kg\cdot m/s^2$ .若取厘米,克秒作为基本单位,那么根据牛顿第二定律可得: $\mbox{克}\cdot \mbox{厘米}/\mbox{秒}^2 \neq N$.A,B错误,C正确.由功的计算公式 $W=Fl$ ,得 $F=\frac{W}{l}$ . 由于功的国际单位是 $\mbox{焦} $ ,长度的国际单位是 $\mbox{米}$ ,所以力的国际单位也可以表示为 $\mbox{焦} / \mbox{米} $ ,D正确.
  
  2.在解一道计算题时(由字母表达结果的计算题) 一个同学解得位移 $x=\frac{F}{2m}(t_1+t_2)$ , 用单位制的方法检查,这个结果
 A.可能是正确的
 B.一定是错误的
 C.如果用国际单位制,结果可能是正确的
 D.用国际单位制,结果错误,如果用其它单位制,结果可能正确

 a.B

 e.可以将右边的力$F$ ,时间 $t$ 和 质量 $m$ 的单位代入公式看得到的单位是否和位移 $x$ 的单位一致;还可以根据 $F=ma$ , $a=\frac{v}{t} $ , $ v= \frac{x}{t} $ ,将公式的物理量全部换算成基本单位,就容易判断了.在 $x=\frac{F}{2m} (t_1+t_2)$ 式中,左边单位是长度单位,而右边的单位推知是速度单位,所以结果一定是错误的,单位制选的不同,不会影响结果的准确性,故 A,C,D 错,B对.


\end{selection}
