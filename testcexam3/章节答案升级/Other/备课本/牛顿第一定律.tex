\section{牛顿第一定律}
\subsection{牛顿第一定律的内容}
一切物体总保持匀速直线运动状态或静止状态,除非作用在它上面的力\CJKunderwave{迫使}它改变这种状态.
\subsection{对牛顿第一定律的理解}
\begin{enumerate}
  \item 物体保持匀速直线运动状态或者静止状态是有条件的,即物体\CJKunderwave{不受外力}.
  \item 力的作用是\CJKunderwave{改变物体的运动状态}.
  \item 牛顿第一定律揭示了一切物体都具有一种\CJKunderwave{固有属性}---惯性.因此牛顿第一定律又叫做惯性定律.
\end{enumerate}
\subsection{惯性}
\subsubsection{定义}
物体具有保持原来\CJKunderwave{匀速直线运动状态}或\CJKunderwave{静止状态}的性质.
\subsubsection{决定因素}
\CJKunderwave{质量}是衡量惯性的\CJKunderwave{唯一标准},质量大则惯性大,反之则小.
\subsection{例题分析}
\begin{selection}
  1.如果正在做自由落体运动的物体的重力忽然消失,那么它的运动状态应该是
  A.悬浮在空中不动
  B.运动速度逐渐减小
  C.做竖直向下的匀速直线运动
  D.以上三种情况都有可能

  a.C

  e.由题意可知,正在做自由落体的物体一定具有一定的速度,如果重力忽然消失,则它将不受外力作用.由牛顿第一定律,物体将会保持静止或都匀速直线运动状态,所以当重力突然消失时,物体将继续保持这个向下运动的速度而做匀速直线运动,所以C正确.

  2.关于物体的惯性,以下说法正确的是
  A.物体的运动速度越大,物体越难停下来,说明运动速度大的物体惯性大
  B.汽车突然减速时,车上的人向前倾,拐弯时人会往外甩,而汽车匀速前进时,车上的人感觉平稳,说明突然减速和拐弯时人有惯性,匀速运动时没有惯性
  C.在同样大小的力的作用下,运动状态越难改变的物体,其惯性一定越大
  D.在长直水平轨道上匀速运动的火车上,门窗紧闭的车厢内有一人向上跳起后,发现落回到原处,这是因为人跳起后,车继续向前运动,人落下后必定向后偏些,但因时间太短,偏后距离太小,不明显而己

  a.C

  e.质量是衡量惯性的唯一标准,与物体的速度无关,所以A错误;一切物体都有惯性,不论物体处于加速,减速还是匀速,B错误;同样大小的力作用于物体上,运动状态越难改变,说明物体保持原来运动状态的本领越大,则惯性越大,所以C正确;人向上跳起后,人在水平方向不受外力作用,由于惯性,人在水平方向的速度不变,与车速相同,因此仍落在车上原处,D错误.故正确选项只有C.

3. 伽利略在对力和运动的研究中,构想了理想实验,其科学研究方法的核心是
A.把猜想和假说结合起来
B.把实验和逻辑推理结合起来
C.把提出问题与猜想结合起来
D.以上说法均不正确

a.B

e.伽利略的理想实验是建立在可靠的实验事实基础之上的,以抽象为推导,抓住主要因素,忽略次要因素,深刻地揭示了自然规律,则A,C,D错误,只有B正确.

\end{selection}
