\section{匀变速直线运动位移与速度的关系}
\subsection{基本关系推导}

有的时候我们不知道时间,但是知道匀变速直线运动中的$v$和$v_0$,以及加速度$a$,如果要计算位移我们需要先根据$v-t$关系计算出时间,再根据$x-t$关系来求位移,这是一个方法,但是不是最简单的方法.本节将导出$x-v$的关系.

利用\eqref{eq:v-t}式解得时间$t=\cfrac{v-v_0}{a}$,将此时间代入\eqref{eq:displacement}式得:
\begin{equation*}
x=\cfrac{(v+v_0)(v-v_0)}{2a}
\end{equation*}
考虑到平方差公式$(v+v_0)(v-v_0)=v^2-v_0^2$,左右同时乘以$2a$,再移项得:
\begin{equation}
v^2-v_0^2=2ax
  \label{eq:x-v}
\end{equation}
上式就是匀变速直线运动的位移与速度的关系.

\subsection{例题分析}
\begin{calculate}
  1.汽车以$10m/s$ 的速度行驶,刹车的加速度大小为$3m/s^2$ ,求它向前滑行$12.5m$ 后的瞬时速度.

  a.$5m/s$ ,方向与初速度方向相同.

  e.当汽车做刹车运动时,它的速度是不会改变方向的,我们称之为不可逆类问题.设初速度方向为正,则$v_0=10m/s$,$a=-3m/s^2$,$x=12.5m$
  \newline
  法一:由匀变速直线运动位移与速度的关系\eqref{eq:x-v}式得
  $$v=\sqrt{v_0^2+2ax}=\sqrt{10^2+2\times(-3)\times12.5}m/s=5m/s$$
  注意,由于汽车的速度不可能改变方向,所以我们可以判断末速度为正.即汽车向前滑行$12.5m$ 后的瞬时速度大小为$5m/s$,方向与初速度方向相同.
  \newline
  法二:由匀变速直线运动位移与时间的关系\eqref{eq:x-t}式解得
  $$t=\cfrac{-v_0+\sqrt{v_0^2+2ax}}{a}=\cfrac{5}{3}s$$
  再由匀变速直线运动速度与时间的关系\eqref{eq:x-t}式得
  $$v=v_0+at=10+(-3)\times\cfrac{5}{3}m/s=5m/s$$
  \newline
  注意:法一和法二相比更加简单,而法二显然走了弯路.所以在匀变速直线运动问题中,如果根据已知条件选用适当的公式则可以使问题大为简化,所以同学们遇到问题时应当尽量考虑一题多解,从而能够获得选用适当解题方法的能力.

\end{calculate}

\subsubsection{数学补充--求根公式}

在上述题目的第二种解法中用到了一元二次方程的求根公式,依然为了同学们学习的连惯性,这里做详细的证明.如下

一元二次方程为
$$ax^2+bx+c=0$$
上式提出a,并将第二项加入2,变成如下形式
$$a(x^2+2\cfrac{b}{2a}x)+c=0$$
在上式括号内加入$(\cfrac{b}{2a})^2$,然后再减去它则方程不变
$$a(x^2+2\cfrac{b}{2a}x+\cfrac{b^2}{4a^2})+c-\cfrac{b^2}{4a}=0$$
上式中圆括号内为完全平方式,同时将圆括号外的部分移到等号右侧,方程左右同时除以$a$
$$(x+\cfrac{b}{2a})^2=\cfrac{b^2-4ac}{4a^2}$$
记右式分子部分为$\Delta = b^2-4ac$,$\Delta$就是判别式,如果$\Delta <0$则此方程无解,如果$\Delta>0$则对上式开方得
$$x+\cfrac{b}{2a}=\pm \cfrac{\sqrt{\Delta}}{2a}$$
移项得
$$x=\cfrac{-b\pm \sqrt{\Delta}}{2a}$$
上式就是一元二次方程的求根公式.

所以由\eqref{eq:x-t}来求时间时就有:
$$x=v_0t+\cfrac{1}{2}at^2$$
上式移项则相当于关于时间$t$的一元二次方程
$$\cfrac{1}{2}at^2+v_0t-x=0$$
用求根公式解得
$$t=\cfrac{-v_0\pm\sqrt{v_0^2+2ax}}{a}$$
由于时间$t$永远大于零,所以只能取根
$$t=\cfrac{-v_0+\sqrt{v_0^2+2ax}}{a}$$
