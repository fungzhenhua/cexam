\section{重力}
\subsubsection{定义}
由于\CJKunderwave{地球的吸引} 而使物体受到的力叫做重力.但是应当注意,地球对物体的吸引力不等价于重力,因为地球在自转,所以地球对物体的吸引力有两个作用,其一保持地球上的物体随地球一同绕地轴做匀速圆周运动(后面会讲到),其二,使物体下落.这个使物体下落的力叫做重力.地球对物体的吸引力叫做万有引力(后面会讲到),重力是万有引力的一个分力.

\subsubsection{计算式}
物体由于只受重力作用而产生的加速度叫做重力加速度,记作:$g=9.8m/s^2$.
则重力的大小为
$$G=mg$$

注意:由于后面会讲到万有引力,其表达式为$F=G\cfrac{m_1m_2}{r^2}$,它里面有一个$G$,所以在不引起误解的情况下表示重力一般用表达式$mg$. 同时,在初中时大家学过这个公式,其中$g$ 当时只说明是一个比例系数,同时单位是$N/kg$ ,容易说明 $N/kg=m/s^2$,这需要从牛顿第二定律(后面会讲到)来看.

\subsubsection{方向}
重力也是一个矢量,它的方向总是\CJKunderwave{竖直向下}.但是注意,竖直向下不是指向地心,万有引力的方向才指向地心,重力只是万有引力的一部分,所以不能说重力的方向是指向地心的.

\subsubsection{作用点}

物体的每一部分都受重力,但是在计算的时候可以找到一个等效作用点,即:认为物体所受重力作用点集中在一个点上,这个等效作用点就叫做重心.所以也可以说重力的作用点在重心上.计算重心要用到微积分,在高中阶段大家的数学水平尚不够充分,所以只要大家暂时记住会用便好.

根据计算,物体的重心取决于物体的质量分布、形状等因素,一个质量分布均匀、形状规则的物体,重心就是物体的几何中心,但是几何中心却不一定在物体上,比如一个圆环,它的重心在圆心上,但是圆心却不在圆环上.

\subsection{例题分析}
\begin{selection}
  1.下列说法正确的是
  A.自由下落的石块速度越来越大,说明石块所受重力越来越大
  B.在空中飞行的物体不受重力
  C.一抛出的石块轨迹是曲线,说明石块所受的重力方向始终在改变
  D.将一石块竖直向上抛出,在先上升后下降的整个过程中,石块所受重力的大小与方向都不变

  a.D

  e.在地球上的同一位置,同一物体的重力为一定值,故A错;只要在地球上,物体所受重力就不零,故B错;重力的方向始终竖直向下,与物体的运动状态无关,故C错误.

  2.关于重心及重力,下列说法正确的是
  A.一个物体放于水中称量时弹簧测力计的求数小于物体在空气中称量时弹簧测力计的示数,因此物体在水中受到的重力小于在空气中受到的重力
  B.据$G=mg$ 可知,两个物体相比较,质量较大的物体的重力一定较大
  C.物体放于水平面上时,重重力方向垂直于水平面向下,当物体静止于斜面上时,其重力方向垂直于斜面向下
  D.物体的形状改变后,其重心位置往往改变

  a.D

  e.由于物体浸没于水中时,受到向上的浮力从而减小了弹簧的拉伸形变,弹簧测力计的拉力减小了,但物体的重力并不改变,选项A错误.当两物体所处的地理位置相同时,$g$ 值相同,质量大的物体的重力必定大,但当两物体所处的地理位置不同时,如质量较小的物体放在两极,但是质量较大的物体放在赤道上,由于$g$ 值不同,质量较大的物体的重力不一定较大,选项B错误.重力的方向是竖直向下的,而不是垂直向下的,选项C错误.物体的重心位置由物体的形状和质量分布情况共同决定 ,当物体的形状改变时,其重心往往发生改变,故选项D正确.

  3.关于重力、重心,下列说法正确的是
  A.风筝升空后,越升越高,说明风筝的重心相对于风筝的位置也越来越高
  B.质量分布均匀、形状规则的物体的重心一定在物体上
  C.舞蹈演员在做各种优美动作时,其重心相对身体的位置不断变化
  D.重力的方向总是坚直向下

  a.CD

  e.物体形状不变时,重心相对物体的位置不变,如果物体形状变化了,重心相对物体的位置改变,所以A错误,C正确.质量分布均匀、形状规则的物体的重心在其几何中心上,但不一定在物体上.如足球,B错误.重力的方向竖直向下,D正确.

  4.一个物体所受的重力在下列情形下要发生变化的有
  A.把它从赤道拿到南极
  B.把它送到月球上去
  C.把它从地面上浸入水中
  D.把它置于向上加速的电梯内

  a.AB

  e.由$G=mg$ 可知,物体质量不变,当重力加速度$g$ 发生变化时,重力$G$ 随之改变.由于地球两极的$g$ 值大于赤道上的$g$ 值,地球上的$g$ 值大于月球上的$g$ 值,所以选项A、B正确.由于重力大小与物体所处的环境和运动状态无关,所以选项C、D两种情况下物体的重力没有发生变化.

 4.如<:
 \begin{tikzpicture}
   \draw [pattern=north west lines] (-1,0.2)--(-1,0)--(1,0)--(1,0.2); 
   \draw (0,0)--(0,-1.5);
   \draw (0,-2.2) circle [radius=0.7];
   \draw [pattern=north west lines] (0,-2.2) circle [radius=0.5];
   \filldraw [color=black] (-0.06,-2.9) rectangle (0.06,-2.7);
 \end{tikzpicture}
 :>所示,一个空心均匀球壳里面注满水,球的正下方有一个小孔,在水由小孔慢慢流出的过程中,空心球壳和水的共同重心将会
 A.一直下降
 B.一直上升
 C.先升高后降低
 D.先降低后升高

 a.D

 e.当球壳中水满的情况时,重心在球心处,但是当水慢慢流出时,重心由对称性可知将会下降,但是当水流完之后,球壳的质量分布将再次均匀,所以重心还是在球心处,所以重心的变化必定是先降低再升高.

\end{selection}

