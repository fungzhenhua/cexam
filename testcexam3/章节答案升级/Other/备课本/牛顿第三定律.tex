\section{牛顿第三定律}
\subsection{作用力与反作用力}

根据力的定义,力是物体与物体间的相互作用,所以每一个力都会涉及两个物体.施力物体对受力物体的力叫做作用力,施力物体同时也受到受力物体对它的作用力,叫做反作用力.由于相互作用的两个物体谁当施力物体,谁当受力物体要看所研究的问题,研究对象是谁,谁就是受力物体.所以作用力与反作用力我们又称为相互作用力.

作用力与反作用力总是相互依存,同时存在,如果其中一个叫做作用力,则另一个则叫做反作用力.

\subsection{牛顿第三定律的内容}
\CJKunderwave{两个物体}之间的\CJKunderwave{作用力}和\CJKunderwave{反作用力}总是\CJKunderwave{大小相等,方向相反,作用在同一条直线上}

\subsection{作用力与反作用力间的关系}
牛顿第三定律是研究了作用力与反作用力的关系,这一小节作一个细致的描述
\begin{enumerate}
  \item 同大小:即作用力与反作用力大小相同.
  \item 同直线:即作用力与反作用力作用在同一直线上.
  \item 同存亡:即作用力与反作用力同时产生,同时变化,同时消失.
  \item 同性质:即作用力与反作用力的性质必然相同.例如:作用力是引力,则反作用力也一定是引力;作用力是弹力,则反作用力也是弹力.
  \item 异向:作用力与反作用力方向相反.
  \item 异体:作用力作用在受力物体上,反作用力作用在施力物体上.
  \item 异效:由于作用力与反作用力作用在不同的物体上,但是不同的物体有不同的属性,也可能各自还受到其它不同的外力,因此作用力与反作用力在不同的物体上分别产生不同的作用效果,并且不能相互抵消.
\end{enumerate}

\subsection{例题分析}
\begin{selection}
  1.一匹马拉着车前行,关于马拉车的力和车拉马的力的大小关系,下列说法中正确的是
  A.马拉车的力总是大小车拉马的力
  B.马拉车的力总是等于车拉马的力
  C.加速运动时,马拉力的力大于车拉马的力
  D.减速运动时,马拉力的力小于车拉马的力

  a.B

  e.马向前拉车的力和车向后拉马的力是一对作用力与反作用力,按牛顿第三定律,它们总是大小相等,方向相反,作用的同一条直线上,与运动状态无关.所以 A,C,D错误,B正确.


  2.质量为$M$ 的人站在地面上,用绳子通过定滑轮将质量为$m$ 的重物从高处放下,如
  <:
  \begin{tikzpicture}
    \draw [pattern=north west lines](-1,1.2)--(-1,1)--(1,1)--(1,1.2);
    \draw (0,0) circle [radius=0.5];
    \draw (0,0) --(0,1);
    \draw (-0.5,0) -- (-0.5,-1.8);
    \draw (0.5,0)--(0.5,-1.5);
    \draw (-0.7,-1.8) rectangle (-0.3,-2.2);
    \draw (-0.5,-2) node {\small $m$};
    \draw [->,>=stealth] (-1,-0.5)--(-1,-1.5) node [anchor=east] {\small $a$};
    \draw (1,-1) circle [radius=2mm];
    \draw (0.5,-1.5)--(1,-1.2);
    \draw (1,-1.2) -- (0.5,-2.5);
    \draw (0.5,-1.5) -- (0.9,-1.5)--(1.2,-2.5);
    \draw[pattern=north west lines] (-1.5,-2.7)--(-1.5,-2.5)--(1.5,-2.5)--(1.5,-2.7);
  \end{tikzpicture}
  :>所示,若重物以大小为$a$的加速度加速下降($a<g$),则人对地面的压力大小为
  A.$(M+m)g-ma$
  B.$M(g-a)-ma$
  C.$(M-m)g+ma$
  D.$Mg-ma$

  a.C

  e.如<:
  \begin{tikzpicture}
    \draw [pattern=north west lines](-1,1.2)--(-1,1)--(1,1)--(1,1.2);
    \draw (0,0) circle [radius=0.5];
    \draw (0,0) --(0,1);
    \draw (-0.5,0) -- (-0.5,-1.8);
    \draw (0.5,0)--(0.5,-1.5);
    \draw (-0.7,-1.8) rectangle (-0.3,-2.2);
    \draw (-0.5,-2) node {\small $m$};
    \draw [->,>=stealth] (-1,-0.5)--(-1,-1.5) node [anchor=east] {\small $a$};
    \draw (1,-1) circle [radius=2mm];
    \draw (0.5,-1.5)--(1,-1.2);
    \draw (1,-1.2) -- (0.5,-2.5);
    \draw (0.5,-1.5) -- (0.9,-1.5)--(1.2,-2.5);
    \draw[pattern=north west lines] (-1.5,-2.7)--(-1.5,-2.5)--(1.5,-2.5)--(1.5,-2.7);
    \draw[->,>=stealth] (-0.5,-2)--(-0.5,-2.5) node [anchor= south east]{\small $mg$};
    \draw[->,>=stealth] (-0.5,-2)--(-0.5,-1.3) node [anchor=east]{\small $T$};
    \draw[->,>=stealth] (0.9,-1.5)--(0.9,-2.2) node [anchor=south west] {\small $Mg$};
    \draw[->,>=stealth] (0.9,-1.5)--(0.9,-1) node [anchor=west] {\small $F_N$};
    \draw[->,>=stealth] (0.9,-1.5)--(0.9,-0.7) node [anchor=east] {\small $T$};
    \filldraw[black] (-0.5,-2) circle [radius=1pt];
    \filldraw[black] (0.9,-1.5) circle [radius=1pt];
  \end{tikzpicture}
  :>所示.设绳子的拉力为$T$,对重物,由牛顿第二定律知
  $$mg-T=ma$$
  所以解得
  $$T=m(g-a)$$
  对人受力分析,受重力,绳的拉力及地面对人的支持力而平衡,则
  $$F_N+T-Mg=0$$
  解得
  $$F_N=Mg-T=(M-m)g+ma$$
  据牛顿第三定律得,人对地面的压力大小也为
  $$F_N^\prime =(M-m)g+ma$$

 2.下列判断正确的是
 A.人行走时向后蹬地,给地面向后的摩擦力,地面给人的摩擦力是人向前的动力
 B.人匀速游泳时,人对水向前用力,水给人的力是阻力,方向向后
 C.放在桌面上的物体,因有重力,才有对桌面的压力,才有桌面的支持力出现,即压力先产生,支持力后产出现
 D.作用力与反作用力,应是先有作用力,再有反作用力,作用力先变化,反作用力随后跟着做相应的变化

 a.A

 e.人走路或游泳时,对地或对水都施加向后的力,另一方给人施加动力,故A对,B错误;作用力与反作用力总是同时产生,同时变化的,不存在谁先谁后,故C,D均错.

\end{selection}
