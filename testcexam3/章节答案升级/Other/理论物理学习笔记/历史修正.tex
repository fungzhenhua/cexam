\section{量子统计中的统计矩阵}
\begin{gather}
\intertext{统计力学中,我们考虑的一个系统通常不可能是绝对孤立的,所以只能把实际的系统看成是一个大的闭合系统的一个子系统.那么这一个子系统不可避免的与外界有相互作用,所以它就没有自己的波函数,但是如果子系统绝对孤立,则它有自己的波函数,而且能量是稳定的,则可以确定它的一系列能量表象下的本征态$\psi_n$,对应本征值$E_n$.但是作为与界有相互作用的系统,则它只能用整体的波函数来描述$\Psi$,而这个波函数可以用孤立的系统本征态来展开,则}
\Psi(x,q)=\sum_n C_n(q)\psi_n(x)
\intertext{任一属于子系统的物理量$f$的平均值为}
\overline{f}=\int \Psi^*(x,q)\hat{f}\Psi(x,q)dqdx\\
\overline{f}=\sum_{mn}\int C^*_n(q)C_m(q)dq \cdot \int \psi^*_n(x) \hat{f}\psi_m(x)dx\\
\overline{f}=\sum_{mn} \omega_{mn}f_{nm}
\intertext{由于系统稳定存在,同时物理量的观测值也不随时间变化,所以它们都与Hamilton量对易,于是上式可以表达为}
\overline{f}=\sum_{n} \omega_{nn}f_{nn}
\intertext{至此还没有应用统计的考虑,由于随时间波函数演化,则考虑所有可能的波函数情况,再取平均,就是系综平均,同时记系综平均后的$\omega_{nn}$为$\omega_n$则}
\overline{f}=\sum_{n} \omega_{n}f_{nn}
\end{gather}

\section{经典和量子体系的态密度}

\subsection{经典情况}

在经典力学中,可以同时指明系统的动量和坐标,同时有刘维定理,则代表点密度不随时间变化,这也就赋予了代表点密度以统计概率的意义.同时由于子系统与外界不断的有微弱的相互作用,不能保证系统具备一个确定的能量,只能是满足某一个小范围的能量.所以这个代表点在相同的能量时有很多个,在不同有能量但是能量差距很小的时候也有很多个.所以可以提在某一相格内有多少代表点的问题.同时还必须满足
\begin{equation}
  \int \rho dpdq=1
\end{equation}
但是这个代表点密度必须满足能量约束$E\sim E+\Delta E$,则原来的$2s$自由度就会减少,所以当对$2s$个广义坐标积分如果$\rho$有限,则积分值为零.因此可以推断
\begin{equation}
  \rho =\mbox{常数} \cdot \delta(E-E_0)
\end{equation}

\subsection{量子情况}

由于子系统与外界不断有相互作用,同时统计力学所研究的对象是大量的粒子构成的系统,那么一个量子态能量和总能量的差距是相当大的.宏观能量可以有无数的能量分配方式,同时能量接近的态也是允许的,因为宏观物体的能量允许一个能量范围$E\sim \Delta E$,这就类似于量子力学中的``简并'',但是又不同,因为能量不是严格相等的在统计中也是允许的.但是量子力学不能确定在这些``准简并''态中每一个可能的量子态的出现的概率,这就需要统计的考虑,无限小能量范围内的量子态可以认为是等概率出现的,因为没有理由认为哪一个量子态是特别的,一般于能量范围的量子态出现的概率是相当大的,所以就要求能量$E_n$的态$\psi_n$出现的概率呈现$\delta$函数的性质,又基于``能级''无限稠密,可以得出一个量子态出现的概率为
\begin{equation}
  d\omega =\mbox{常数}\cdot \delta(E-E_0)d\Gamma(E)
\end{equation}
