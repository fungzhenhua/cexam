2018年我参加了研究生考试,成绩出来后较去年有所提升,但是仍然不理想,勉强达到可调剂的分数.但是我没有选择调剂,综合考虑后我还是执著的坚持再考虑一年.在2018的的考研准备工作中,由于要同时任教高中四个班级的物理课,我没有大量的时间像在校大学生一样做题,这是我不够熟练的第一个原因.同时,对于统计物理学概念理解的不深入,对于电动力学的理解也不是足够深入,这是第二个原因.英语一直是我的弱项,多少年来,我一直努力去学习它,但是收效甚微,可能是我对于这个学科不是太过敏感吧,在2019年的工作中,我一定要坚持下来学习英语,以期在2019年,顺利通过考试.

在学习的过程中,有些概念和计算理解起来有一定的难度,这也是学习过程中的核心,所以我计划在学习过程中按照自己的理解,使用\LaTeXe{}整理下来,以方便我今后的学习.在参考了大量的书籍后,我发现朗道的《理论物理教程》共十卷,很适合我的要求,所以坚持以它为主教材,然后在精读一遍后再做一些习题来应对2019年的研究生考试,不怕任何困难,相信自己一定能够实现自己的理想.
