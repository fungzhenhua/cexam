\chapter{化学反应}

\section{化学平衡条件}

在朗道书第267页,第三段中列举了化学反应
\begin{equation}
  2H_2+O_2=2H_2O
  \label{eq:HOfanying}
\end{equation}
在将其写成热力学表达式时,朗道书中写的是
\begin{equation}
  2H_2+O_2-2H_2O=0
\end{equation}
由于每一个化学反应都可以在正逆两个方向发生,所以朗道的这个写法按规定就是水的分解,其中$H_2$和$O_2$是作为生成物写出的.但是在汪志诚《热力学和统计物理学》中 \S 4.5 化学平衡条件 一节中写成了式
\begin{equation}
  2H_2O-2H_2-O_2=0
\end{equation}
此式写的是$H_2$ 和 $O_2$ 的化合反应.我最初的教材是汪志诚的书,由于这两个不同的写法,导致我在2019年6月30号计算朗道书第270页习题1时,出现了困难,我的计算结果不能与朗道书中的结果对应起来,这个麻烦是在7月1号上班后,仔细对比后确定出了两本书的不同.
