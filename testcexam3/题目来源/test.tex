\documentclass[a4paper,fontset = windows]{ctexbook}
\usepackage{xifthen}
\usepackage{calc}
\usepackage{graphicx}
\usepackage{tikz}
%\usepackage[user=teacher,option=random,sourcecolor=blue]{cexam}
\usepackage[user=teacher,option=random,source=on,sourceyear=off,sourcestar=off,sourcecolor=red]{cexam}
\usepackage[fontwarning=on]{ctrlwarning}

\begin{document}
\chapter{选择题选项排版测试}

\section{第一节题目}
 \begin{calculations}
    11.\source[3]{2021}{德州一模}从斜面上某一位置每隔0.1s释放一颗小球,在连续释放几颗后,对斜面上正在运
    动着的小球拍下部分照片,如图所示。现测得$x_{AB}=15cm$,$x_{BC}=20cm$,已知小球在斜面上做匀加速直线运动,且加速度大小相同。
    \qitem[1] 求小球的加速度。
    \qitem[3] D、C两球相距多远?
    \qitem[4] A球上面正在运动着的小球共有几颗?
    \qitem[2] 求拍摄时B球的速度。

    e.所以第 \refitem[2] 小问的解析为:求拍摄时B球的速度。

    11.从斜面上某一位置每隔0.1s释放一颗小球,在连续释放几颗后,对斜面上正在运
    动着的小球拍下部分照片,如图所示。现测得$x_{AB}=15cm$,$x_{BC}=20cm$,已知小球在斜面上做匀加速直线运动,且加速度大小相同。
    \qitem[2] 求拍摄时B球的速度。
    \qitem[1] 求小球的加速度。
    \qitem[3] D、C两球相距多远?
    \qitem[4] A球上面正在运动着的小球共有几颗?

    e.所以第 \refitem[2] 小问的解析为:求拍摄时B球的速度。

 \end{calculations}

\section{第二节题目}
 \begin{calculations}
    11.从斜面上某一位置每隔0.1s释放一颗小球,在连续释放几颗后,对斜面上正在运
    动着的小球拍下部分照片,如图所示。现测得$x_{AB}=15cm$,$x_{BC}=20cm$,已知小球在斜面上做匀加速直线运动,且加速度大小相同。
    \qitem[1] 求小球的加速度。
    \qitem[2] 求拍摄时B球的速度。
    \qitem[4] A球上面正在运动着的小球共有几颗?
    \qitem[3] D、C两球相距多远?

    e.所以第 \refitem[2] 小问的解析为:求拍摄时B球的速度。

    11.从斜面上某一位置每隔0.1s释放一颗小球,在连续释放几颗后,对斜面上正在运
    动着的小球拍下部分照片,如图所示。现测得$x_{AB}=15cm$,$x_{BC}=20cm$,已知小球在斜面上做匀加速直线运动,且加速度大小相同。
    \qitem[1] 求小球的加速度。
    \qitem[4] A球上面正在运动着的小球共有几颗?
    \qitem[2] 求拍摄时B球的速度。
    \qitem[3] D、C两球相距多远?

    e.所以第 \refitem[2] 小问的解析为:求拍摄时B球的速度。

    11.从斜面上某一位置每隔0.1s释放一颗小球,在连续释放几颗后,对斜面上正在运
    动着的小球拍下部分照片,如图所示。现测得$x_{AB}=15cm$,$x_{BC}=20cm$,已知小球在斜面上做匀加速直线运动,且加速度大小相同。
    \qitem A球上面正在运动着的小球共有几颗?
    \qitem 求拍摄时B球的速度。
    \qitem 求小球的加速度。
    \qitem D、C两球相距多远?

    e.所以第 \refitem[2] 小问的解析为:求拍摄时B球的速度。

 \end{calculations}


 \end{document}
