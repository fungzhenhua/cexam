\documentclass[a4paper,fontset = windowsnew]{ctexbook}
\usepackage{graphicx}
%\usepackage{tikz}
\usepackage[user=teacher]{cexam}
%\usepackage{xcolor}

\begin{document}
\chapter{运动的描述}
\section{运动学基本概念}
 \lettersink[2cm][2pt][cyan!50!magenta!10!yellow]
 物理学的发展,推动了工业、农业和信息技术等方面的进步,引发了一次次的产业革命,改变了人类的生产和生活方式。技术的进步又为物理学的研究提供了更为强大的手段,并引发了人们对物理问题进行更深入的思考,从而反过来促进物理学的发展。
 创立于17世纪的牛顿力学,被广泛地应用于工程技术,大大推动了社会的发展。18 ~ 19世纪,工程上对蒸汽机的改进需求,又迫使人们对热的问题进行深入研究,引发了热力学的巨大进步。
  19 ~20世纪初,电磁学的发展,直接导致发电机和无线电通信的诞生,使电能被广泛利用。电走进了千家万户,世界被电灯点亮,电话和电报把各地的人们连接起来,人类从此进入了电气时代。


\end{document}
