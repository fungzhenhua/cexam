\documentclass[a4paper,fontset = windowsnew]{ctexbook}
\usepackage{xifthen}
\usepackage{calc}
\usepackage{graphicx}
\usepackage{tikz}
\usepackage{cexam}

\begin{document}
\chapter{行数统计程序}

\parindent=0pt


行数统计程序的latex3改写

\ExplSyntaxOn

      \cexam_type_ii:nnnnnnnnn 
      {r}
      {
	\begin{tikzpicture}
	  \draw (0,0) rectangle (3,4.7);
	\end{tikzpicture}
      }
%      {10pt}{0pt}{0pt}{0pt}{20pt}{0pt}    
      {0pt}{5pt}{0pt}{0pt}{20pt}{0pt}    
    {
	这是第三版的图文试题排版测试
	这是第三版的图文试
	\begin{equation}
	  e=mc^2
	\end{equation}
	这是第三版的图文试题排版测试
	这是第三版的图文试题排版测试
	\begin{equation}
	  e=mc^2
	\end{equation}
	这是第三版的图文试题排版测试
	这是第三版的图文试题排版测试
	这是第三版的图文试题排版测试
	这是第三版的图文试题排版测试
	\begin{equation}
	  e=mc^2
	\end{equation}
	这是第三版的图文试题排版测试
	这是第三版的图文试题排版测试
	这是第三版的图文试题排版测试
	这是第三版的图文试题排版测试
	这是第三版的图文试题排版测试
	这是第三版的图文试题排版测试
	这是第三版的图文试题排版测试
	\begin{equation}
	  e=mc^2
	\end{equation}
	这是第三版的图文试题排版测试
	这是第三版的图文试题排版测试
	这是第三版的图文试题排版测试
      }
      \newline
      这是附加的选项部分.
      这是附加的选项部分.
      这是附加的选项部分.
      这是附加的选项部分.
      这是附加的选项部分.

    \newpage
    本页是第二种排版方式测试. 

    \par

    \cexam_type_i:nnnnnnn 
    {l}
      {
	\begin{tikzpicture}
	  \draw (0,0) rectangle (3,3);
	\end{tikzpicture}
      }
      {10pt}{0pt}{0pt}{0pt}
      {
	这是第三版的图文试题排版测试
	这是第三版的图文试题排版测试
	这是第三版的图文试题排版测试
	\begin{equation}
	  e=mc^2
	\end{equation}
	这是第三版的图文试题排版测试
	这是第三版的图文试题排版测试
	这是第三版的图文试题排版测试
	这是第三版的图文试题排版测试
	这是第三版的图文试题排版测试
	这是第三版的图文试题排版测试
	这是第三版的图文试题排版测试
	这是第三版的图文试题排版测试
	这是第三版的图文试题排版测试
	\begin{equation}
	  e=mc^2
	\end{equation}
	这是第三版的图文试题排版测试
	这是第三版的图文试题排版测试
	这是第三版的图文试题排版测试
	这是第三版的图文试题排版测试
	这是第三版的图文试题排版测试
	这是第三版的图文试题排版测试
	\begin{equation}
	  e=mc^2
	\end{equation}
	这是第三版的图文试题排版测试
	这是第三版的图文试题排版测试
	这是第三版的图文试题排版测试
	这是第三版
	这是第三版的图文试题排版测试
      }
      \newline
      这是附加的选项
      这是附加的选项
      这是附加的选项
      这是附加的选项
      这是附加的选项
      这是附加的选项
      这是附加的选项


    \end{document}
