%  \section{\pkg{doc}宏包中定义的基本命令}
% 
% \subsection{General conventions}
% 
% \begin{function}[added=2018-03-26]{macrocode}
%   \begin{syntax}
%     %    \tn{begin}\{macrocode\}
%   and after this part a line
%     %    \tn{end}\{macrocode\}
%   \end{syntax}
%  On the other hand,if the documentation of these macros is to be produced,the 'definition parts' should be typeset verbatim. To achieve this,these parts are surrounded by the macrocode environment. More exactly:before a 'definition part' there should be a line containing
%  
% \hspace*{\MacroIndent}\verb*+%    \begin{macrocode}+\\
% and after this part a line\\
% \hspace*{\MacroIndent}\verb*+%    \end{macrocode}+\\
%  
% There must be {\em exactly\/} four spaces between the |%|
% and |\end{macrocode}| --- \TeX{} is looking for this string
% and not for the macro while processing a `definition part'.
%  
%  Inside a 'definition part' all \TeX{} commands are allowed;even the percent sign could be used to suppress unwanted spaces etc.
%  
%   \end{function}
%  
%  \begin{function}[added=2018-03-26]{macrocode*}
%   Instead of the \pkg{macrocode} environment one can also use the \pkg{macrocode*} environment which produces the same results except that spaces are printed as \nopagebreak\verb*+ +  characters.
%   \end{function}
%  
% \subsection{Describing the usage of new macros}
%  
%  这一节中我把部分命令用ctxdoc宏包中的方式作了格式上的修改,以方便统一格式。但是在修改过程中发现,function 环境只能编辑一个命令或环境,而在一段文字中描述多外相关的命令或者环境时 function环境不如 |\DescribeMacro|和|\DescribeEnv|如用,为了保证能够根据此文档顺利使用dtx文学化编程,这里保留部分英文文档ltxdoc.dtx中的源码方便学习和分析。
% 
% \begin{function}[added=2018-03-28]{\DescribeMacro}
%   \begin{syntax}
%     \tn{DescribeMacro}\Arg{\tn{foo}}
%   \end{syntax}
%
% When you describe a new macro you may use |\DescribeMacro| to
% indicate that at this point the usage of a specific macro is
% explained. It takes one argument which will be printed in the margin
% and also produces a special index entry.  For example, I used
% |\DescribeMacro{\DescribeMacro}| to make clear that this is the
% point where the usage of |\DescribeMacro| is explained.
% \end{function}
%
%
% \begin{function}[added=2018-03-28]{\DescribeEnv}
%   \begin{syntax}
%     \tn{DescribeEnv}\Arg{foo}
%   \end{syntax}
% An analogous macro |\DescribeEnv| should be used to indicate
% that a \LaTeX{} environment is explained. It will produce a somewhat
% different index entry. Below I used |\DescribeEnv{verbatim}|.
% \end{function}
%
% \begin{function}[added=2018-03-28]{verbatim}
%   \begin{syntax}
%      \tn{begin}\{verbatim\}
%	your macrocode and text.
%      \tn{end}\{verbatim\}
%   \end{syntax}
% It is often a good idea to include examples of the usage of new macros
% in the text. Because of the |%| sign in the first column of every
% row, the \textsf{verbatim} environment is slightly altered to suppress
% those  characters.
% \end{function}
%
% \begin{function}[added=2018-03-28]{verbatim*}
%   \begin{syntax}
%      \tn{begin}\{verbatim*\}
%	your macrocode and text.
%      \tn{end}\{verbatim*\}
%   \end{syntax}
% The \textsf{verbatim$*$} environment is changed in the same way.
% \end{function}
%
% \begin{function}[added=2018-03-28]{\verb}
%   \begin{syntax}
%     \tn{verb}\{foo\}
%   \end{syntax}
% The "\verb" command is re-implemented to give an error report if a
% newline appears in its argument.
% The \textsf{verbatim} and \textsf{verbatim$*$} environments set text
% in the style.
% \end{function}
%  
% \subsection{Describing the definition of new macros}
%
% \begin{function}[added=2018-03-28]{macro}
% To describe the definition of a new macro we use the \textsf{macro}
% environment. It has one argument: the name of the new  macro.
% This argument is also used to print the name in the margin and to
% produce an index entry.
% Actually the index entries for usage and definition are different to
% allow an easy reference.
% This environment might be nested. In this case the
% labels in the margin are placed under each other.
% There should be some text---even  if it's just an empty
% "\mbox{}"---in this environment before "\begin{macrocode}" or the
% marginal label won't print in the right place.
% \end{function}
%  
% \DescribeMacro\MacrocodeTopsep
% \DescribeMacro\MacroTopsep
% There also exist four style parameters: |\MacrocodeTopsep| and
% |\MacroTopsep| are used to control the vertical spacing above
% and below the \textsf{macrocode} and the \textsf{macro}
% \DescribeMacro\MacroIndent
% environment, |\MacroIndent| is used to indent the lines of code
% and
% \DescribeMacro\MacroFont 
% |\MacroFont| holds the font and a possible size change command
% for the code lines, the "verbatim"["*"] environment and the macro
% names printed in the margin.  If you want
% to change their default values in a
% class file (like \texttt{ltugboat.cls}) use the |\DocstyleParms|
% command described below. Starting with release 2.0a it can now
% be changed directly as long as the redefinition happens before
%
%
% \begin{function}[added=2018-03-28]{\SpecialEscapechar}
% If one defines complicated macros it is sometimes necessary to
% introduce a new escape character because the `|\|' has got a
% special |\catcode|. In this case one can use
% |\SpecialEscapechar| to indicate which character is actually
% used to play the r\^ole of the `|\|'. A scheme like this is
% needed because the \textsf{macrocode} environment and its counterpart
% \textsf{macrocode$*$} produce an index entry for every occurrence of a
% macro name. They would be very confused if you didn't tell them that
% you'd changed |\catcode|$\,$s.  The argument to
% |\SpecialEscapechar| is a single-letter control sequence, that
% is, one has to use "\|" for example to denote that `\verb+|+'
% is used as an escape character. |\SpecialEscapechar| only
% changes the behavior of the next \textsf{macrocode} or
% \textsf{macrocode$*$} environment.
%
%  The actual index entries created will all be printed with |\|
% rather than \verb+|+, but this probably reflects their usage, if not
% their definition, and anyway must be preferable to not having any
% entry at all.  The entries {\em could\/} be formatted appropriately,
% but the effort is hardly worth it, and the resulting index might be
% more confusing (it would certainly be longer!).
% \end{function}
%
% \subsection{Cross-referencing all macros used}
%
% \DescribeMacro\DisableCrossrefs 
% \DescribeMacro\EnableCrossrefs 
% As already mentioned, every new macro name used within a
% \textsf{macrocode} or \textsf{macrocode$*$} environment will produce
% an index entry. In this way one can easily find out where a specific
% macro is used.  Since \TeX{} is considerably slower when it has to
% produce such a bulk of index entries one can turn off this feature
% by using |\DisableCrossrefs| in the driver file. To turn it on again
% just use |\EnableCrossrefs|.\footnote{Actually, \texttt{\bslash
% EnableCrossrefs} changes things more drastically; any following
% \texttt{\bslash DisableCrossrefs} which might be present in the
% source will be ignored.}
%
%
% \DescribeMacro\DoNotIndex
% But also finer control is provided. The |\DoNotIndex| macro
% takes a list of macro names separated by commas. Those names won't
% show up in the index. You might use several |\DoNotIndex|
% commands: their lists will be concatenated.  In this article I used
% |\DoNotIndex| for
% all macros which are already defined in \LaTeX.
%
% All three above declarations are local to the current group.
%
% Production (or not) of the index (via the "\makeindex" commend) is
% controlled by using or omitting the following declarations in the
% driver file preamble; if neither is used, no index is produced.
%  
% \DescribeMacro\PageIndex
% Using "\PageIndex" makes all index
% entries refer to their page number; with
% \DescribeMacro\CodelineIndex 
%"\CodelineIndex", index entries
% produced by "\DescribeMacro" and "\DescribeEnv" refer to page number
% but those produced by the \textsf{macro} environment refer to the
% code lines, which will be numbered automatically.\footnote{The line
% number is actually that of the first line of the first
% \textsf{macrocode} environment in the \textsf{macro} environment.}
% \DescribeMacro\theCodelineNo
% The style of this numbering can be controlled by defining the macro
% "\theCodelineNo".  Its default definition is to use scriptsize
% arabic numerals; a user-supplied definition won't be overwritten.
%
% \DescribeMacro\CodelineNumbered
% When you don't wish to get an index but want your code lines
% numbered use "\CodelineNumbered" instead of "\CodelineIndex". This
% prevents the generation of an unnecessary ".idx" file.
%
% 以上是一些我从doc.dtx文件中节选的,暂时保留这些,如有需要再作相应调整。
%  
