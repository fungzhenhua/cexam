% \iffalse meta-comment
%
% Copyright (C) 2005-2021 by Tsinghua University TUNA Association <tuna@tsinghua.edu.cn>
%
% This work may be distributed and/or modified under the
% conditions of the LaTeX Project Public License, either version 1.3c
% of this license or (at your option) any later version.
% The latest version of this license is in
%    https://www.latex-project.org/lppl.txt
% and version 1.3c or later is part of all distributions of LaTeX
% version 2008 or later.
%
% \fi
%
% \iffalse
%<*driver>
\ProvidesFile{bnuthesis.dtx}[2024/04/06 0.0.3 Beijing Normal University Thesis Template]
\documentclass{ltxdoc}
\usepackage{dtx-style}

\EnableCrossrefs
\CodelineIndex

\begin{document}
  \DocInput{\jobname.dtx}
\end{document}
%</driver>
% \fi
%
% \DoNotIndex{\newenvironment,\@bsphack,\@empty,\@esphack,\sfcode}
% \DoNotIndex{\addtocounter,\label,\let,\linewidth,\newcounter}
% \DoNotIndex{\noindent,\normalfont,\par,\parskip,\phantomsection}
% \DoNotIndex{\providecommand,\ProvidesPackage,\refstepcounter}
% \DoNotIndex{\RequirePackage,\setcounter,\setlength,\string,\strut}
% \DoNotIndex{\textbackslash,\texttt,\ttfamily,\usepackage}
% \DoNotIndex{\begin,\end,\begingroup,\endgroup,\par,\\}
% \DoNotIndex{\if,\ifx,\ifdim,\ifnum,\ifcase,\else,\or,\fi}
% \DoNotIndex{\let,\def,\xdef,\edef,\newcommand,\renewcommand}
% \DoNotIndex{\expandafter,\csname,\endcsname,\relax,\protect}
% \DoNotIndex{\Huge,\huge,\LARGE,\Large,\large,\normalsize}
% \DoNotIndex{\small,\footnotesize,\scriptsize,\tiny}
% \DoNotIndex{\normalfont,\bfseries,\slshape,\sffamily,\interlinepenalty}
% \DoNotIndex{\textbf,\textit,\textsf,\textsc}
% \DoNotIndex{\hfil,\par,\hskip,\vskip,\vspace,\quad}
% \DoNotIndex{\centering,\raggedright,\ref}
% \DoNotIndex{\c@secnumdepth,\@startsection,\@setfontsize}
% \DoNotIndex{\ ,\@plus,\@minus,\p@,\z@,\@m,\@M,\@ne,\m@ne}
% \DoNotIndex{\@@par,\DeclareOperation,\RequirePackage,\LoadClass}
% \DoNotIndex{\AtBeginDocument,\AtEndDocument}
%
% \GetFileInfo{\jobname.dtx}
%
% \def\indexname{索引}
% \IndexPrologue{\section{\indexname}}
%
% \title{\bfseries\color{violet}\bnuthesis:北京师范大学学位论文模板}
% \author{{\fangsong 北京师范大学图书馆}\\[5pt]\texttt{zq@bnu.edu.cn}}
% \date{v\fileversion\ (\filedate)}
% \maketitle\thispagestyle{empty}
%
%
% \begin{abstract}\noindent
%   此宏包旨在建立一个简单易用的北京师范大学学位论文模板,包括本科毕业论文、硕士
%   论文和博士论文。
% \end{abstract}
%
% \vskip2cm
% \def\abstractname{免责声明}
% \begin{abstract}
% \noindent
% \begin{enumerate}
% \item 本模板的发布遵守 \href{https://www.latex-project.org/lppl/lppl-1-3c.txt}{\LaTeX{} Project Public License (1.3.c)},使用前请认真阅读协议内
%   容。
% \item 本模板为作者根据
%   清华大学 TUNA 协会 制作的\href{https://github.com/tuna/thuthesis}
%   {清华大学学位论文模板}修改得到的,
%   在此基础上,参考了北京师范大学教务部(研究生院)颁发的《
%    北京师范大学研究生学位论文编写规则%
%   》(2023年修订)、以及
%   《\href{https://jwb.bnu.edu.cn//docs/2021-11/400cb9c075b746879552180a1e31e5e2.pdf}{北京师范大学本科生毕业论文写作规范}》
%   编写而成,旨在供北京师范大学毕业生撰写学位论文使用。
% \item 任何个人或组织以本模板为基础进行修改、扩展而生成的新的专用模板,请严格遵
%   守 \LaTeX{} Project Public License 协议。由于违犯协议而引起的任何纠纷争端均与
%   本模板作者无关。
% \end{enumerate}
% \end{abstract}
%
%
% \clearpage
% \pagestyle{fancy}
% \begin{multicols}{2}[
%   \setlength{\columnseprule}{.4pt}
%   \setlength{\columnsep}{18pt}]
%   \tableofcontents
% \end{multicols}
% \clearpage
%
% \section{模板介绍}
% \bnuthesis{}(\textbf{B}eijing \textbf{N}ormal \textbf{U}niversity \LaTeX{} \textbf{Thesis} Template)
% 是为了帮助北京师范大学毕业生撰写毕业论文而编写
% 的 \LaTeX{} 论文模板。
%
% 本文档将尽量完整地介绍模板的使用方法,如有不清楚之处,或者想提出改进建议,
% 可以在 \href{https://github.com/tuna/thuthesis/issues/}{GitHub Issues}
% 参与讨论或提问。
% 有兴趣者都可以参与完善此手册,也非常欢迎对代码的贡献。
%
% \note[注意:]{模板的作用在于减少论文写作过程中格式调整的时间。前提是遵守模板的
% 用法,否则即便用了 \bnuthesis{} 也难以保证输出的论文符合学校规范。}
%
% 用户如果遇到 bug,或者发现与学校《写作指南》 的要求不一致,可以尝试以下办法:
% \begin{enumerate}
%   \item 将模板升级到最新,见第~\ref{sec:upgrade} 节;
%   \item 阅读 \href{https://github.com/tuna/thuthesis/wiki/FAQ}{FAQ};
%   \item 在 GitHub Issues 中按照说明
%     \href{https://github.com/tuna/thuthesis/issues/new?template=bug_report.md}{报告 bug}。
% \end{enumerate}
%
% \section{贡献者}
% \label{sec:contributors}
%
% \thuthesis{} 的开发过程中,主要的维护者包括:
%
% \begin{itemize}
%  \item 薛瑞尼(\githubuser{xueruini}):最早的开发者,2005 年创建 \thuthesis{} 并长期进行维护工作。
%  \item 赵涛(\githubuser{alick}):2011-2015 年活跃,较早期阶段的开发者。
%  \item 李泽平(\githubuser{zepinglee}):2016 年至今活跃,为目前主要维护者。
%  \item 陈晟祺(\githubuser{Harry-Chen}):2020 年至今活跃,主要负责非开发性事宜。
% \end{itemize}
%
% 同时,也要感谢所有在 GitHub 上提出问题与贡献代码的同学、老师们。
% \bnuthesis{} 的持续发展,离不开你们的帮助与支持。
%
% \section{安装}
% \label{sec:installation}
%
% \bnuthesis{} 已经包含在主要的 \TeX{} 发行版中,但是通常版本较旧,而且不方便更新。
% 建议从下列途径下载最新版:
% \begin{description}
%  \item[GitHub] \url{https://github.com/tuna/thuthesis},从 Release 中下载 zip 文件。
%  \item[TUNA 镜像站] \url{https://mirrors.tuna.tsinghua.edu.cn/github-release/tuna/thuthesis/},也可在首页选择“获取下载链接——应用软件——\bnuthesis{}论文模板”。
% \end{description}
%
%
% 模板支持在 TeX Live、MacTeX 和 MiKTeX 平台下进行编译,但要求 2017 年或更新的发行版。
% 当然,尽可能使用最新的版本可以避免 bug。
%
% \subsection{模板的组成}
% 下表列出了 \bnuthesis{} 的主要文件及其功能介绍:
%
% \begin{longtable}{l|p{8cm}}
% \toprule
% {\heiti 文件(夹)} & {\heiti 功能描述}\\\midrule
% \endfirsthead
% \midrule
% {\heiti 文件(夹)} & {\heiti 功能描述}\\\midrule
% \endhead
% \endfoot
% \endlastfoot
% bnuthesis.ins & \textsc{DocStrip} 驱动文件(开发用) \\
% bnuthesis.dtx & \textsc{DocStrip} 源文件(开发用)\\\midrule
% bnuthesis.cls & 模板类文件\\
% bnuthesis-*.bst & \hologo{BibTeX} 参考文献表样式文件\\
% bnuthesis-*.bbx & BibLaTeX 参考文献表样式文件\\
% bnuthesis-*.cbx & BibLaTeX 参考文献引用样式文件\\
% bnu-name-bachelor.pdf & 校名 logo,本科生封面使用 \\\midrule
% bnuthesis-example.tex & 示例文档主文件\\
% spine.tex & 书脊示例文档\\
% ref/ & 示例文档参考文献目录\\
% data/ & 示例文档章节具体内容\\
% figures/ & 示例文档图片路径\\
% bnusetup.tex & 示例文档基本配置\\\midrule
% Makefile & Makefile\\
% latexmkrc & latexmk 配置文件 \\
% README.md & Readme\\
% \textbf{bnuthesis.pdf} & 用户手册(本文档)\\\bottomrule
% \end{longtable}
%
% 几点说明:
% \begin{itemize}
% \item \file{bnuthesis.cls} 可由 \file{bnuthesis.ins}
%   和 \file{bnuthesis.dtx} 生成,但为了降低新手用户的使用难度,故
%   将 \file{bnuthesis.cls} 文件一起发布。
% \item 使用前阅读文档:\file{bnuthesis.pdf}。
% \end{itemize}
%
% \subsection{生成模板}
% \label{sec:generate-cls}
% 模板的源文件(\file{bnuthesis.dtx})中包含了大量的注释,需要将注释去掉生成轻量
% 级的 \file{.cls} 文件供 \cs{documentclass} 调用。
%
% \begin{shell}
%   $ xetex bnuthesis.ins
% \end{shell}
%
% \note[注意:]{如果没有生成的模板 \file{bnuthesis.cls} 文件
%   (跟 \file{bnuthesis-example.tex} 同一目录下),
%   \LaTeX{} 在编译时可能找到发行版中较旧版本的 \file{.cls},从而造成冲突。}
%
% \subsection{编译论文}
% \label{sec:generate-thesis}
% 本节介绍几种常见的生成论文的方法。用户可根据自己的情况选择。
%
% 在撰写论文时,我们\textbf{不推荐}使用原有的 \file{bnuthesis-example.tex} 这一名称。
% 建议将其复制一份,改为其他的名字(如 \file{thesis.tex} 或者 \file{main.tex})。
% 需要注意,如果使用了来自 \file{data} 目录中的 \file{tex} 文件,
% 则重命名主文件后,其顶端的 \texttt{!TeX root} 选项也需要相应修改。
%
% \subsubsection{GNU make}
% \label{sec:make}
% 如果用户可以使用 GNU make 工具,这是最方便的办法。
% 所以 \bnuthesis{} 提供了 \file{Makefile}:
% \begin{shell}
%   $ make thesis    # 生成论文示例 bnuthesis-example.pdf
%   $ make spine     # 生成书脊 spine.pdf
%   $ make doc       # 生成说明文档 bnuthesis.pdf
%   $ make clean     # 清理编译生成的辅助文件
% \end{shell}
%
% 需要注意,如果更改了主文件的名称,则需要修改 \file{Makefile} 顶端的 \texttt{THESIS} 变量定义。
%
% \subsubsection{latexmk}
% \label{sec:latexmk}
% \texttt{latexmk} 命令支持全自动生成 \LaTeX{} 编写的文档,并且支持使用不同的工具
% 链来进行生成,它会自动运行多次工具直到交叉引用都被解决。
% \begin{shell}
%   $ latexmk bnuthesis-example.tex  # 生成示例论文 bnuthesis-example.pdf
%   $ latexmk spine.tex              # 生成书脊 spine.pdf
%   $ latexmk bnuthesis.dtx          # 生成说明文档 bnuthesis.pdf
%   $ latexmk -c                     # 清理编译生成的辅助文件
% \end{shell}
% \texttt{latexmk} 的编译过程是通过 \file{latexmkrc} 文件来配置的,如果要进一步了解,
% 可以参考 \pkg{latexmk} 文档。
%
% \subsubsection{\XeLaTeX}
% \label{sec:xelatex}
% 如果用户无法使用以上两种较为方便的编译方法,就只能按照以下复杂的办法手动编译。
%
% 首先,更新模板:
% \begin{shell}
%   $ xetex bnuthesis.ins                       # 生成 bnuthesis.cls
% \end{shell}
%
% 然后,生成论文以及书脊:
% \begin{shell}
%   $ xelatex bnuthesis-example.tex
%   $ bibtex bnuthesis-example.aux              # 生成 bbl 文件
%   $ bibtex bnuthesis-example-appendix-a.aux   # 附录 A 的的参考文献
%   $ bibtex bnuthesis-example-appendix-b.aux   # 附录 B 的的参考文献……
%   $ bibtex bnuthesis-example-index.aux        # 本科生的书面翻译对应的原文索引
%   $ xelatex bnuthesis-example.tex             # 解决引用
%   $ xelatex bnuthesis-example.tex             # 生成论文 PDF
%
%   $ xelatex spine.tex                         # 生成书脊 PDF
% \end{shell}
%
% 下面的命令用来生成用户手册:
% \begin{shell}
%   $ xelatex -shell-escape bnuthesis.dtx
%   $ makeindex -s gind.ist -o bnuthesis.ind bnuthesis.idx
%   $ xelatex -shell-escape bnuthesis.dtx
%   $ xelatex -shell-escape bnuthesis.dtx  # 生成说明文档 bnuthesis.pdf
% \end{shell}
%
% \subsection{升级}
% \label{sec:upgrade}
% 如果需要升级 \bnuthesis{},应当从 GitHub 下载最新的版本,
% 将 \file{bnuthesis.dtx},\file{bnuthesis.ins},\file{bnu-name-bachelor.pdf} 和
% \file{bnuthesis-*.bst} 拷贝至工作目录覆盖相应的文件,然后按照
% 第~\ref{sec:generate-cls} 节的内容生成新的模板和使用说明。
%
% 有时模板可能进行了重要的修改,不兼容已写好的正文内容,用户应按照示例
% 文档重新调整。
%
% \section{使用说明}
% \label{sec:usage}
% 本手册假定用户已经能处理一般的 \LaTeX{} 文档,并对 \hologo{BibTeX} 有一定了解。如果
% 从未接触过 \TeX{} 和 \LaTeX,建议先学习相关的基础知识。
%
% \subsection{示例文件}
% \label{sec:userguide}
%
% 模板核心文件有:\file{bnuthesis.cls},\file{bnu-name-bachelor.pdf},
% \file{bnuthesis-*.bst}(\hologo{BibTeX}),
% \file{bnuthesis-*.bbx} 和 \file{bnuthesis-*.cbx}(BibLaTeX),
% 但如果没有示例文档会较难下手,所以推荐从模板自带的示例文档入手。其中包括了论文
% 写作用到的所有命令及其使用方法,只需用自己的内容进行相应替换就可以。对于不清
% 楚的命令可以查阅本手册。下面的例子描述了模板中章节的组织形式,来自于示例文档,
% 具体内容可以参考模板附带的 \file{bnuthesis-example.tex} 和 \file{data/}。
%
% \subsection{论文选项}
% \label{sec:option}
%
% \subsubsection{学位}
% \DescribeOption{degree}
% 选择学位,可选:
% \option{bachelor},\option{master},\option{doctor}(默认)。
% 本节中的 \emph{key-value} 选项只能在文档类的选项中进行设置,
% 不能用于 \cs{bnusetup} 命令。
% \begin{latex}
%   % 博士论文
%   \documentclass[degree=doctor]{bnuthesis}
% \end{latex}
%
% \subsubsection{学位类型}
% \label{sec:degree-type}
% \DescribeOption{degree-type}
% 定义研究生学位的类型,可选:\option{academic}(默认)、\option{professional},
% 本科生不受影响。
% \begin{latex}
%   \documentclass[degree=master, degree-type=professional]{bnuthesis}
% \end{latex}
%
% \subsubsection{字体配置}
% \label{sec:font-config}
% \DescribeOption{fontset}
% 模板默认会自动根据操作系统配置合适的字体,
% 用户也可以通过 \option{fontset} 时指定使用预设的字库,如:
% \begin{latex}
%   \documentclass[fontset=windows]{bnuthesis}
% \end{latex}
% 允许的选项有 \option{windows}、\option{mac}、\option{ubuntu} 和 \option{fandol},
% 具体使用的字体见表~\ref{tab:fontset}。
% 用户也可以设置为 \option{none} 并自行配置字体。
%
% \begin{table}[htb]
%   \centering
%   \caption{\bnuthesis{} 预设的字体}
%   \label{tab:fontset}
%   \begin{tabular}{cccc}
%     \toprule
%     \option{windows} & \option{mac}    & \option{ubuntu} & \option{fandol} \\
%     \midrule
%     Times New Roman  & Times New Roman & TeX Gyre Termes & TeX Gyre Termes \\
%     Arial            & Arial           & TeX Gyre Heros  & TeX Gyre Heros  \\
%     Courier          & Menlo           & TeX Gyre Cursor & TeX Gyre Cursor \\
%     中易宋体         & 华文宋体        & 思源宋体        & Fandol 宋体     \\
%     中易黑体         & 华文黑体        & 思源黑体        & Fandol 黑体     \\
%     中易仿宋         & 华文仿宋        & Fandol 仿宋     & Fandol 仿宋     \\
%     \bottomrule
%   \end{tabular}
% \end{table}
%
% 需要注意,建议用户在提交终版前使用 Windows 平台的字体进行编译。
% 这样中文字体同 Word 模板一致。
%
% 关于字体的配置,
% 详见 \pkg{fontspec}、\pkg{xeCJK}、\pkg{ctex} 等宏包的使用说明和代码。
%
% \DescribeOption{font}
% 配置全文使用的西文字体。所有可选项目为 \option{auto}(默认)、\option{times}、\option{termes}、
% \option{stix}、\option{xits}、\option{libertinus}、\option{newcm}、\option{lm}、
% \option{newtx}、\option{none}。
% 通常来说,用户\textbf{不需要}调整此选项。
%
% \DescribeOption{cjk-font}
% 配置全文使用的中文字体。所有可选项为 \option{auto}(默认)、\option{windows}、\option{windows-local}、
% \option{mac}、\option{mac-word}、\option{noto}、\option{fandol} 和 \option{none}。
% 通常来说,用户\textbf{不需要}调整此选项,模板会自动通过 \option{fontset} 选项选择合适的字体。
%
% \DescribeOption{windows-font-dir}
% 配置搜索 Windows 字体的路径,仅适用于 Overleaf 等不方便全局安装字体的环境。如果此目录下能找到中易宋体,
% 则将自动使用这些字体编译。如有可能,始终建议全局安装相应字体,模板能够自动检测。
%
% \subsection{论文设置}
% 论文的设置可以通过统一命令 \cs{bnusetup} 设置 \emph{key=value} 形式完成。
%
% \DescribeMacro{\bnusetup}
% \cs{bnusetup} 用法与常见 \emph{key=value} 命令相同,如下:
% \begin{latex}
%   \bnusetup{
%     key1 = value1,
%     key2 = {a value, with comma},
%   }
%   % 可以多次调用
%   \bnusetup{
%     key3 = value3,
%     key1 = value11,  % 覆盖 value1
%   }
% \end{latex}
%
% \note[注意:]{\cs{bnusetup} 使用 \pkg{kvsetkeys} 机制,所以配置项之间不能有空行,否则
% 会报错。}
%
% \subsubsection{输出格式}
% \DescribeOption{output}
% 选择输出的格式是打印版还是电子版(用于提交),可选:\option{print}(默认)、\option{electronic}。
% 打印版 \option{print} 自动在单面打印的部分插入空白页(比如封面),并且保证正文第 1 页在右侧。
% 电子版 \option{electronic} 选项会去掉空白页,这是因为一些院系要求提交的电子版不含空白页。
% 注意在不同选项下,生成的声明页码很可能不同。为了避免页码错误,
% \bnuthesis{}将会在插入扫描的 PDF 文件时自动生成页码,因此\textbf{扫描声明页时请移除底部的页码},以防重叠。
%
% \begin{latex}
%   \bnusetup{
%     output = electronic,
%   }
% \end{latex}
%
% 另外本科生要求有 0.2cm 留给装订线的宽度,这只有在打印版中才会生效。
%
%
% \subsubsection{书写语言}
% \DescribeOption{language}
% 注意,目前学校并未给出英文论文的格式要求,用户撰写英文论文前请与导师和院系老师协商。
%
% 在导言区设置 \option{language} 会修改论文的主要语言,如章节标题等。
% 在正文中设置 \option{language} 只修改接下来部分的书写语言,
% 如标点格式、图表名称,但不影响章节标题等。
%
% \begin{latex}
%   \bnusetup{
%     language = english,
%   }
% \end{latex}
%
% 论文的一些部分(如英文摘要、本科生的外文调研报告)要求使用特定的语言,
% 模板已经进行配置,并在这些部分结束后自动恢复为主要语言。
%
%
% \subsubsection{开题报告}
% \DescribeOption{thesis-type}
% 模板还支持博士生论文开题报告的格式,可以通过设置 \option{thesis-type=proposal} 得到。
%
% 开题报告与学位论文有两点不同:
% \begin{enumerate}
%   \item 封面的信息和格式有区别,尤其是增加了一行“学号”信息,需要通过 \option{student-id} 填写;
%   \item 开题报告不含英文标题页。
% \end{enumerate}
% \begin{latex}
%   \bnusetup{
%     thesis-type = proposal,
%     student-id  = {2000310000},
%   }
% \end{latex}
%
% \subsection{封面信息}
% \label{sec:titlepage}
% 封面信息可以通过统一设置命令 \cs{bnusetup} 设置 \emph{key=value} 形式完成;
% 带 * 号的键通常是对应的英文。
%
% \subsubsection{论文标题}
% 中英文标题。可以在标题内部使用换行|\\|。
% \begin{latex}
%   \bnusetup{
%     title  = {论文中文题目},
%     title* = {Thesis English Title},
%   }
% \end{latex}
%
% \subsubsection{申请学位名称}
% \label{sec:degree-category}
% 学位名称的设置比较复杂,见表~\ref{tab:degree-category}。
%
% \begin{table}[h]
%   \caption{学位名称的要求}
%   \label{tab:degree-category}
%   \begin{tabular}{p{2cm}p{6cm}p{6cm}}
%     \toprule
%     学位类型 & degree-category & degree-category* \\
%     \midrule
%     学术型博士 & 需注明所属的学科门类,例如:
%         哲学、经济学、法学、教育学、文学、历史学、理学、工学、农学、医学、
%         军事学、管理学、艺术学
%       & Doctor of Philosophy \\
%     \midrule
%     学术型硕士 & 同上
%       & 哲学、文学、历史学、法学、教育学、艺术学门类
%         填写“Master of Arts“,其它填写“Master of Science” \\
%     \midrule
%     专业型研究生学位 & 专业学位的名称,例如:教育博士、工程硕士
%       & 专业学位的名称,例如:Doctor of Education, Master of Engineering \\
%     \midrule
%     本科生 & - & - \\
%     \bottomrule
%   \end{tabular}
% \end{table}
%
% \begin{latex}
%   \bnusetup{
%     degree-category  = {您要申请什么学位},
%     degree-category* = {Degree in English},
%   }
% \end{latex}
%
% \subsubsection{院系名称}
% 院系名称。
% \begin{latex}
%   \bnusetup{
%     department = {系名全称},
%   }
% \end{latex}
%
% \subsubsection{学科名称}
%
% \begin{itemize}
%   \item 研究生学术型学位:获得一级学科授权的学科填写一级学科名称,其他填写二级学科名称;
%   \item 本科生:专业名称,第二学位论文需标注“(第二学位)”。
% \end{itemize}
%
% \begin{latex}
%   \bnusetup{
%     discipline  = {学科名称},
%     discipline* = {Discipline in English},
%   }
% \end{latex}
%
% \subsubsection{专业领域}
%
% 仅用于研究生专业型学位。
%
% \begin{itemize}
%   \item 设置专业领域的专业学位类别,填写相应专业领域名称;
%   \item 2019 级及之前工程硕士学位论文,在 \option{engineering-field} 填写相应工程领域名称;
%   \item 其他专业学位类别的学位论文无需此信息。
% \end{itemize}
%
% \begin{latex}
%   \bnusetup{
%     professional-field  = {专业领域},
%     professional-field* = {Professional field},
%   }
% \end{latex}
%
%
% \subsubsection{作者姓名}
% 作者姓名。
% \begin{latex}
%   \bnusetup{
%     author  = {中文姓名},
%     author* = {Name in Pinyin},
%   }
% \end{latex}
%
% \subsubsection{学号}
% 学号,仅用于博士生论文开题报告。
% \begin{latex}
%   \bnusetup{
%     student-id  = {20000310000},
%   }
% \end{latex}
%
% \subsubsection{导师}
% \myentry{导师}
% 导师的姓名与职称之间以“,”(西文逗号,U+002C)隔开,下同。
% \begin{latex}
%   \bnusetup{
%     supervisor  = {导师姓名, 教授},
%     supervisor* = {Professor Supervisor Name},
%   }
% \end{latex}
%
% \myentry{副导师}
% 本科生的辅导教师,硕士的副指导教师。
% \begin{latex}
%   \bnusetup{
%     associate-supervisor  = {副导师姓名, 副教授},
%     associate-supervisor* = {Professor Assoc-Supervisor Name},
%   }
% \end{latex}
%
% \myentry{联合指导教师}
% \begin{latex}
%   \bnusetup{
%     co-supervisor  = {联合指导教师姓名, 教授},
%     co-supervisor* = {Professor Join-Supervisor Name},
%   }
% \end{latex}
%
% \subsubsection{成文日期}
% 默认为当前日期,也可以自己指定,要求使用 ISO 格式。
% \begin{latex}
%   \bnusetup{
%     date = {2011-07-01},
%   }
% \end{latex}
%
% \subsubsection{密级}
% \label{sec:setup-secret}
% 定义秘密级别和年限。
% \begin{latex}
%   \bnusetup{
%     secret-year  = 10,
%     secret-level = {秘密},
%   }
% \end{latex}
%
% \subsubsection{博士后专用参数}
% \begin{latex}
%   \bnusetup{
%     clc                = {分类号},
%     udc                = {udc},
%     id                 = {id},
%     discipline-level-1 = {流动站(一级学科)名称},
%     discipline-level-2 = {专业(二级学科)名称},
%     start-date         = {2011-07-01}, % 研究工作起始时间
%   }
% \end{latex}
%
% \myentry{生成封面}
% \DescribeMacro{\maketitle}
% 生成封面,不含授权说明,摘要等。
% \begin{latex}
%   % 直接生成封面
%   \maketitle
% \end{latex}
%
% \subsection{前言部分}
%
% \subsubsection{指导小组、公开评阅人和答辩委员会名单}
% \myentry{答辩委员会名单}
% \DescribeEnv{committee}
% 学位论文指导小组、公开评阅人和答辩委员会名单可以由 \env{committee} 环境生成,
% 其中的可选参数可以使用 \option{name} 根据是有无指导小组设置合适的标题,比如
% \begin{latex}
%   \begin{committee}[name={学位论文公开评阅人和答辩委员会名单}]
%     ...
%   \end{committee}
% \end{latex}
%
% 答辩委员会名单中的表格使用 LaTeX 生成可能略麻烦,也可以导入 Word 版转成的 PDF 文件,
% \begin{latex}
%   \begin{committee}[file=figures/committee.pdf]
%   \end{committee}
% \end{latex}
%
% \subsubsection{授权说明}
% \myentry{授权说明}
% \DescribeMacro{\copyrightpage}
% 可选参数为扫描得到的 PDF 文件名,例如:
% \begin{latex}
%   % 将签字扫描后授权文件 scan-copyright.pdf 替换原始页面
%   \copyrightpage[file=scan-copyright.pdf]
% \end{latex}
%
% \subsubsection{摘要}
% \myentry{摘要正文}
% \DescribeEnv{abstract}
% \DescribeEnv{abstract*}
%
% 摘要直接在正文中使用 \env{abstract}、\env{abstract*} 环境生成。
%
% \begin{latex}
%   \begin{abstract}
%     摘要请写在这里...
%   \end{abstract}
%
%   \begin{abstract*}
%     Here comes the abstract in English...
%   \end{abstract*}
% \end{latex}
%
% \myentry{关键词}
% 关键词需要使用 \cs{bnusetup} 进行设置。关键词之间以\emph{西文逗号}隔开,模板会
% 自动调整为要求的格式。关键词的设置只要在摘要环境结束前即可。
% \begin{latex}
%   \bnusetup{
%     keywords  = {关键词 1, 关键词 2},
%     keywords* = {keyword 1, keyword 2},
%   }
% \end{latex}
%
% \subsubsection{目录和索引表}
% 目录、插图、表格、公式和算法等索引命令分别如下,将其插入到期望的位置即可(带*的命令表
% 示对应的索引表不会出现在目录中):
%
% \DescribeMacro{\tableofcontents}
% \DescribeMacro{\listoffigures}
% \DescribeMacro{\listoffigures*}
% \DescribeMacro{\listoftables}
% \DescribeMacro{\listoftables*}
% \DescribeMacro{\listofequations}
% \DescribeMacro{\listofequations*}
% \DescribeMacro{\listofalgorithms}
% \DescribeMacro{\listofalgorithms*}
% \begin{longtable}{ll}
% \toprule
%   {\heiti 用途} & {\heiti 命令} \\\midrule
% 目录     & \cs{tableofcontents} \\\midrule
% 插图索引 & \cs{listoffigures}   \\
%          & \cs{listoffigures*}  \\\midrule
% 表格索引 & \cs{listoftables}    \\
%          & \cs{listoftables*}   \\\midrule
% 公式索引 & \cs{listofequations} \\
%          & \cs{listofequations*}\\\midrule
% 算法索引 & \cs{listofalgorithms} \\
%          & \cs{listofalgorithms*}\\\bottomrule
% \end{longtable}
%
% \DescribeOption{toc-chapter-style}
% 本科生《写作指南》关于目录章标题要求“目录从第 1 章开始,每章标题用黑体小四号字”,
% 所以其中的西文和数字默认使用 Times 字体,跟正文的章标题一致。
% 但是论文样例\footnote{此前发布在法学院网站,目前已经无公开可获得的链接。}的目录章标题的西文和数字却使用了 Times。
% 如果审查老师这样要求,需要在生成目录前设置
% \begin{latex}
%   \bnusetup{
%     toc-chapter-style = times,
%   }
% \end{latex}
% 该选项只对本科生有效。
%
% \LaTeX{} 默认支持插图和表格索引,是通过 \cs{caption} 命令完成的,因此它们必须出
% 现在浮动环境中,否则不被计数。
%
% 如果不想让某个表格或者图片出现在索引里面,那么请使用命令 \cs{caption*},这
% 个命令不会给表格编号,也就是出来的只有标题文字而没有“表~xx”,“图~xx”,否则
% 索引里面序号不连续就显得不伦不类,这也是 \LaTeX{} 里星号命令默认的规则。
%
% 如果的确想让其编号,但又不想出现在索引中的话,目前模板暂不支持。
%
% 公式索引为本模板扩展,模板扩展了 \pkg{amsmath} 几个内部命令,使得公式编号样式和
% 自动索引功能非常方便。一般来说,你用到的所有数学环境编号都没问题了,这个可以参
% 看示例文档。如果你有个非常特殊的数学环境需要加入公式索引,那么请使
% 用 \cs{equcaption}\marg{编号}。此命令表示 equation caption,带一个参数,即显示
% 在索引中的编号。因为公式与图表不同,我们很少给一个公式附加一个标题,之所以起这
% 么个名字是因为图表就是通过 \cs{caption} 加入索引的,\cs{equcaption} 完全就是为
% 了生成公式列表,不产生什么标题。
%
% 使用方法如下。假如有一个非 equation 数学环境 \texttt{mymath},只要在其中写一
% 句 \cs{equcaption} 就可以将它加入公式列表。
% \begin{latex}
%   \begin{mymath}
%     \label{eq:emc2}\equcaption{\ref{eq:emc2}}
%     E=mc^2
%   \end{mymath}
% \end{latex}
%
% \texttt{mymath} 中公式的编号需要自己来做。
%
% 同图表一样,附录中的公式有时也不希望它跟全文统一编号,而且不希望它出现在公式
% 索引中。目前的办法是利用 \cs{tag*}\marg{公式编号} 来解决。用法比较简单,此
% 处不再罗嗦,实例请参看示例文档附录 A 的前两个公式。
%
% \subsubsection{符号对照表}
% \DescribeEnv{denotation}
% 主要符号表环境,跟 \env{description} 类似,使用方法参见示例文件。带一个可选参数,
% 用来指定符号列的宽度(默认为 2.5cm)。
% \begin{latex}
%   \begin{denotation}
%     \item[E] 能量
%     \item[m] 质量
%     \item[c] 光速
%   \end{denotation}
% \end{latex}
%
% 如果默认符号列的宽度不满意,可以通过参数来调整:
% \begin{latex}
%   \begin{denotation}[1.5cm] % 设置为 1.5cm
%     \item[E] 能量
%     \item[m] 质量
%     \item[c] 光速
%   \end{denotation}
% \end{latex}
%
% 符号对照表的另外一种方法是调用 \pkg{nomencl} 宏包,需要在导言区设置:
%
% \begin{latex}
%   \usepackage{nomencl}
%   \makenomenclature
% \end{latex}
%
% 然后在正文中任意位置使用 \cs{nomenclature} 声明需要添加到主要符号表的符号:
%
% \begin{latex}
%   \nomenclature{$m$}{The mass of one angel}
% \end{latex}
%
% 最后使用 \cs{printnomenclature} 命令生成符号表。更详细的使用方法参
% 见 \pkg{nomencl} 宏包的文档。
%
% \subsection{正文部分}
% \subsubsection{图表编号}
% \DescribeOption{figure-number-separator}
% \DescribeOption{table-number-separator}
% \DescribeOption{equation-number-separator}
% 研究生要求图表的编号使用“.”连接,模板默认使用句点“.”。
% 本科生研究生要求公式的编号使用“-”连接。
% 用户也可以通过 \option{figure-number-separator}、\option{table-number-separator}
% 等选项分别设置:
% \DescribeOption{figurestables-chapternumber}
% 图表编号是否章节有关,可选:\option{true} 和 \option{false}。
% \begin{latex}
%   \bnusetup{
%     figure-number-separator = {-},
%     table-number-separator = {-},
%     equation-number-separator = {-},
%     figurestables-chapternumber = false,
%   }
% \end{latex}
% \DescribeOption{number-separator}
% 也可以使用 \option{number-separator} 同时设置图、表、公式三项的编号连接符,
% 比如 |\bnusetup{number-separator = -}|。
%
% 本科生要求“附录中图、表、公式的编号,应与正文中的编号区分开”,
% 应理解为将章号改变为附录对应的大写字母编号,连接符不宜改变。
%
% \subsubsection{数学符号}
% \label{sec:math}
% 中文论文的数学符号默认遵循 GB/T 3102.11—1993《物理科学和技术中使用的数学符号》
% \footnote{原 GB 3102.11—1993,自 2017 年 3 月 23 日起,该标准转为推荐性标准。}。
% 该标准参照采纳 ISO 31-11:1992 \footnote{目前已更新为 ISO 80000-2:2019。},
% 但是与 \TeX{} 默认的美国数学学会(AMS)的习惯有许多差异。
% 这将在下文详细论述。
%
% \DescribeOption{math-style}
% 用户可以通过设置 \option{math-style} 选择数学符号样式(可选:
% \option{GB}(中文默认),\option{TeX}(英文默认)和 \option{ISO}),比如:
% \begin{latex}
%   \bnusetup{
%     math-style = ISO,
%   }
% \end{latex}
%
% 用户也可以逐项修改数学样式。
% \newcommand\dif{\mathop{}\!\mathrm{d}}
% \begin{enumerate}
%   \item \DescribeOption{uppercase-greek}
%     大写希腊字母的正/斜体,可选:\option{italic}、\option{upright}。
%     有限增量符号 $\increment x$ 固定使用正体,推荐使用 \cs{increment} 表示。
%   \item \DescribeOption{less-than-or-equal}
%     小于等于号和大于等于号的字形,可选:\option{slanted}、\option{horizontal}。
%     这将控制 \cs{le}、\cs{ge}、\cs{leq} 和 \cs{geq} 的符号
%     是“$\leqslant$、$\geqslant$”还是“$\leq$、$\geq$”。
%   \item \DescribeOption{integral}
%     积分号的正/斜体,可选:\option{upright}、\option{slanted}。
%     该选项需要字体的支持,目前仅限 \option{xits}、\option{stix}、
%     \option{libertinus} 和 \option{newcm}。参考下文关于数学字体的选择。
%   \item \DescribeOption{integral-limits}
%     积分号上下限的位置,可选:\option{true}(在上下)、\option{false}(在右侧)。
%     这个设置只影响行间公式,行内公式统一居右侧,不受影响。
%   \item \DescribeOption{partial}
%     偏微分符号的正/斜体,可选:\option{upright}、\option{italic}。
%   \item \DescribeOption{math-ellipsis}
%     省略号 \cs{dots} 的样式,可选:\option{centered}(按照中文的习惯固定居中)、
%     \option{lower} 和 \option{AMS}(取决于前后符号的位置)。
%     其他的省略号命令如 \cs{lots}、\cs{cdots} 则不受影响。
%   \item \DescribeOption{real-part}
%     实部 \cs{Re} 和虚部 \cs{Im} 的字体,可选:\option{roman} 和 \option{fraktur}。
% \end{enumerate}
%
% 如果数学符号选择国标样式 |math-style = GB|,相当于设置了
% \begin{latex}
%   \bnusetup{
%     uppercase-greek    = italic,
%     less-than-or-equal = slanted,
%     integral           = upright,
%     integral-limits    = false,
%     partial            = upright,
%     math-ellipsis      = centered,
%     real-part          = roman, 
%   }
% \end{latex}
%
% 另外,国标的数学样式与 AMS 还有一些差异无法统一设置,需要用户在写作时进行处理。
% \begin{enumerate}
%   \item 数学常数和特殊函数名用正体,如 $\uppi = 3.14\dots$;$\symup{i}^2 = -1$;
%     $\symup{e} = \lim_{n \to \infty} \left( 1 + \frac{1}{n} \right)^n$。
%   \item 微分号使用正体,比如 $\dif y / \dif x$。
%   \item 向量、矩阵和张量用粗斜体(\cs{symbf}),如 $\symbf{x}$、$\symbf{\Sigma}$、$\symbfsf{T}$。
% \end{enumerate}
%
% 需要注意,上述关于数学符号风格的设置在设置数学字体(\option{math-font})时才会生效。
%
% \DescribeOption{math-font}
% 模板使用默认使用 XITS Math 作为数学字体。
% 用户也可以使用 \option{math-font} 选项切换其他数学字体,可选:
% \option{stix}(STIX Two Math)、
% \option{libertinus}(Libertinus Math)、
% \option{newcm}(New Computer Modern Math)、
% \option{lm}(Latin Modern Math)。
%
% 其中 \option{lm} 和 \option{newcm} 的字形比较搭配 TeX 原生的 Computer Modern 字体,
% 但与《指南》要求的西文字体 Times New Roman 并不搭配。
% 可能会造成正文和公式中的数字字体不一致,需要谨慎使用。
%
% 以上字体都是 OpenType 格式的字体,需要配合
% \href{http://mirrors.ctan.org/macros/unicodetex/latex/unicode-math/unicode-math.pdf}{\pkg{unicode-math}}
% 宏包使用。
% 全部数学符号的命令参考
% \href{http://mirrors.ctan.org/macros/unicodetex/latex/unicode-math/unimath-symbols.pdf}{\pkg{unimath-symbols}}。
% 注意,\pkg{unicode-math} 宏包与 \pkg{amsfonts}、\pkg{amssymb}、\pkg{bm}、
% \pkg{mathrsfs}、\pkg{upgreek} 等宏包\emph{不}兼容。
% 模板作了处理,用户可以直接使用这些宏包的命令,如 \cs{bm}、\cs{mathscr}、
% \cs{uppi}。
%
% 另外,模板还为 `math-font` 提供了传统的 Type 1 字体 \option{newtx}。
% 该选项会调用 \pkg{newtxmath} 宏包。
% 但是,如果西文字体已经使用了 OpenType 的 Times New Roman,
% 混用 Type 1 字体可能会导致问题,尤其是使用 \pkg{siunitx} 宏包时。
% 该选项还处于测试阶段,需要谨慎使用。
%
% \subsubsection{定理环境}
% \label{sec:theorem}
% \bnuthesis{} 定义了常用的数学环境:
%
% \begin{center}
% \begin{tabular}{*{7}{l}}\toprule
%   axiom & theorem & definition & proposition & lemma & conjecture &\\
%   公理 & 定理 & 定义 & 命题 & 引理 & 猜想 &\\\midrule
%   proof & corollary & example & exercise & assumption & remark & problem \\
%   证明 & 推论 & 例子& 练习 & 假设 & 注释 & 问题\\\bottomrule
% \end{tabular}
% \end{center}
%
% 比如:
% \begin{latex}
%   \begin{definition}
%     道千乘之国,敬事而信,节用而爱人,使民以时。
%   \end{definition}
% \end{latex}
% 产生(自动编号):
% \medskip
%
% \noindent\framebox[\linewidth][l]{{\heiti 定义~1.1~~~} % {道千乘之国,敬事而信,节用而爱人,使民以时。}}
%
% \smallskip
% 列举出来的数学环境毕竟是有限的,如果想用\emph{胡说}这样的数学环境,那么可以定义:
% \begin{latex}
%   \newtheorem{nonsense}{胡说}[chapter]
% \end{latex}
%
% 然后这样使用:
% \begin{latex}
%   \begin{nonsense}
%     契丹武士要来中原夺武林秘笈。—— 慕容博
%   \end{nonsense}
% \end{latex}
%
% 产生(自动编号):
%
% \medskip
% \noindent\framebox[\linewidth][l]{{\heiti 胡说~1.1~~~} % {契丹武士要来中原夺武林秘笈。—— 慕容博}}
%
% \subsubsection{列表环境}
% \DescribeEnv{itemize}
% \DescribeEnv{enumerate}
% \DescribeEnv{description}
% 为了适合中文习惯,模板将这三个常用的列表环境用 \pkg{enumitem} 进行了纵向间距压
% 缩。一方面清除了多余空间,另一方面用户可以自己指定列表环境的样式(如标签符号,
% 缩进等)。细节请参看 \pkg{enumitem} 文档,此处不再赘述。
%
% \subsubsection{引用方式}
% \label{sec:citestyle}
% 模板支持两种引用方式,分别为理工科常用的“顺序编码制”和文科常用
% 的“著者-出版年制”。
% 使用者在设置参考文献表的格式
% (\cs{bibliographystyle},见第~\ref{sec:bibliography} 节)时,
% 正文中引用文献的标注会自动调整为对应的格式。
%
% 如果需要标出引文的页码,可以写在 \cs{cite} 的可选参数中,如
% |\cite[42]{knuth84}|。
%
% \paragraph{顺序编码制}
% \DescribeMacro{\inlinecite}
% 顺序编码制的参考文献引用分为两种模式:
% \begin{enumerate}
%   \item 上标模式,比如“同样的工作有很多\textsuperscript{[1-2]}……”;
%   \item 正文模式,比如“文 [3] 中详细说明了……”。
% \end{enumerate}
%
% \DescribeOption{cite-style}
% 用户可以将引用标注的格式设为正文模式:
% \begin{latex}
%   \bnusetup{
%     cite-style = inline,
%   }
% \end{latex}
% 也可以使用 \cs{inlinecite}\marg{key} 临时使用正文模式的引用标注。
%
% \paragraph{著者-出版年制}
% 著者-出版年制的参考文献引用有两种模式:
% \begin{enumerate}
%   \item \cs{citep}:著者与年份均在括号中,比如“(Zhang, 2008)”,
%     同默认的 \cs{cite} 命令;
%   \item \cs{citet}:著者姓名作为正文的一部分,比如“Zhang (2008)”;
% \end{enumerate}
%
% 另外,\pkg{natbib} 还提供了其他方便引用的命令,
% 比如 \cs{citeauthor}、\cs{citeyear} 等,
% 更多细节参考 \pkg{natbib} 的文档。
%
% \subsection{其他部分}
%
% \subsubsection{参考文献}
% \label{sec:bibliography}
%
% 参考文献通常可以使用 \hologo{BibTeX} 或 biblatex 生成。
% \hologo{BibTeX} 是 LaTeX 处理参考文献的传统的方式,
% 需要在使用 \cs{bibliographystyle}\marg{style} 选择样式
% 并用 \cs{bibliography} 设置 \file{.bib} 的路径。
% 然后使用 \texttt{bibtex} 对 \file{.aux} 文件进行编译得到 \file{.bbl} 文件。
% 其中的参考文献表内容会在后续编译时替换到 \cs{bibliography} 的位置。
% Biblatex 是较新的方式,需要在载入宏包时通过 \option{style} 选择样式,
% 在导言区使用 \cs{addbibresource} 声明数据库的路径,
% 并在输出参考文献表的位置使用 \cs{printbibliography} 命令,
% 而且编译参考文献的命令需要换为 biber。
% 这两种方式各有优缺点,比如 BibTeX 无法对中文按照拼音排序,一些样式更新不够及时;
% Biblatex 运行较缓慢,无法对多个参考文献表使用不同样式。
% 用户需要根据实际选择合适的方式。
%
% 研究生要求的参考文献格式基于《信息与文献 参考文献著录规则》(GB/T 7714—2015)
% 进行了少量改编(如英文姓名不使用全大写),
% 可以选择“顺序编码制”和“著者-出版年制”。
% 如果使用 BibTeX 的方式,需要在导言区载入 \pkg{natbib} 宏包并选择样式,如:
% \begin{latex}
%   % 顺序编码制
%   \usepackage[sort]{natbib}
%   \bibliographystyle{bnuthesis-numeric}
% \end{latex}
% 或
% \begin{latex}
%   % 著者-出版年制
%   \usepackage{natbib}
%   \bibliographystyle{bnuthesis-author-year}
% \end{latex}
% 其中的 \option{sort} 选项会将同一处引用的多个文献编号严格按照顺序排序,
% 这并非《写作指南》要求,但是推荐使用。
% 这里调用的样式由 \href{http://ctan.org/pkg/gbt7714}{\pkg{gbt7714}} 的 \file{.bst} 进行了少量修改。
%
% 参考文献表采用“著者-出版年”制组织时,各篇文献首先按文种集中,然后按著者字
% 顺和出版年排列;中文文献可以按著者汉语拼音字顺排列,也可以按著者的笔画笔顺排列。
% 但由于 \hologo{BibTeX} 功能的局限性,无法自动获取著者姓名的拼音或笔画笔顺进行正确排序。
% 一种解决方法是在 \file{.bib} 数据库的中文文献的 |key| 域手动录入著者姓名的拼音,
% 这比较适合中文文献数量较少的情况,如:
% \begin{latex}
%   @book{capital,
%     author = {马克思 and 恩格斯},
%     key    = {ma3 ke4 si1 & en1 ge2 si1},
%     ...
%   }
% \end{latex}
% 另一种方式是使用 biblatex,应在导言区设置
% \begin{latex}
%   \usepackage[style=thuthesis-author-year]{biblatex}
%   \addbibresource{ref/refs.bib}
% \end{latex}
% 这里的样式由 \href{https://ctan.org/pkg/biblatex-gb7714-2015}{biblatex-gb7714-2015} 进行了少量改编,
% 一些额外用法可以参考该宏包的文档。
% 注意 \pkg{biblatex} 跟 \pkg{natbib} 不兼容,
% 而且 \cs{addbibresource} 必须在导言区设置。
% 输出参考文献表应使用 \cs{printbibliography} 命令。
%
% 本科生要求的中文参考文献格式严格遵从 GB/T 7714—2015,
% 附录中调研报告的英文参考文献可以自行选择合适的风格。
% 但是 biblatex 不支持同一文档中使用不同的格式,
% 所以只能使用 \hologo{BibTeX}:
% \begin{latex}
%   % 本科生参考文献的著录格式
%   \usepackage[sort]{natbib}
%   \bibliographystyle{bnuthesis-bachelor}
% \end{latex}
% 调研报告的参考文献需要选择与 \pkg{natbib} 兼容的样式。
%
% 本科生外文系要求使用 APA 或 MLA。
% APA 的 BibTeX 样式由 \pkg{apacite} 宏包提供,需要在导言区调用:
% \begin{latex}
%   \usepackage[natbibapa]{apacite}
%   \bibliographystyle{apacite}
% \end{latex}
% 其中 \option{natbibapa} 会调用 \pkg{natbib} 来处理引用,
% 这也是宏包推荐的用法。
% 注意目前的 \pkg{apacite} 只支持到 APA 第 6 版。
% 更推荐使用已经更新到 APA 第 7 版的 \pkg{biblatex-apa}:
% \begin{latex}
%   \usepackage[style=apa]{biblatex}
%   \addbibresource{refs-apa.bib}
% \end{latex}
% 注意,如果参考文献中引用了中文文献的话,这两种方法都不能正确调整格式,
% 需要手动进行修改 \file{.bbl} 文件的内容,
% 这时 BibTeX 比 biblatex 更简单些。
%
% BibTeX 没有用于 MLA 的样式,所以对于 MLA 只能使用 biblatex:
% \begin{latex}
%   \usepackage[style=mla-new]{biblatex}
%   \addbibresource{refs-apa.bib}
% \end{latex}
% 注意这里 \option{mla-new} 对应于 MLA 第 8 版的格式,
% \option{mla} 是第 7 版的。
%
% \subsubsection{致谢}
%
% \DescribeEnv{acknowledgements}
% 把致谢做成一个环境更好一些,直接往里面写感谢的话就可以啦。
%
% \begin{latex}
%   \begin{acknowledgements}
%     …
%     还要特别感谢 \bnuthesis{} 节省了论文排版时间!
%   \end{acknowledgements}
% \end{latex}
%
% \subsubsection{声明}
% \DescribeMacro{\statement}
% 直接使用 \cs{statement} 命令可以编译生成声明页。
% 如果要插入扫描后的声明页,将可选参数指定为扫描后的 PDF 文件名,例如:
%
% \begin{latex}
%   \statement[file=scan-statement.pdf]
% \end{latex}
%
% 由于本科生的打印版和电子版有空白页的差别,声明的页码可能不同。
% 所以编译生成声明页时默认不加页脚(\option{empty}),
% 在签字后插入扫描页时再补上页脚(\option{plain}),防止页码冲突。
% 研究生不存在空白页的问题,在编译生成声明时默认加页眉页脚(\option{plain}),
% 而插入扫描版时不再重复(\option{empty})。
%
% 声明的页眉页脚也可以通过 \option{page-style} 参数手动控制,
% 比如编译生成时固定不加页眉页脚:
% \begin{latex}
%   \statement[page-style=empty]
% \end{latex}
% 插入扫描版声明补上页眉页脚:
% \begin{latex}
%   \statement[file=scan-statement.pdf, page-style=plain]
% \end{latex}
%
% \subsubsection{附录}
%
% 附录由 \cs{appendix} 命令开启,然后像正文一样书写。
% \begin{latex}
%   \appendix
%   \chapter{...}
%   ...
% \end{latex}
%
% \DescribeOption{toc-depth}
% 一些院系要求目录中只出现附录的章标题,不出现附录中的一级、二级节标题。模板默认
% 如此设置,用户也可以在 \cs{appendix} 命令后手动控制加入目录的标题层级,其
% 中 |0| 表示章标题,|1| 表示一级节标题,以此类推。
%
% \begin{latex}
%   \appendix
%   \bnusetup{toc-depth=0}  % 目录只出现章标题
% \end{latex}
%
% \DescribeEnv{survey}
% \DescribeEnv{translation}
% 本科生《写作指南》要求附录 A 为外文资料的调研阅读报告或书面翻译,二者择一。
% 调研报告(或书面翻译)的题目和参考文献是独立于论文的,相当一篇独立的小文章,
% 所以模板相应定义了 \env{survey} 和 \env{translation}。在这两个环境内部可以
% 像论文正文一样使用标题和参考文献的命令,但不会影响外部。
% 但是需要使用 \hologo{BibTeX} 对 \file{*-survey.aux} 或者
% \file{*-translation.aux} 进行编译,才能生成参考文献(见 \ref{sec:xelatex} 节)。
% 如果使用 \texttt{latexmk},则无需额外处理。
%
% 同时,阅读报告默认切换书写语言为英语,书面翻译默认切换为中文。
% 如有需要,可以通过 \cs{bnusetup} 的 \option{language} 参数再次更改。
%
% \begin{latex}
%   \begin{survey}
%     \title{...}
%     \maketitle
%     \tableofcontents
%     ... \cite{...} % 报告内容及其引用
%     \bibliographystyle{...}
%     \bibliography{...} % 报告的参考文献
%   \end{survey}
% \end{latex}
%
% “书面翻译对应的原文索引”区别于译文的参考文献,需要使用 \env{translation-index}
% 环境,另外需要使用 \hologo{BibTeX} 编译 \file{*-index.aux},\texttt{latexmk} 同样会自动处理。
%
% \begin{latex}
%   \begin{translation}
%     ... \cite{...} % 书面翻译内容及其引用
%     \bibliographystyle{...}
%     \bibliography{...}  % 书面翻译的参考文献
%     \begin{translation-index}
%       \nocite{...}
%       \bibliographystyle{...}
%       \bibliography{...}  % 书面翻译对应的原文索引
%     \end{translation-index}
%   \end{translation}
% \end{latex}
%
% \subsubsection{个人简历、在学期间完成的相关学术成果}
% \DescribeEnv{resume}
% 研究生的标题为“个人简历、在学期间完成的相关学术成果”,
% 本科生的标题为“在学期间参加课题的研究成果”或“PUBLICATIONS”。
%
% \DescribeEnv{achievements}
% 本章的其他标题同样使用 \cs{section*},\cs{subsection*} 等命令生成,
% 研究成果用 \env{achievements} 环境罗列。
%
% \begin{latex}
%   \begin{resume}
%     \section*{个人简历}
%     ……
%
%     \section*{在学期间完成的相关学术成果}
%
%     \subsection*{学术论文}
%     \begin{achievements}
%       \item ……
%       \item ……
%     \end{achievements}
%
%     \subsection*{专利}
%     \begin{achievements}
%       \item ……
%       \item ……
%     \end{achievements}
%   \end{resume}
% \end{latex}
%
% \subsubsection{综合论文训练记录表}
% \DescribeMacro{\record}
% 本科生需要在最后附上综合论文训练记录表,可以用如下命令:
%
% \begin{latex}
%   \record{file=scan-record.pdf}
% \end{latex}
%
%
% \subsection{书脊}
% \DescribeMacro{\spine}
% \DescribeOption{spine-font}
% \DescribeOption{spine-title}
% \DescribeOption{spine-author}
% 生成装订的书脊,为竖排格式。内容默认使用论文的标题和作者。
% 可以设置 \option{spine-title} 和 \option{spine-author} 来修改。
%
% 博士论文的书脊字体默认为三号字,硕士的为小三号。
% 本科生要求字体大小根据论文的薄厚而定,可以使用 \option{spine-font} 设置字号。
% \begin{latex}
%   \bnusetup{
%     spine-font   = {\zihao{3}},
%     spine-title  = {书脊的标题},
%     spine-author = {书脊的作者姓名},
%     spine-school = {北京师范大学},
%   }
% \end{latex}
%
% 由于 Fandol 字体在 \XeTeX 中的竖排存在一些问题,如果书脊使用的字体是 Fandol 仿宋
%(\option{fontset} 为 \texttt{fandol} 或者 \texttt{ubuntu} 时),则它\textbf{必须作为独立文件生成},
% 否则可能导致后续内容文字方向错乱的问题。
%
% \DescribeOption{include-spine}
% 一些院系要求把书脊插进论文里,需要在 \cs{maketitle} 前设置。
% \begin{latex}
%   \bnusetup{
%     include-spine = true,
%   }
% \end{latex}
% 打开此选项后,书籍会出现在中文封面后面的第一个空白页。如果有英文封面,则在英文封面之前。
% 如果需要书籍出现在其他位置,请手工使用 \cs{spine} 生成,不要使用此选项。
%
% \section{致谢}
% \label{sec:thanks}
% 感谢这些年来一直陪伴 \bnuthesis{} 成长的新老同学!
%
% 欢迎各位到 \href{http://github.com/tuna/thuthesis/}{\bnuthesis{} GitHub 主页}贡献!
%
%
% ^^A redefine some commands in markdown package to remove annoying section numbering
% \renewcommand{\markdownRendererHeadingTwo}[1]{\subsection*{#1}}
% \renewcommand{\markdownRendererHeadingThree}[1]{\subsubsection*{#1}}
% ^^A render changelog from markdown
% \markdownInput{CHANGELOG.md}
%
%
% \StopEventually{\PrintIndex}
% \clearpage
%
% \section{实现细节}
%
% \subsection{基本信息}
%    \begin{macrocode}
%<cls>\NeedsTeXFormat{LaTeX2e}[2017/04/15]
%<cls>\ProvidesClass{bnuthesis}
%<cls>[2024/04/06 0.0.3 Beijing Normal University Thesis Template]
%    \end{macrocode}
%
% 报错
%    \begin{macrocode}
\newcommand\bnu@error[1]{%
  \ClassError{bnuthesis}{#1}{}%
}
\newcommand\bnu@warning[1]{%
  \ClassWarning{bnuthesis}{#1}%
}
\newcommand\bnu@debug[1]{%
  \typeout{Package bnuthesis Info: #1}%
}
\newcommand\bnu@patch@error[1]{%
  \bnu@error{Failed to patch command \protect#1}%
}
\newcommand\bnu@deprecate[2]{%
  \def\bnu@@tmp{#2}%
  \bnu@warning{%
    The #1 is deprecated%
    \ifx\bnu@@tmp\@empty\else
      . Use #2 instead%
    \fi
  }%
}
%    \end{macrocode}
%
% 检查 \LaTeXe{} kernel 版本
%    \begin{macrocode}
\@ifl@t@r\fmtversion{2017/04/15}{}{
  \bnu@error{%
    TeX Live 2017 or later version is required to compile this document%
  }
}
%    \end{macrocode}
%
% 检查编译引擎,要求使用 \XeLaTeX。
%    \begin{macrocode}
\RequirePackage{iftex}
\ifXeTeX\else
  \ifLuaTeX\else
    \bnu@error{XeLaTeX or LuaLaTeX is required to compile this document}
  \fi
\fi
%    \end{macrocode}
%
% 载入用于测试的配置。
%    \begin{macrocode}
\InputIfFileExists{bnuthesis-pdf-test-config.tex}{}{
  \InputIfFileExists{bnuthesis-log-test-config.tex}{}{}
}
%    \end{macrocode}
%
% \subsection{定义选项}
% \label{sec:defoption}
% 定义论文类型以及是否涉密
%    \begin{macrocode}
%<*cls>
\hyphenation{BNU-Thesis}
\def\bnuthesis{BNUThesis}
\def\thuthesis{ThuThesis}
\def\version{0.0.3}
\RequirePackage{kvdefinekeys}
\RequirePackage{kvsetkeys}
\RequirePackage{kvoptions}
\SetupKeyvalOptions{
  family=bnu,
  prefix=bnu@,
  setkeys=\kvsetkeys}
%    \end{macrocode}
%
% \begin{macro}{\bnusetup}
% 提供一个 \cs{bnusetup} 命令支持 \emph{key-value} 的方式来设置。
%    \begin{macrocode}
\let\bnu@setup@hook\@empty
\newcommand\bnusetup[1]{%
  \let\bnu@setup@hook\@empty
  \kvsetkeys{bnu}{#1}%
  \bnu@setup@hook
}
%    \end{macrocode}
% \end{macro}
%
% 同时用 \emph{key-value} 的方式来定义这些接口:
% \begin{latex}
%   \bnu@define@key{
%     <key> = {
%       name = <name>,
%       choices = {
%         <choice1>,
%         <choice2>,
%       },
%       default = <default>,
%     },
%   }
% \end{latex}
%
% 其中 |choices| 设置允许使用的值,默认为第一个(或者 \meta{default});
% \meta{code} 是相应的内容被设置时执行的代码。
%
%    \begin{macrocode}
\newcommand\bnu@define@key[1]{%
  \kvsetkeys{bnu@key}{#1}%
}
\kv@set@family@handler{bnu@key}{%
%    \end{macrocode}
%
% \cs{bnusetup} 会将 \meta{value} 存到 \cs{bnu@\meta{key}},
% 但是宏的名字包含 “-” 这样的特殊字符时不方便直接调用,比如 |key = math-style|,
% 这时可以用 |name| 设置 \meta{key} 的别称,比如 |key = math@style|,
% 这样就可以通过 \cs{bnu@math@style} 来引用。
% |default| 是定义该 \meta{key} 时默认的值,缺省为空。
%
%    \begin{macrocode}
  \@namedef{bnu@#1@@name}{#1}%
  \def\bnu@@default{}%
  \def\bnu@@choices{}%
  \kv@define@key{bnu@value}{name}{%
    \@namedef{bnu@#1@@name}{##1}%
  }%
%    \end{macrocode}
%
% 由于在定义接口时,\cs{bnu@\meta{key}@@code} 不一定有定义,
% 而且在文档类/宏包中还有可能对该 |key| 的 |code| 进行添加。
% 所以 \cs{bnu@\meta{key}@@code} 会检查如果在定义文档类/宏包时则推迟执行,否则立即执行。
%
%    \begin{macrocode}
  \@namedef{bnu@#1@@check}{}%
  \@namedef{bnu@#1@@code}{}%
%    \end{macrocode}
%
% 保存下 |choices = {}| 定义的内容,在定义 \cs{bnu@\meta{name}} 后再执行。
%
%    \begin{macrocode}
  \kv@define@key{bnu@value}{choices}{%
    \def\bnu@@choices{##1}%
    \@namedef{bnu@#1@@reset}{}%
%    \end{macrocode}
%
% \cs{bnu@\meta{key}@check} 检查 |value| 是否有效,
% 并设置 \cs{ifbnu@\meta{name}@\meta{value}}。
%
%    \begin{macrocode}
    \@namedef{bnu@#1@@check}{%
      \@ifundefined{%
        ifbnu@\@nameuse{bnu@#1@@name}@\@nameuse{bnu@\@nameuse{bnu@#1@@name}}%
      }{%
        \bnu@error{Invalid value "#1 = \@nameuse{bnu@\@nameuse{bnu@#1@@name}}"}%
      }%
      \@nameuse{bnu@#1@@reset}%
      \@nameuse{bnu@\@nameuse{bnu@#1@@name}@\@nameuse{bnu@\@nameuse{bnu@#1@@name}}true}%
    }%
  }%
  \kv@define@key{bnu@value}{default}{%
    \def\bnu@@default{##1}%
  }%
  \kvsetkeys{bnu@value}{#2}%
  \@namedef{bnu@\@nameuse{bnu@#1@@name}}{}%
%    \end{macrocode}
%
% 第一个 \meta{choice} 设为 \meta{default},
% 并且对每个 \meta{choice} 定义 \cs{ifbnu@\meta{name}@\meta{choice}}。
%
%    \begin{macrocode}
  \kv@set@family@handler{bnu@choice}{%
    \ifx\bnu@@default\@empty
      \def\bnu@@default{##1}%
    \fi
    \expandafter\newif\csname ifbnu@\@nameuse{bnu@#1@@name}@##1\endcsname
    \expandafter\g@addto@macro\csname bnu@#1@@reset\endcsname{%
      \@nameuse{bnu@\@nameuse{bnu@#1@@name}@##1false}%
    }%
  }%
  \kvsetkeys@expandafter{bnu@choice}{\bnu@@choices}%
%    \end{macrocode}
%
% 将 \meta{default} 赋值到 \cs{bnu@\meta{name}},如果非空则执行相应的代码。
%
%    \begin{macrocode}
  \expandafter\let\csname bnu@\@nameuse{bnu@#1@@name}\endcsname\bnu@@default
  \expandafter\ifx\csname bnu@\@nameuse{bnu@#1@@name}\endcsname\@empty\else
    \@nameuse{bnu@#1@@check}%
  \fi
%    \end{macrocode}
%
% 定义 \cs{bnusetup} 接口。
%
%    \begin{macrocode}
  \kv@define@key{bnu}{#1}{%
    \@namedef{bnu@\@nameuse{bnu@#1@@name}}{##1}%
    \@nameuse{bnu@#1@@check}%
    \@nameuse{bnu@#1@@code}%
  }%
}
%    \end{macrocode}
%
% 定义接口向 |key| 添加 |code|:
%
%    \begin{macrocode}
\newcommand\bnu@option@hook[2]{%
  \expandafter\g@addto@macro\csname bnu@#1@@code\endcsname{#2}%
}
%    \end{macrocode}
%
%    \begin{macrocode}
\bnu@define@key{
  thesis-type = {
    name = thesis@type,
    choices = {
      thesis,
      proposal,
    },
    default = thesis,
  },
  degree = {
    choices = {
      bachelor,
      master,
      doctor,
      postdoc,
    },
    default = doctor,
  },
  degree-type = {
    choices = {
      academic,
      professional,
    },
    name = degree@type,
  },
%    \end{macrocode}
%
% 论文的主要语言。
%    \begin{macrocode}
  main-language = {
    name = main@language,
    choices = {
      chinese,
      english,
    },
  },
%    \end{macrocode}
%
% 用于设置局部语言。
%    \begin{macrocode}
  language = {
    choices = {
      chinese,
      english,
    },
  },
%    \end{macrocode}
%
% 字体
%    \begin{macrocode}
  system = {
    choices = {
      auto,
      mac,
      unix,
      windows,
    },
    default = auto,
  },
  fontset = {
    choices = {
      auto,
      windows,
      mac,
      ubuntu,
      fandol,
      none,
    },
    default = auto,
  },
  font = {
    choices = {
      auto,
      times,
      termes,
      stix,
      xits,
      libertinus,
      newcm,
      lm,
      newtx,
      none,
    },
    default = auto,
  },
  cjk-font = {
    name = cjk@font,
    choices = {
      auto,
      windows,
      windows-local,
      mac,
      mac-word,
      noto,
      fandol,
      none,
    },
    default = auto,
  },
  windows-font-dir = {
    name = windows@font@dir,
    default = {.},
  },
  math-font = {
    name = math@font,
    choices = {
      auto,
      stix,
      xits,
      libertinus,
      newcm,
      lm,
      newtx,
      none,
    },
    default = auto,
  },
  math-style = {
    name = math@style,
    choices = {
      GB,
      ISO,
      TeX,
    },
  },
  uppercase-greek = {
    name = uppercase@greek,
    choices = {
      italic,
      upright,
    },
  },
  less-than-or-equal = {
    name = leq,
    choices = {
      slanted,
      horizontal,
    },
  },
  integral = {
    choices = {
      upright,
      slanted,
    },
  },
  integral-limits = {
    name = integral@limits,
    choices = {
      true,
      false,
    },
  },
  partial = {
    choices = {
      upright,
      italic,
    },
  },
  math-ellipsis = {
    name = math@ellipsis,
    choices = {
      centered,
      lower,
      AMS,
    },
  },
  real-part = {
    name = real@part,
    choices = {
      roman,
      fraktur,
    },
  },
%    \end{macrocode}
%
% 选择打印版还是用于上传的电子版。
%    \begin{macrocode}
  output = {
    choices = {
      print,
      electronic,
    },
    default = print,
  },
}
\newif\ifbnu@degree@graduate
\newcommand\bnu@set@graduate{%
  \bnu@degree@graduatefalse
  \ifbnu@degree@doctor
    \bnu@degree@graduatetrue
  \fi
  \ifbnu@degree@master
    \bnu@degree@graduatetrue
  \fi
}
\bnu@set@graduate
\bnu@option@hook{degree}{%
  \bnu@set@graduate
}
%    \end{macrocode}
%
% 设置默认 \option{openany}。
%    \begin{macrocode}
\DeclareBoolOption[false]{openright}
\DeclareComplementaryOption{openany}{openright}
%    \end{macrocode}
%
% \option{raggedbottom} 选项(默认打开)
%    \begin{macrocode}
\DeclareBoolOption[true]{raggedbottom}
%    \end{macrocode}
%
% 将选项传递给 \pkg{ctexbook}。
%    \begin{macrocode}
\DeclareDefaultOption{\PassOptionsToClass{\CurrentOption}{ctexbook}}
%    \end{macrocode}
%
% 解析用户传递过来的选项,并加载 \pkg{ctexbook}。
%    \begin{macrocode}
\ProcessKeyvalOptions*
%    \end{macrocode}
%
% 设置默认 \option{openany}。
%    \begin{macrocode}
\ifbnu@openright
  \PassOptionsToClass{openright}{book}
\else
  \PassOptionsToClass{openany}{book}
\fi
%    \end{macrocode}
%
% \pkg{unicode-math} 和 \pkg{newtx} 都不需要 \pkg{fontspec} 设置数学字体。
%    \begin{macrocode}
\PassOptionsToPackage{no-math}{fontspec}
%    \end{macrocode}
%
% 使用 \pkg{ctexbook} 类,优于调用 \pkg{ctex} 宏包。
%    \begin{macrocode}
\LoadClass[a4paper,UTF8,zihao=-4,scheme=plain,fontset=none]{ctexbook}[2017/04/01]
%    \end{macrocode}
%
%
% \subsection{装载宏包}
% \label{sec:loadpackage}
%
% 引用的宏包和相应的定义。
%    \begin{macrocode}
\RequirePackage{etoolbox}
\RequirePackage{filehook}
\RequirePackage{xparse}
%    \end{macrocode}
%
%    \begin{macrocode}
\RequirePackage{geometry}%
%    \end{macrocode}
%
% 利用 \pkg{fancyhdr} 设置页眉页脚。
%    \begin{macrocode}
\RequirePackage{fancyhdr}
%    \end{macrocode}
%
%    \begin{macrocode}
\RequirePackage{titletoc}
%    \end{macrocode}
%
% 利用 \pkg{notoccite} 避免目录中引用编号混乱。
%    \begin{macrocode}
\RequirePackage{notoccite}
%    \end{macrocode}
%
% \AmSTeX\ 宏包,用来排出更加漂亮的公式。
%    \begin{macrocode}
\RequirePackage{amsmath}
%    \end{macrocode}
%
% 图形支持宏包。
%    \begin{macrocode}
\RequirePackage{graphicx}
%    \end{macrocode}
%
% 并排图形。\pkg{subfigure}、\pkg{subfig} 已经不再推荐,用新的 \pkg{subcaption}。
% 浮动图形和表格标题样式。\pkg{caption2} 已经不推荐使用,采用新的 \pkg{caption}。
%    \begin{macrocode}
\RequirePackage[labelformat=simple]{subcaption}
%    \end{macrocode}
%
% \pkg{pdfpages} 宏包便于我们插入扫描后的授权说明和声明页 PDF 文档。
%
% 由于 \pkg{pdfpages} 跟 \pkg{TikZ} 的 \pkg{external} 库冲突,
% 需要在导言区的结尾进行处理,见
% \href{https://github.com/tuna/thuthesis/issues/693}{\#693}。
%    \begin{macrocode}
\RequirePackage{pdfpages}
\includepdfset{fitpaper=true}
\AtEndPreamble{
  \ifx\tikzifexternalizing\@undefined\else
    \tikzifexternalizing{
      \renewcommand*\includepdf[2][]{}
    }{}
  \fi
}
%    \end{macrocode}
%
% 更好的列表环境。
%    \begin{macrocode}
\RequirePackage[shortlabels]{enumitem}
\RequirePackage{environ}
%    \end{macrocode}
%
% 禁止 \LaTeX{} 自动调整多余的页面底部空白,并保持脚注仍然在底部。
% 脚注按页编号。
%    \begin{macrocode}
\ifbnu@raggedbottom
  \RequirePackage[bottom,perpage,hang]{footmisc}
  \raggedbottom
\else
  \RequirePackage[perpage,hang]{footmisc}
\fi
%    \end{macrocode}
%
% 利用 \pkg{xeCJKfntef} 实现汉字的下划线和盒子内两段对齐,并可以避免
% \cs{makebox}\oarg{width}\oarg{s} 可能产生的 underful boxes。
%    \begin{macrocode}
\ifXeTeX
  \RequirePackage{xeCJKfntef}
\else
  \RequirePackage{ulem}
\fi
%    \end{macrocode}
%
% 表格控制
%    \begin{macrocode}
\RequirePackage{array}
%    \end{macrocode}
%
% 使用三线表:\cs{toprule},\cs{midrule},\cs{bottomrule}。
%    \begin{macrocode}
\RequirePackage{booktabs}
%    \end{macrocode}
%
%    \begin{macrocode}
\RequirePackage{url}
%    \end{macrocode}
%
% 如果用户在导言区未调用 \pkg{biblatex},则自动调用 \pkg{natbib}。
%    \begin{macrocode}
\AtEndPreamble{
  \@ifpackageloaded{biblatex}{}{
    \@ifpackageloaded{apacite}{}{
      \RequirePackage{natbib}
    }
  }
}
\AtEndOfPackageFile*{natbib}{
  \@ifpackageloaded{apacite}{}{
    \RequirePackage{bibunits}
  }
}
%    \end{macrocode}
%
% 对冲突的宏包报错。
%    \begin{macrocode}
\newcommand\bnu@package@conflict[2]{%
  \AtEndOfPackageFile*{#1}{%
    \AtBeginOfPackageFile*{#2}{%
      \bnu@error{The "#2" package is incompatible with "#1"}%
    }%
  }%
}
\bnu@package@conflict{biblatex}{bibunits}
\bnu@package@conflict{biblatex}{chapterbib}
\bnu@package@conflict{biblatex}{cite}
\bnu@package@conflict{biblatex}{multibib}
\bnu@package@conflict{biblatex}{natbib}

\bnu@package@conflict{bibunits}{biblatex}
\bnu@package@conflict{bibunits}{chapterbib}
\bnu@package@conflict{bibunits}{multibib}

\bnu@package@conflict{unicode-math}{amscd}
\bnu@package@conflict{unicode-math}{amsfonts}
\bnu@package@conflict{unicode-math}{amssymb}
\bnu@package@conflict{unicode-math}{bbm}
\bnu@package@conflict{unicode-math}{bm}
\bnu@package@conflict{unicode-math}{eucal}
\bnu@package@conflict{unicode-math}{eufrak}
\bnu@package@conflict{unicode-math}{mathrsfs}
\bnu@package@conflict{unicode-math}{newtxmath}
\bnu@package@conflict{unicode-math}{upgreek}

\bnu@package@conflict{natbib}{biblatex}
\bnu@package@conflict{natbib}{cite}

\bnu@package@conflict{newtxmath}{amsfonts}
\bnu@package@conflict{newtxmath}{amssymb}
\bnu@package@conflict{newtxmath}{unicode-math}
\bnu@package@conflict{newtxmath}{upgreek}
%    \end{macrocode}
%
% \pkg{amsthm} 需要在 \pkg{newtx} 前载入,参考 \pkg{newtx} 的文档。
%    \begin{macrocode}
\AtBeginOfPackageFile*{amsthm}{
  \@ifpackageloaded{newtxmath}{
    \bnu@error{The "amsthm" package should be loaded before setting "newtxmath"}
  }{}
}%
%    \end{macrocode}
%
% \subsection{页面设置}
% \label{sec:layout}
%
% 研究生《学位论文编排简明示范》:
% 页边距:左边距:25mm;上边距:25mm;右边距:20mm;下边距:20mm。装订线 0 厘米;
%
% 本科生《本科生毕业论文写作规范》:
% 页边距:上左:2.5 厘米,下右:2.0 厘米。装订线:0 厘米。
% 本科生 Word 模板:
% 装订线 0 厘米;页眉距边界:1.5 厘米,页脚距边界:1.75 厘米。
%
% \pkg{fancyhdr} 的页眉是沿底部对齐的,所以只需设置 \cs{headsep},
% \cs{headheight} 可以适当增加高度允许多行页眉。
% 研究生:\cs{headsep} = $\SI{3}{cm} - \SI{2.2}{cm} - \SI{10.5}{bp} \times 1.3
% \approx \SI{0.3}{cm}$。
%
%    \begin{macrocode}
\geometry{
  paper          = a4paper,  % 210 * 297mm
  marginparwidth = 2cm,
  marginparsep   = 0.5cm,
}
\newcommand\bnu@set@geometry{%
  \ifbnu@degree@bachelor
    \geometry{
      top        = 2.5cm,
      bottom     = 2.0cm,
      left       = 2.5cm,
      right      = 2.0cm,
      headheight = 0.5cm,
      headsep    = 0.5cm,
      footskip   = 14.24323pt,
    }%
    \ifbnu@output@print
      \geometry{
        left       = 2.5cm,
        right      = 2cm,
      }%
    \else
      \geometry{
        hmargin    = 3cm,
      }%
    \fi
  \else
    \ifbnu@degree@graduate
      \geometry{
        top        = 2.54cm,
        bottom     = 2.54cm,
        left       = 3.17cm,
        right      = 3.17cm,
      }%
    \else
      \geometry{
        margin     = 3cm,
        headheight = 0.5cm,
        headsep    = 0.3cm,
        footskip   = 0.8cm,
      }%
    \fi
  \fi
}
\setlength{\headheight}{16pt}
\bnu@set@geometry
\bnu@option@hook{degree}{\bnu@set@geometry}
\bnu@option@hook{output}{\bnu@set@geometry}
%    \end{macrocode}
%
%
% \subsection{语言设置}
%
% 定义 \cs{bnu@main@language},当在导言区修改 \option{language} 时,
% 保存为论文的主要语言;
% \cs{bnu@reset@main@language} 则用于正文中恢复为主要语言。
%    \begin{macrocode}
\bnusetup{main-language=\bnu@language}%
\let\bnu@main@language\bnu@language
\bnu@option@hook{language}{%
  \ifx\@begindocumenthook\@undefined\else
    \bnusetup{main-language=\bnu@language}%
    \let\bnu@main@language\bnu@language
  \fi
}
\newcommand\bnu@reset@main@language{%
  \bnusetup{language = \bnu@main@language}%
  \let\bnu@language\bnu@main@language
}
%    \end{macrocode}
%
% 根据语言设置各章节的名称,只有在导言区设置 \option{degree} 和
% \option{language} 时会修改,而在正文局部切换语言时则不变。
%    \begin{macrocode}
\newcommand\bnu@set@chapter@names{%
  \ifbnu@main@language@chinese
    \def\bnu@comments@name{指导教师评语}%
    \def\bibname{参考文献}%
    \def\appendixname{附录}%
    \def\indexname{索引}%
    \def\bnu@resolution@name{答辩委员会决议书}%
    \ifbnu@degree@bachelor
      \def\contentsname{目\qquad 录}%
      \def\listfigurename{插图索引}%
      \def\listtablename{表格索引}%
      \def\bnu@list@figure@table@name{插图和附表索引}%
      \def\bnu@list@algorithm@name{算法索引}%
      \def\bnu@acknowledgements@name{致\qquad 谢}%
      \def\listequationname{公式索引}%
      \def\bnu@denotation@name{主要符号表}%
      \def\bnu@resume@name{在学期间完成的相关学术成果}%
    \else
      \def\listfigurename{插图清单}%
      \def\listtablename{附表清单}%
      \def\bnu@list@figure@table@name{插图和附表清单}%
      \def\bnu@list@algorithm@name{算法清单}%
      \def\listequationname{公式清单}%
      \def\bnu@acknowledgements@name{致\quad 谢}%
      \ifbnu@degree@graduate
        \def\contentsname{目\quad 录}%
        \def\bnu@denotation@name{符号和缩略语说明}%
        \def\bnu@resume@name{在学期间完成的相关学术成果}%
      \else  % degree = postdoc
        \def\contentsname{目\qquad 次}%
        \def\bnu@denotation@name{符号表}%
        \def\bnu@resume@name{个人简历、发表的学术论文与科研成果}%
      \fi
    \fi
  \else
    \ifbnu@main@language@english
      \def\bnu@comments@name{Comments from Thesis Supervisor}%
      \def\bnu@resolution@name{Resolution of Thesis Defense Committee}%
      \def\indexname{Index}%
      \ifbnu@degree@bachelor
        \def\contentsname{CONTENTS}%
        \def\listfigurename{FIGURES}%
        \def\listtablename{TABLES}%
        \def\bnu@list@figure@table@name{FIGURES AND TABLES}%
        \def\bnu@list@algorithm@name{ALGORITHMS}%
        \def\listequationname{EQUATIONS}%
        \def\bnu@denotation@name{ABBREVIATIONS}%
        \def\bibname{REFERENCES}%
        \def\appendixname{APPENDIX}%
        \def\bnu@acknowledgements@name{ACKNOWLEDGEMENTS}%
        \def\bnu@resume@name{PUBLICATIONS}%
      \else
        \def\contentsname{Table of Contents}%
        \def\listfigurename{List of Figures}%
        \def\listtablename{List of Tables}%
        \def\bnu@list@figure@table@name{List of Figures and Tables}%
        \def\bnu@list@algorithm@name{List of Algorithms}%
        \def\listequationname{List of Equations}%
        \def\bnu@denotation@name{List of Symbols and Acronyms}%
        \def\bibname{References}%
        \def\appendixname{Appendix}%
        \def\bnu@acknowledgements@name{Acknowledgements}%
        \def\bnu@resume@name{Resume}%
      \fi
    \fi
  \fi
}
\bnu@set@chapter@names
\bnu@option@hook{degree}{\bnu@set@chapter@names}
\bnu@option@hook{main-language}{\bnu@set@chapter@names}
%    \end{macrocode}
%
% 这部分名称在正文中局部地修改语言时会发生变化,比如英文摘要、
% 本科生附录的阅读报告。
%    \begin{macrocode}
\newcommand\bnu@set@names{%
  \ifbnu@language@chinese
    \ctexset{
      figurename = 图,
      tablename  = 表,
    }%
    \def\bnu@algorithm@name{算法}%
    \def\bnu@equation@name{公式}%
    \def\bnu@assumption@name{假设}%
    \def\bnu@definition@name{定义}%
    \def\bnu@proposition@name{命题}%
    \def\bnu@lemma@name{引理}%
    \def\bnu@theorem@name{定理}%
    \def\bnu@axiom@name{公理}%
    \def\bnu@corollary@name{推论}%
    \def\bnu@exercise@name{练习}%
    \def\bnu@example@name{例}%
    \def\bnu@remark@name{注释}%
    \def\bnu@problem@name{问题}%
    \def\bnu@conjecture@name{猜想}%
    \def\bnu@proof@name{证明}%
    \def\bnu@theorem@separator{:}%
  \else
    \ifbnu@language@english
      \ctexset{
        figurename = {Figure},
        tablename  = {Table},
      }%
      \def\bnu@algorithm@name{Algorithm}%
      \def\bnu@equation@name{Equation}%
      \def\bnu@assumption@name{Assumption}%
      \def\bnu@definition@name{Definition}%
      \def\bnu@proposition@name{Proposition}%
      \def\bnu@lemma@name{Lemma}%
      \def\bnu@theorem@name{Theorem}%
      \def\bnu@axiom@name{Axiom}%
      \def\bnu@corollary@name{Corollary}%
      \def\bnu@exercise@name{Exercise}%
      \def\bnu@example@name{Example}%
      \def\bnu@remark@name{Remark}%
      \def\bnu@problem@name{Problem}%
      \def\bnu@conjecture@name{Conjecture}%
      \def\bnu@proof@name{Proof}%
      \def\bnu@theorem@separator{: }%
    \fi
  \fi
}
\bnu@set@names
\bnu@option@hook{language}{\bnu@set@names}
%    \end{macrocode}
%
% 带圈数字和星号使用中文字体。
%    \begin{macrocode}
\ifLuaTeX
  % ctex 将带圈数字 U+2460–U+24FF 归入字符范围 3(ALchar),这里改回范围 6(JAchar)
  \ltjdefcharrange{3}{%
    "2000-"243F, "2500-"27BF, "2900-"29FF, "2B00-"2BFF}
  \ltjdefcharrange{6}{%
    "2460-"24FF, "2E80-"2EFF, "3000-"30FF, "3190-"319F, "31F0-"4DBF,
    "4E00-"9FFF, "F900-"FAFF, "FE10-"FE1F, "FE30-"FE6F, "FF00-"FFEF,
    "1B000-"1B16F, "1F100-"1F2FF, "20000-"3FFFF, "E0100-"E01EF}
\else
  \ifXeTeX
    \xeCJKDeclareCharClass{CJK}{"2460 -> "2473}
    \xeCJKDeclareCharClass{CJK}{"2605}
  \fi
\fi
%    \end{macrocode}
%
% \newcommand\unicodechar[1]{U+#1(\symbol{"#1})}
% 由于 Unicode 的一些标点符号是中西文混用的:
% \unicodechar{00B7}、
% \unicodechar{2013}、
% \unicodechar{2014}、
% \unicodechar{2018}、
% \unicodechar{2019}、
% \unicodechar{201C}、
% \unicodechar{201D}、
% \unicodechar{2025}、
% \unicodechar{2026}、
% \unicodechar{2E3A},
% 所以要根据语言设置正确的字体。
% \footnote{\url{https://github.com/CTeX-org/ctex-kit/issues/389}}
% 此外切换语言时,有一部分名称是需要被重新定义的。
%    \begin{macrocode}
\newcommand\bnu@set@punctuations{%
  \ifbnu@language@chinese
    \ifLuaTeX
      \ltjsetparameter{jacharrange={+9}}
    \else
      \ifXeTeX
        \xeCJKDeclareCharClass{FullLeft}{"2018, "201C}%
        \xeCJKDeclareCharClass{FullRight}{
          "00B7, "2019, "201D, "2013, "2014, "2025, "2026, "2E3A,
        }%
      \fi
    \fi
  \else
    \ifbnu@language@english
      \ifLuaTeX
        \ltjsetparameter{jacharrange={-9}}
      \else
        \ifXeTeX
          \xeCJKDeclareCharClass{HalfLeft}{"2018, "201C}%
          \xeCJKDeclareCharClass{HalfRight}{
            "00B7, "2019, "201D, "2013, "2014, "2025, "2026, "2E3A,
          }%
        \fi
      \fi
    \fi
  \fi
}
\bnu@set@punctuations
\bnu@option@hook{language}{\bnu@set@punctuations}
%    \end{macrocode}
%
% \subsection{字体}
% \label{sec:font}
%
% \subsubsection{字号}
%
% \begin{macro}{\normalsize}
% 正文小四号(12bp)字,固定行间距20磅。
% 其他字号的行距按照相同的比例设置。
%
% 注意重定义 \cs{normalsize} 应在 \pkg{unicode-math} 的 \cs{setmathfont} 前。
%
% 公式的行间距为1.5倍,即段前空 6 磅,段后空 6 磅。
%
% \cs{small} 等其他命令通常用于表格等环境中,这部分要求单倍行距,与正文的字号-行距比例不同,
% 所以保留默认的 1.2 倍字号的行距,作为单倍行距在英文(1.15 倍字号)和中文(1.3倍字号)
% 两种情况的折衷。
%    \begin{macrocode}
\renewcommand\normalsize{%
  \@setfontsize\normalsize{12bp}{20bp}%
  \abovedisplayskip 6bp%
  \abovedisplayshortskip 6bp%
  \belowdisplayshortskip 6bp%
  \belowdisplayskip \abovedisplayskip
}
\normalsize
\ifx\MakeRobust\@undefined \else
    \MakeRobust\normalsize
\fi
%    \end{macrocode}
% \end{macro}
%
% WORD 中的字号对应该关系如下(1bp = 72.27/72 pt):
% \begin{center}
% \begin{longtable}{llll}
% \toprule
% 初号 & 42bp & 14.82mm & 42.1575pt \\
% 小初 & 36bp & 12.70mm & 36.135 pt \\
% 一号 & 26bp & 9.17mm & 26.0975pt \\
% 小一 & 24bp & 8.47mm & 24.09pt \\
% 二号 & 22bp & 7.76mm & 22.0825pt \\
% 小二 & 18bp & 6.35mm & 18.0675pt \\
% 三号 & 16bp & 5.64mm & 16.06pt \\
% 小三 & 15bp & 5.29mm & 15.05625pt \\
% 四号 & 14bp & 4.94mm & 14.0525pt \\
% 小四 & 12bp & 4.23mm & 12.045pt \\
% 五号 & 10.5bp & 3.70mm & 10.59375pt \\
% 小五 & 9bp & 3.18mm & 9.03375pt \\
% 六号 & 7.5bp & 2.56mm & \\
% 小六 & 6.5bp & 2.29mm & \\
% 七号 & 5.5bp & 1.94mm & \\
% 八号 & 5bp & 1.76mm & \\\bottomrule
% \end{longtable}
% \end{center}
%
% \begin{macro}{\bnu@def@fontsize}
% 根据习惯定义字号。用法:
%
% \cs{bnu@def@fontsize}\marg{字号名称}\marg{磅数}
%
% 避免了字号选择和行距的紧耦合。所有字号定义时为单倍行距,并提供选项指定行距倍数。
%    \begin{macrocode}
\def\bnu@def@fontsize#1#2{%
  \expandafter\newcommand\csname #1\endcsname[1][1.3]{%
    \fontsize{#2}{##1\dimexpr #2}\selectfont}}
%    \end{macrocode}
% \end{macro}
%
% 一组字号定义。
%    \begin{macrocode}
\bnu@def@fontsize{chuhao}{42bp}
\bnu@def@fontsize{xiaochu}{36bp}
\bnu@def@fontsize{yihao}{26bp}
\bnu@def@fontsize{xiaoyi}{24bp}
\bnu@def@fontsize{erhao}{22bp}
\bnu@def@fontsize{xiaoer}{18bp}
\bnu@def@fontsize{sanhao}{16bp}
\bnu@def@fontsize{xiaosan}{15bp}
\bnu@def@fontsize{sihao}{14bp}
\bnu@def@fontsize{xiaosi}{12bp}
\bnu@def@fontsize{wuhao}{10.5bp}
\bnu@def@fontsize{xiaowu}{9bp}
\bnu@def@fontsize{liuhao}{7.5bp}
\bnu@def@fontsize{xiaoliu}{6.5bp}
\bnu@def@fontsize{qihao}{5.5bp}
\bnu@def@fontsize{bahao}{5bp}
%    \end{macrocode}
%
% 检测系统。
%    \begin{macrocode}
\ifbnu@system@auto
  \IfFileExists{/System/Library/Fonts/Menlo.ttc}{
    \bnusetup{system = mac}
  }{
    \IfFileExists{/dev/null}{
      \IfFileExists{null:}{
        \bnusetup{system = windows}
      }{
        \bnusetup{system = unix}
      }
    }{
      \bnusetup{system = windows}
    }
  }
  \bnu@debug{Detected system: \bnu@system}
\fi
%    \end{macrocode}
%
% 使用 \pkg{fontspec} 配置字体。
%    \begin{macrocode}
\newcommand\bnu@mac@word@font@dir{%
  /Applications/Microsoft Word.app/Contents/Resources/DFonts%
}
\ifbnu@fontset@auto
  \ifbnu@system@windows
    \bnusetup{fontset = windows}
  \else
    \IfFontExistsTF{SimSun}{
      \bnusetup{fontset = windows}
    }{
      \IfFileExists{\bnu@windows@font@dir/Simsun.ttc}{
        \bnusetup{fontset = windows, cjk-font = windows-local}
      }{
        \IfFileExists{\bnu@mac@word@font@dir/Simsun.ttc}{
          \bnusetup{fontset = windows, cjk-font = mac-word}
        }{
          \ifbnu@system@mac
            \bnusetup{fontset = mac}
          \else
            \IfFontExistsTF{Noto Serif CJK SC}{
              \bnusetup{fontset = ubuntu}
            }{
              \bnusetup{fontset = fandol}
            }
          \fi
        }
      }
    }
  \fi
  \bnu@debug{Detected fontset: \bnu@fontset}
\fi
%    \end{macrocode}
%
% \subsubsection{西文字体}
%
% 《指南》要求西文字体使用 Times New Roman 和 Arial,
% 但是在 Linux 下没有这两个字体,所以使用它们的克隆版 TeX Gyre Termes 和
% TeX Gyre Heros。
%    \begin{macrocode}
\newcommand\bnu@set@font{%
  \@nameuse{bnu@set@font@\bnu@font}%
}
\bnu@option@hook{font}{\bnu@set@font}
%    \end{macrocode}
%
%    \begin{macrocode}
\newcommand\bnu@set@font@auto{%
  \ifbnu@font@auto
    \ifbnu@fontset@windows
      \bnusetup{font=times}%
    \else
      \ifbnu@fontset@mac
        \bnusetup{font=times}%
      \else
        \bnusetup{font=termes}%
      \fi
    \fi
  \fi
}
\bnu@option@hook{math-font}{\g@addto@macro\bnu@setup@hook{\bnu@set@font@auto}}
\AtBeginOfPackageFile*{siunitx}{\bnu@set@font@auto}
\AtEndPreamble{\bnu@set@font@auto}
%    \end{macrocode}
%
% Times New Roman + Arial
%    \begin{macrocode}
\newcommand\bnu@set@font@times{%
  \setmainfont{Times New Roman}%
  \setsansfont{Arial}%
  \ifbnu@fontset@mac
    \setmonofont{Menlo}[Scale = MatchLowercase]%
  \else
    \setmonofont{Courier New}[Scale = MatchLowercase]%
  \fi
}
%    \end{macrocode}
%
% TeX Gyre Termes
%    \begin{macrocode}
\newcommand\bnu@set@font@termes{%
  \setmainfont{texgyretermes}[
    Extension      = .otf,
    UprightFont    = *-regular,
    BoldFont       = *-bold,
    ItalicFont     = *-italic,
    BoldItalicFont = *-bolditalic,
  ]%
  \bnu@set@texgyre@sans@mono
}
\newcommand\bnu@set@texgyre@sans@mono{%
  \setsansfont{texgyreheros}[
    Extension      = .otf,
    UprightFont    = *-regular,
    BoldFont       = *-bold,
    ItalicFont     = *-italic,
    BoldItalicFont = *-bolditalic,
  ]%
  \setmonofont{texgyrecursor}[
    Extension      = .otf,
    UprightFont    = *-regular,
    BoldFont       = *-bold,
    ItalicFont     = *-italic,
    BoldItalicFont = *-bolditalic,
    Scale          = MatchLowercase,
    Ligatures      = CommonOff,
  ]%
}
%    \end{macrocode}
%
% STIX Two 字体。
% STIX 文件名在 v2.10 2020-12-19 从
% \file{STIX2Text-Regular.otf}、\file{STIX2Math.otf} 分别改为
% \file{STIXTwoText-Regular.otf}、\file{STIXTwoMath-Regular.otf}。
%    \begin{macrocode}
\let\bnu@font@family@stix\@empty
\newcommand\bnu@set@stix@names{%
  \ifx\bnu@font@family@stix\@empty
    \IfFontExistsTF{STIXTwoText-Regular.otf}{%
      \gdef\bnu@font@family@stix{STIXTwoText}%
      \gdef\bnu@font@name@stix@math{STIXTwoMath-Regular}%
    }{%
      \gdef\bnu@font@family@stix{STIX2Text}%
      \gdef\bnu@font@name@stix@math{STIX2Math}%
    }%
  \fi
}
\newcommand\bnu@set@font@stix{%
  \bnu@set@stix@names
  \setmainfont{\bnu@font@family@stix}[
    Extension      = .otf,
    UprightFont    = *-Regular,
    BoldFont       = *-Bold,
    ItalicFont     = *-Italic,
    BoldItalicFont = *-BoldItalic,
  ]%
  \bnu@set@texgyre@sans@mono
}
%    \end{macrocode}
%
% XITS 字体。
% XITS 的文件名在 v1.109 2018-09-30
% 从 \file{xits-regular.otf}、\file{xits-math.otf} 分别改为
% \file{XITS-Regular.otf}、\file{XITSMath-Regular.otf}。
%    \begin{macrocode}
\let\bnu@font@family@xits\@empty
\newcommand\bnu@set@xits@names{%
  \ifx\bnu@font@family@xits\@empty
    \IfFontExistsTF{XITSMath-Regular.otf}{%
      \gdef\bnu@font@family@xits{XITS}%
      \gdef\bnu@font@style@xits@rm{Regular}%
      \gdef\bnu@font@style@xits@bf{Bold}%
      \gdef\bnu@font@style@xits@it{Italic}%
      \gdef\bnu@font@style@xits@bfit{BoldItalic}%
      \gdef\bnu@font@name@xits@math{XITSMath-Regular}%
    }{%
      \gdef\bnu@font@family@xits{xits}%
      \gdef\bnu@font@style@xits@rm{regular}%
      \gdef\bnu@font@style@xits@bf{bold}%
      \gdef\bnu@font@style@xits@it{italic}%
      \gdef\bnu@font@style@xits@bfit{bolditalic}%
      \gdef\bnu@font@name@xits@math{xits-math}%
    }%
  \fi
}
\newcommand\bnu@set@font@xits{%
  \bnu@set@xits@names
  \setmainfont{\bnu@font@family@xits}[
    Extension      = .otf,
    UprightFont    = *-\bnu@font@style@xits@rm,
    BoldFont       = *-\bnu@font@style@xits@bf,
    ItalicFont     = *-\bnu@font@style@xits@it,
    BoldItalicFont = *-\bnu@font@style@xits@bfit,
  ]%
  \bnu@set@texgyre@sans@mono
}
%    \end{macrocode}
%
% Libertinus 字体。
% Libertinus 的文件名在 v6.7 2019-04-03 从小写改为驼峰式,
% 在大小写敏感的平台上需要进行判断。
%    \begin{macrocode}
\let\bnu@font@family@libertinus\@empty
\newcommand\bnu@set@libertinus@names{%
  \ifx\bnu@font@family@libertinus\@empty
    \IfFontExistsTF{LibertinusSerif-Regular.otf}{%
      \gdef\bnu@font@family@libertinus@serif{LibertinusSerif}%
      \gdef\bnu@font@family@libertinus@sans{LibertinusSans}%
      \gdef\bnu@font@name@libertinus@math{LibertinusMath-Regular}%
      \gdef\bnu@font@style@libertinus@rm{Regular}%
      \gdef\bnu@font@style@libertinus@bf{Bold}%
      \gdef\bnu@font@style@libertinus@it{Italic}%
      \gdef\bnu@font@style@libertinus@bfit{BoldItalic}%
    }{%
      \gdef\bnu@font@family@libertinus@serif{libertinusserif}%
      \gdef\bnu@font@family@libertinus@sans{libertinussans}%
      \gdef\bnu@font@name@libertinus@math{libertinusmath-regular}%
      \gdef\bnu@font@style@libertinus@rm{regular}%
      \gdef\bnu@font@style@libertinus@bf{bold}%
      \gdef\bnu@font@style@libertinus@it{italic}%
      \gdef\bnu@font@style@libertinus@bfit{bolditalic}%
    }%
  \fi
}
\newcommand\bnu@set@font@libertinus{%
  \bnu@set@libertinus@names
  \setmainfont{\bnu@font@family@libertinus@serif}[
    Extension      = .otf,
    UprightFont    = *-\bnu@font@style@libertinus@rm,
    BoldFont       = *-\bnu@font@style@libertinus@bf,
    ItalicFont     = *-\bnu@font@style@libertinus@it,
    BoldItalicFont = *-\bnu@font@style@libertinus@bfit,
  ]%
  \setsansfont{\bnu@font@family@libertinus@sans}[
    Extension      = .otf,
    UprightFont    = *-\bnu@font@style@libertinus@rm,
    BoldFont       = *-\bnu@font@style@libertinus@bf,
    ItalicFont     = *-\bnu@font@style@libertinus@it,
  ]%
  \setmonofont{lmmonolt10}[
    Extension      = .otf,
    UprightFont    = *-regular,
    BoldFont       = *-bold,
    ItalicFont     = *-oblique,
    BoldItalicFont = *-boldoblique,
  ]%
}
%    \end{macrocode}
%
% New Computer Modern
%    \begin{macrocode}
\newcommand\bnu@set@font@newcm{%
  \setmainfont{NewCM10}[
    Extension      = .otf,
    UprightFont    = *-Book,
    BoldFont       = *-Bold,
    ItalicFont     = *-BookItalic,
    BoldItalicFont = *-BoldItalic,
  ]%
  \setsansfont{NewCMSans10}[
    Extension         = .otf,
    UprightFont       = *-Book,
    BoldFont          = *-Bold,
    ItalicFont        = *-BookOblique,
    BoldItalicFont    = *-BoldOblique,
  ]%
  \setmonofont{NewCMMono10}[
    Extension           = .otf,
    UprightFont         = *-Book,
    ItalicFont          = *-BookItalic,
    BoldFont            = *-Bold,
    BoldItalicFont      = *-BoldOblique,
  ]%
}
%    \end{macrocode}
%
% Latin Modern
%    \begin{macrocode}
\newcommand\bnu@set@font@lm{%
  \setmainfont{lmroman10}[
    Extension      = .otf,
    UprightFont    = *-regular,
    BoldFont       = *-bold,
    ItalicFont     = *-italic,
    BoldItalicFont = *-bolditalic,
  ]%
  \setsansfont{lmsans10}[
    Extension      = .otf,
    UprightFont    = *-regular,
    BoldFont       = *-bold,
    ItalicFont     = *-oblique,
    BoldItalicFont = *-boldoblique,
  ]%
  \setmonofont{lmmonolt10}[
    Extension      = .otf,
    UprightFont    = *-regular,
    BoldFont       = *-bold,
    ItalicFont     = *-oblique,
    BoldItalicFont = *-boldoblique,
  ]%
}
%    \end{macrocode}
%
% NewTX
%    \begin{macrocode}
\newcommand\bnu@set@font@newtx{%
  \RequirePackage{newtxtext}%
}
%    \end{macrocode}
%
% \subsubsection{中文字体}
%
%    \begin{macrocode}
\ifbnu@cjk@font@auto
  \ifbnu@fontset@mac
    \bnusetup{cjk-font = mac}
  \else
    \ifbnu@fontset@windows
      \IfFontExistsTF{SimSun}{
        \bnusetup{cjk-font = windows}
      }{
        \IfFileExists{\bnu@windows@font@dir/Simsun.ttc}{
          \bnusetup{cjk-font = windows-local}
        }{
          \IfFileExists{\bnu@mac@word@font@dir/Simsun.ttc}{
            \bnusetup{cjk-font = mac-word}
          }{
            \bnu@error{Cannot find "SimSun" font}
          }
        }
      }
    \else
      \ifbnu@fontset@ubuntu
        \bnusetup{cjk-font = noto}
      \else
        \bnusetup{cjk-font = fandol}
      \fi
    \fi
  \fi
  \bnu@debug{Detected CJK font: \bnu@cjk@font}
\fi
%    \end{macrocode}
%
% Windows 的中易字体。
%    \begin{macrocode}
\newcommand\bnu@set@cjk@font@windows{%
  \setCJKmainfont{SimSun}[
    AutoFakeBold = 3,
    ItalicFont   = KaiTi,
  ]%
  \setCJKsansfont{SimHei}[AutoFakeBold = 3]%
  \setCJKmonofont{FangSong}%
  \setCJKfamilyfont{zhsong}{SimSun}[AutoFakeBold = 3]%
  \setCJKfamilyfont{zhhei}{SimHei}[AutoFakeBold = 3]%
  \setCJKfamilyfont{zhkai}{KaiTi}%
  \setCJKfamilyfont{zhfs}{FangSong}%
}
%    \end{macrocode}
%
% 使用本地的 Windows 字体文件。
%
% Windows 的中易楷体和仿宋字体文件名分别为 \file{Simkai.ttf} 和
% \file{Simfang.ttf}(见
% \url{https://learn.microsoft.com/en-us/typography/fonts/windows_11_font_list}),
% 而 macOS 版 Word 对应的字体名为 \file{Kaiti.ttf} 和 \file{Fangsong.ttf}。
% 所以需要进行判断。
%    \begin{macrocode}
\@namedef{bnu@set@cjk@font@windows-local}{%
  \IfFileExists{\bnu@windows@font@dir/Kaiti.ttf}{
    \setCJKmainfont{SimSun}[%
      Path         = \bnu@windows@font@dir/,
      Extension    = .ttc,
      AutoFakeBold = 3,
      ItalicFont   = Kaiti,
      ItalicFeatures = {Extension = .ttf},
    ]%
    \setCJKmonofont{Fangsong}[
      Path         = \bnu@windows@font@dir/,
      Extension    = .ttf,
    ]%
    \setCJKfamilyfont{zhkai}{Kaiti}[
      Path         = \bnu@windows@font@dir/,
      Extension    = .ttf,
    ]%
    \setCJKfamilyfont{zhfs}{Fangsong}[
      Path         = \bnu@windows@font@dir/,
      Extension    = .ttf,
    ]%
  }{
    \setCJKmainfont{SimSun}[%
      Path         = \bnu@windows@font@dir/,
      Extension    = .ttc,
      AutoFakeBold = 3,
      ItalicFont   = Simkai,
      ItalicFeatures = {Extension = .ttf},
    ]%
    \setCJKmonofont{Simfang}[
      Path         = \bnu@windows@font@dir/,
      Extension    = .ttf,
    ]%
    \setCJKfamilyfont{zhkai}{Simkai}[
      Path         = \bnu@windows@font@dir/,
      Extension    = .ttf,
    ]%
    \setCJKfamilyfont{zhfs}{Simfang}[
      Path         = \bnu@windows@font@dir/,
      Extension    = .ttf,
    ]%
  }
  \setCJKsansfont{SimHei}[%
    Path         = \bnu@windows@font@dir/,
    Extension    = .ttf,
    AutoFakeBold = 3,
  ]%
  \setCJKfamilyfont{zhsong}{SimSun}[%
    Path         = \bnu@windows@font@dir/,
    Extension    = .ttc,
    AutoFakeBold = 3,
  ]%
  \setCJKfamilyfont{zhhei}{SimHei}[%
    Path         = \bnu@windows@font@dir/,
    Extension    = .ttf,
    AutoFakeBold = 3,
  ]%
}
%    \end{macrocode}
%
% macOS 的 Microsoft Word 字体。
%    \begin{macrocode}
\@namedef{bnu@set@cjk@font@mac-word}{%
  \let\bnu@windows@font@dir\bnu@mac@word@font@dir
  \@nameuse{bnu@set@cjk@font@windows-local}%
}
%    \end{macrocode}
%
% macOS 的华文字体。
%    \begin{macrocode}
\newcommand\bnu@set@cjk@font@mac{%
  \setCJKmainfont{Songti SC}[
    UprightFont    = * Light,
    BoldFont       = * Bold,
    ItalicFont     = Kaiti SC Regular,
    BoldItalicFont = Kaiti SC Bold,
  ]%
  \setCJKsansfont{Heiti SC}[
    UprightFont    = * Light,
    BoldFont       = * Medium,
  ]%
  \setCJKmonofont{STFangsong}
  \setCJKfamilyfont{zhsong}{Songti SC}[
    UprightFont    = * Light,
    BoldFont       = * Bold,
  ]%
  \setCJKfamilyfont{zhhei}{Heiti SC}[
    UprightFont    = * Light,
    BoldFont       = * Medium,
  ]%
  \setCJKfamilyfont{zhfs}{STFangsong}%
  \setCJKfamilyfont{zhkai}{Kaiti SC}[
    UprightFont    = * Regular,
    BoldFont       = * Bold,
  ]%
  \setCJKfamilyfont{zhli}{Baoli SC}%
  \setCJKfamilyfont{zhyuan}{Yuanyi SC}[
    UprightFont    = * Light,
    BoldFont       = * Bold,
  ]%
}
%    \end{macrocode}
%
% 思源字体。
% 注意 Noto CJK 的 regular 字重名字不带“Regular”。
%    \begin{macrocode}
\newcommand\bnu@set@cjk@font@noto{%
  \setCJKmainfont{Noto Serif CJK SC}[
    UprightFont    = * Light,
    BoldFont       = * Bold,
    ItalicFont     = FandolKai-Regular,
    ItalicFeatures = {Extension = .otf},
  ]%
  \setCJKsansfont{Noto Sans CJK SC}[BoldFont = * Medium]%
  \setCJKmonofont{Noto Sans Mono CJK SC}%
  \setCJKfamilyfont{zhsong}{Noto Serif CJK SC}[
    UprightFont    = * Light,
    UprightFont    = * Bold,
  ]%
  \setCJKfamilyfont{zhhei}{Noto Sans CJK SC}[BoldFont = * Medium]%
  \setCJKfamilyfont{zhfs}{FandolFang}[
    Extension      = .otf,
    UprightFont    = *-Regular,
  ]%
  \setCJKfamilyfont{zhkai}{FandolKai}[
    Extension      = .otf,
    UprightFont    = *-Regular,
  ]%
}
%    \end{macrocode}
%
% Fandol 字体。
%    \begin{macrocode}
\newcommand\bnu@set@cjk@font@fandol{%
  \setCJKmainfont{FandolSong}[
    Extension   = .otf,
    UprightFont = *-Regular,
    BoldFont    = *-Bold,
    ItalicFont  = FandolKai-Regular,
    ItalicFeatures = {Extension = .otf},
  ]%
  \setCJKsansfont{FandolHei}[
    Extension   = .otf,
    UprightFont = *-Regular,
    BoldFont    = *-Bold,
  ]%
  \setCJKmonofont{FandolFang}[
    Extension   = .otf,
    UprightFont = *-Regular,
  ]%
  \setCJKfamilyfont{zhsong}{FandolSong}[
    Extension   = .otf,
    UprightFont = *-Regular,
    BoldFont    = *-Bold,
  ]%
  \setCJKfamilyfont{zhhei}{FandolHei}[
    Extension   = .otf,
    UprightFont = *-Regular,
    BoldFont    = *-Bold,
  ]%
  \setCJKfamilyfont{zhfs}{FandolFang}[
    Extension   = .otf,
    UprightFont = *-Regular,
  ]%
  \setCJKfamilyfont{zhkai}{FandolKai}[
    Extension   = .otf,
    UprightFont = *-Regular,
  ]%
}
\ifbnu@cjk@font@none\else
  \providecommand\songti{\CJKfamily{zhsong}}
  \providecommand\heiti{\CJKfamily{zhhei}}
  \providecommand\fangsong{\CJKfamily{zhfs}}
  \providecommand\kaishu{\CJKfamily{zhkai}}
\fi
\newcommand\bnu@set@cjk@font{%
  \@nameuse{bnu@set@cjk@font@\bnu@cjk@font}%
}
\bnu@set@cjk@font
\bnu@option@hook{cjk-font}{\bnu@set@cjk@font}
%    \end{macrocode}
%
% \subsubsection{数学字体}
%
% 使用 \pkg{unicode-math} 配置数学符号格式。
%    \begin{macrocode}
\newcommand\bnu@set@math@style{%
  \ifbnu@math@style@TeX
    \bnusetup{
      uppercase-greek    = upright,
      less-than-or-equal = horizontal,
      integral           = slanted,
      integral-limits    = false,
      partial            = italic,
      math-ellipsis      = AMS,
      real-part          = fraktur,
    }%
  \else
    \bnusetup{
      uppercase-greek = italic,
      integral        = upright,
      partial         = upright,
      real-part       = roman,
    }%
    \ifbnu@math@style@ISO
      \bnusetup{
        less-than-or-equal = horizontal,
        integral-limits    = true,
        math-ellipsis      = lower,
      }%
    \else
      \ifbnu@math@style@GB
        \bnusetup{
          less-than-or-equal = slanted,
          integral-limits    = false,
          math-ellipsis      = centered,
        }%
      \fi
    \fi
  \fi
}
\ifbnu@main@language@chinese
  \bnusetup{math-style=GB}%
\else
  \bnusetup{math-style=TeX}%
\fi
\bnu@set@math@style
\bnu@option@hook{math-style}{\bnu@set@math@style}
\bnu@option@hook{main-language}{%
  \ifbnu@main@language@chinese
    \bnusetup{math-style=GB}%
  \else
    \bnusetup{math-style=TeX}%
  \fi
}
%    \end{macrocode}
%
% 针对 \pkg{unicode-math} 逐项配置数学符号。
%    \begin{macrocode}
\newcommand\bnu@set@unimath@leq{%
  \ifbnu@leq@horizontal
    \ifx\@begindocumenthook\@undefined
      \let\le\bnu@save@leq
      \let\ge\bnu@save@geq
      \let\leq\bnu@save@leq
      \let\geq\bnu@save@geq
    \else
      \AtBeginDocument{%
        \let\le\bnu@save@leq
        \let\ge\bnu@save@geq
        \let\leq\bnu@save@leq
        \let\geq\bnu@save@geq
      }%
    \fi
  \else
    \ifbnu@leq@slanted
      \ifx\@begindocumenthook\@undefined
        \let\le\leqslant
        \let\ge\geqslant
        \let\leq\leqslant
        \let\geq\geqslant
      \else
        \AtBeginDocument{%
          \let\le\leqslant
          \let\ge\geqslant
          \let\leq\leqslant
          \let\geq\geqslant
        }%
      \fi
    \fi
  \fi
}
\newcommand\bnu@set@unimath@integral@limits{%
  \ifbnu@integral@limits@true
    \removenolimits{%
      \int\iint\iiint\iiiint\oint\oiint\oiiint
      \intclockwise\varointclockwise\ointctrclockwise\sumint
      \intbar\intBar\fint\cirfnint\awint\rppolint
      \scpolint\npolint\pointint\sqint\intlarhk\intx
      \intcap\intcup\upint\lowint
    }%
  \else
    \addnolimits{%
      \int\iint\iiint\iiiint\oint\oiint\oiiint
      \intclockwise\varointclockwise\ointctrclockwise\sumint
      \intbar\intBar\fint\cirfnint\awint\rppolint
      \scpolint\npolint\pointint\sqint\intlarhk\intx
      \intcap\intcup\upint\lowint
    }%
  \fi
}
\newcommand\bnu@set@unimath@ellipsis{%
  \ifbnu@math@ellipsis@centered
    \DeclareRobustCommand\mathellipsis{\mathinner{\unicodecdots}}%
  \else
    \DeclareRobustCommand\mathellipsis{\mathinner{\unicodeellipsis}}%
  \fi
}
\newcommand\bnu@set@unimath@real@part{%
  \ifbnu@real@part@roman
    \AtBeginDocument{%
      \def\Re{\operatorname{Re}}%
      \def\Im{\operatorname{Im}}%
    }%
  \else
    \AtBeginDocument{%
      \let\Re\bnu@save@Re
      \let\Im\bnu@save@Im
    }%
  \fi
}
\newcommand\bnu@set@unimath@style{%
  \ifbnu@uppercase@greek@upright
    \unimathsetup{math-style = TeX}%
  \else
    \ifbnu@uppercase@greek@italic
      \unimathsetup{math-style = ISO}%
    \fi
  \fi
  \ifbnu@math@style@TeX
    \unimathsetup{bold-style = TeX}%
  \else
    \unimathsetup{bold-style = ISO}%
  \fi
  \bnu@set@unimath@leq
  \bnu@set@unimath@integral@limits
  \ifbnu@partial@upright
    \unimathsetup{partial = upright}%
  \else
    \ifbnu@partial@italic
      \unimathsetup{partial = italic}%
    \fi
  \fi
  \bnu@set@unimath@ellipsis
  \bnu@set@unimath@real@part
}
%    \end{macrocode}
%
%    \begin{macrocode}
\newcommand\bnu@qed{\rule{1ex}{1ex}}
\newcommand\bnu@load@unimath{%
  \@ifpackageloaded{unicode-math}{}{%
    \RequirePackage{unicode-math}%
    \AtBeginDocument{%
      \let\bnu@save@leq\leq
      \let\bnu@save@geq\geq
      \let\bnu@save@Re\Re
      \let\bnu@save@Im\Im
    }%
%    \end{macrocode}
%
% 兼容旧的粗体命令:\pkg{bm} 的 \cs{bm} 和 \pkg{amsmath} 的 \cs{boldsymbol}。
%    \begin{macrocode}
    \DeclareRobustCommand\bm[1]{{\symbfit{##1}}}%
    \DeclareRobustCommand\boldsymbol[1]{{\symbfit{##1}}}%
%    \end{macrocode}
%
% 兼容 \pkg{amsfonts} 和 \pkg{amssymb} 中的一些命令。
%    \begin{macrocode}
    \newcommand\square{\mdlgwhtsquare}%
    \newcommand\blacksquare{\mdlgblksquare}%
    \AtBeginDocument{%
      \renewcommand\checkmark{\ensuremath{\symbol{"2713}}}%
    }%
%    \end{macrocode}
%
% 兼容 \pkg{amsthm} 的 \cs{qedsymbol}。
%    \begin{macrocode}
    \renewcommand\bnu@qed{\ensuremath{\QED}}%
  }%
}
%    \end{macrocode}
%
% STIX Two Math
%    \begin{macrocode}
\newcommand\bnu@set@math@font@stix{%
  \bnu@set@stix@names
  \setmathfont{\bnu@font@name@stix@math}[
    Extension    = .otf,
    Scale        = MatchLowercase,
    StylisticSet = \bnu@xits@integral@stylistic@set,
  ]%
  \setmathfont{\bnu@font@name@stix@math}[
    Extension    = .otf,
    Scale        = MatchLowercase,
    StylisticSet = 1,
    range        = {scr,bfscr},
  ]%
}
%    \end{macrocode}
%
% XITS Math
%    \begin{macrocode}
\newcommand\bnu@xits@integral@stylistic@set{%
  \ifbnu@integral@upright
    8%
  \fi
}
\newcommand\bnu@set@math@font@xits{%
  \bnu@set@xits@names
  \setmathfont{\bnu@font@name@xits@math}[
    Extension    = .otf,
    StylisticSet = \bnu@xits@integral@stylistic@set,
  ]%
  \setmathfont{\bnu@font@name@xits@math}[
    Extension    = .otf,
    StylisticSet = 1,
    range        = {cal,bfcal},
  ]%
}
%    \end{macrocode}
%
% Libertinus Math
%    \begin{macrocode}
\newcommand\bnu@libertinus@integral@stylistic@set{%
  \ifbnu@integral@slanted
    8%
  \fi
}
\newcommand\bnu@set@math@font@libertinus{%
  \bnu@set@libertinus@names
  \setmathfont{\bnu@font@name@libertinus@math}[
    Extension    = .otf,
    StylisticSet = \bnu@libertinus@integral@stylistic@set,
  ]%
}
%    \end{macrocode}
%
% New Computer Modern Math
%    \begin{macrocode}
\newcommand\bnu@newcm@integral@stylistic@set{%
  \ifbnu@integral@upright
    2%
  \fi
}
\newcommand\bnu@set@math@font@newcm{%
  \setmathfont{NewCMMath-Book}[
    Extension    = .otf,
    StylisticSet = \bnu@newcm@integral@stylistic@set,
  ]%
  \setmathfont{NewCMMath-Book}[
    Extension    = .otf,
    StylisticSet = 1,
    range        = {scr,bfscr},
  ]%
  \setmathrm{NewCM10}[
    Extension      = .otf,
    UprightFont    = *-Book,
    BoldFont       = *-Bold,
    ItalicFont     = *-BookItalic,
    BoldItalicFont = *-BoldItalic,
  ]%
  \setmathsf{NewCMSans10}[
    Extension         = .otf,
    UprightFont       = *-Book,
    BoldFont          = *-Bold,
    ItalicFont        = *-BookOblique,
    BoldItalicFont    = *-BoldOblique,
  ]%
  \setmathtt{NewCMMono10}[
    Extension           = .otf,
    UprightFont         = *-Book,
    ItalicFont          = *-BookItalic,
    BoldFont            = *-Bold,
    BoldItalicFont      = *-BoldOblique,
  ]%
}
%    \end{macrocode}
%
% Latin Modern Math
%    \begin{macrocode}
\newcommand\bnu@set@math@font@lm{%
  \setmathfont{latinmodern-math}[Extension=.otf]%
  \setmathrm{lmroman10}[
    Extension      = .otf,
    UprightFont    = *-regular,
    BoldFont       = *-bold,
    ItalicFont     = *-italic,
    BoldItalicFont = *-bolditalic,
  ]%
  \setmathsf{lmsans10}[
    Extension      = .otf,
    UprightFont    = *-regular,
    BoldFont       = *-bold,
    ItalicFont     = *-oblique,
    BoldItalicFont = *-boldoblique,
  ]%
  \setmathtt{lmmonolt10}[
    Extension      = .otf,
    UprightFont    = *-regular,
    BoldFont       = *-bold,
    ItalicFont     = *-oblique,
    BoldItalicFont = *-boldoblique,
  ]%
}
%    \end{macrocode}
%
% NewTX Math
%    \begin{macrocode}
\newcommand\bnu@set@math@font@newtx{%
  \ifbnu@font@newtx\else
    \let\bnu@save@encodingdefault\encodingdefault
    \let\bnu@save@rmdefault\rmdefault
    \let\bnu@save@sfdefault\sfdefault
    \let\bnu@save@ttdefault\ttdefault
    \RequirePackage[T1]{fontenc}%
    \renewcommand{\rmdefault}{ntxtlf}%
    \renewcommand{\sfdefault}{qhv}%
    \renewcommand{\ttdefault}{ntxtt}%
  \fi
  \ifbnu@uppercase@greek@italic
    \PassOptionsToPackage{slantedGreek}{newtxmath}%
  \fi
  \ifbnu@integral@upright
    \PassOptionsToPackage{upint}{newtxmath}%
  \fi
  \RequirePackage{newtxmath}
  \let\bnu@save@leq\leq
  \let\bnu@save@geq\geq
  \ifbnu@leq@slanted
    \let\le\leqslant
    \let\ge\geqslant
    \let\leq\leqslant
    \let\geq\geqslant
  \fi
  \ifbnu@integral@limits@true
    \let\ilimits@\displaylimits
  \fi
  \let\bnu@save@partial\partial
  \ifbnu@partial@upright
    \let\partial\uppartial
  \fi
  \ifbnu@math@ellipsis@centered
    \DeclareRobustCommand\mathellipsis{\mathinner{\cdotp\cdotp\cdotp}}%
  \else
    \DeclareRobustCommand\mathellipsis{\mathinner{\ldotp\ldotp\ldotp}}%
  \fi
  \let\bnu@save@Re\Re
  \let\bnu@save@Im\Im
  \ifbnu@real@part@roman
    \def\Re{\operatorname{Re}}%
    \def\Im{\operatorname{Im}}%
  \fi
  \RequirePackage{bm}%
  \ifbnu@font@newtx\else
    \let\encodingdefault\bnu@save@encodingdefault
    \let\rmdefault\bnu@save@rmdefault
    \let\sfdefault\bnu@save@sfdefault
    \let\ttdefault\bnu@save@ttdefault
  \fi
  \DeclareRobustCommand\symup[1]{{\mathrm{##1}}}%
  \DeclareRobustCommand\symbf[1]{{\bm{##1}}}%
  \DeclareRobustCommand\symbfsf[1]{{\bm{\mathsf{##1}}}}%
  \let\increment\upDelta%
  \renewcommand\bnu@qed{\openbox}%
}
%    \end{macrocode}
%
%    \begin{macrocode}
\newcommand\bnu@set@math@font{%
  \ifbnu@math@font@none\else
    \ifbnu@math@font@newtx
      \bnu@set@math@font@newtx
    \else
      \bnu@load@unimath
      \bnu@set@unimath@style
      \@nameuse{bnu@set@math@font@\bnu@math@font}%
    \fi
  \fi
}
\bnu@option@hook{math-font}{\g@addto@macro\bnu@setup@hook{\bnu@set@math@font}}
\newcommand\bnu@set@math@font@auto{%
  \ifbnu@math@font@auto
    \bnusetup{math-font=xits}%
  \fi
}
\AtBeginOfPackageFile*{siunitx}{\bnu@set@math@font@auto}
\AtEndPreamble{\bnu@set@math@font@auto}
%    \end{macrocode}
%
%
% \subsection{主文档格式}
% \label{sec:mainbody}
%
% \subsubsection{Three matters}
% \begin{macro}{\cleardoublepage}
% 对于 \textsl{openright} 选项,必须保证章首页右开,且如果前章末页无内容须
% 清空其页眉页脚。
%    \begin{macrocode}
\def\cleardoublepage{%
  \clearpage
  \if@twoside
    \ifbnu@output@print
      \ifodd\c@page
      \else
        \thispagestyle{empty}%
        \hbox{}%
        \newpage
        \if@twocolumn
          \hbox{}\newpage
        \fi
      \fi
    \fi
  \fi
}
%    \end{macrocode}
% \end{macro}
%
% \begin{macro}{\frontmatter}
% \begin{macro}{\mainmatter}
% \begin{macro}{\backmatter}
% 我们的单面和双面模式与常规的不太一样。
%    \begin{macrocode}
\renewcommand\frontmatter{%
  \cleardoublepage
  \@mainmatterfalse
  \pagenumbering{Roman}%
}
\renewcommand\mainmatter{%
  \cleardoublepage
  \@mainmattertrue
  \pagenumbering{arabic}%
}
\newif\ifbnu@backmatter
\renewcommand\backmatter{%
  \if@openright
    \cleardoublepage
  \else
    \clearpage
  \fi
  \@mainmatterfalse
  \bnu@backmattertrue
  \bnusetup{toc-depth = 0}%
}
%    \end{macrocode}
% \end{macro}
% \end{macro}
% \end{macro}
%
% \subsubsection{页眉页脚}
% \label{sec:headerfooter}
%
% \pkg{fancyhdr} 定义页眉页脚很方便,但是有一个非常隐蔽的坑。
% 第一次调用 \pkg{fancyhdr} 定义的样式时会修改 \cs{chaptermark},
% 这会导致页眉信息错误(多余章号并且英文大写)。
% 这是因为在 \cs{ps@fancy} 中对 \cs{chaptermark} 进行重定义,
% 所以我们先调用 \cs{ps@fancy},再修改 \cs{chaptermark}。
%    \begin{macrocode}
\pagestyle{fancy}
%    \end{macrocode}
%
% 定义页眉和页脚。
% 研究生要求:
% 页眉宋体五号字,宋体五号字居中书写;
% 页码阿拉伯数字“1、2、3……”编排,居中,五号 Times New Roman 体。
%
% 本科生要求:
% 页眉:无;
% 页码:位于页面底端,居中书写,小五 Times New Roman 体。
%
% 本科外文专业要求页码字号 12pt。
%    \begin{macrocode}
\fancypagestyle{plain}{%
  \fancyhf{}%
  \renewcommand\footrulewidth{0pt}%
  \ifbnu@degree@bachelor
    \renewcommand\headrulewidth{0pt}%
    \fancyfoot[C]{
      \ifbnu@main@language@chinese
        \xiaowu
      \else
        \normalsize
      \fi
      \thepage
    }%
    \let\@mkboth\@gobbletwo
    \let\chaptermark\@gobble
  \else
    \renewcommand\headrulewidth{0.75bp}%
    \fancyhead[C]{%
      \wuhao
      \ifbnu@main@language@chinese
        \leftmark
      \else
        \MakeUppercase{\leftmark}%
      \fi
      }%
    \fancyfoot[C]{\wuhao\thepage}%
    \let\@mkboth\markboth
    \def\chaptermark##1{%
      \markboth{%
        \CTEXifname{%
          \CTEXthechapter
          \ifbnu@main@language@chinese
            \quad
          \else
            \space
          \fi
        }{}##1%
      }{}%
    }%
  \fi
  \let\sectionmark\@gobble
}
\pagestyle{plain}
%    \end{macrocode}
%
% \cs{chapter} 会调用特殊的 page style。
%    \begin{macrocode}
\def\ps@chapter{}
\ctexset{chapter/pagestyle = chapter}
%    \end{macrocode}
%
%
% \subsubsection{段落}
% \label{sec:paragraph}
%
% 全文首行缩进 2 字符,标点符号用全角
%    \begin{macrocode}
\ctexset{%
  punct=quanjiao,
}
\newcommand\bnu@set@indent{%
  \ifbnu@main@language@chinese
    \ctexset{autoindent=2}%
  \else
    \ifbnu@degree@bachelor
      \ctexset{autoindent=0.8cm}%
    \else
      \ctexset{autoindent=0.74cm}%
    \fi
  \fi
}
\bnu@set@indent
\bnu@option@hook{degree}{\bnu@set@indent}
\bnu@option@hook{main-language}{\bnu@set@indent}
%    \end{macrocode}
%
% 设置 url 样式,与上下文一致
%    \begin{macrocode}
\urlstyle{same}
%    \end{macrocode}
%
% 使用 \pkg{xurl} 的方法,增加 URL 可断行的位置。
%    \begin{macrocode}
\g@addto@macro\UrlBreaks{%
  \do0\do1\do2\do3\do4\do5\do6\do7\do8\do9%
  \do\A\do\B\do\C\do\D\do\E\do\F\do\G\do\H\do\I\do\J\do\K\do\L\do\M
  \do\N\do\O\do\P\do\Q\do\R\do\S\do\T\do\U\do\V\do\W\do\X\do\Y\do\Z
  \do\a\do\b\do\c\do\d\do\e\do\f\do\g\do\h\do\i\do\j\do\k\do\l\do\m
  \do\n\do\o\do\p\do\q\do\r\do\s\do\t\do\u\do\v\do\w\do\x\do\y\do\z
}
\Urlmuskip=0mu plus 0.1mu
%    \end{macrocode}
%
% 取消列表的间距,以符合中文习惯。
%    \begin{macrocode}
\partopsep=\z@skip
\def\@listi{\leftmargin\leftmargini
            \parsep \z@skip
            \topsep \z@skip
            \itemsep\z@skip}
\let\@listI\@listi
\@listi
\def\@listii {\leftmargin\leftmarginii
              \labelwidth\leftmarginii
              \advance\labelwidth-\labelsep
              \topsep    \z@skip
              \parsep    \z@skip
              \itemsep   \z@skip}
\def\@listiii{\leftmargin\leftmarginiii
              \labelwidth\leftmarginiii
              \advance\labelwidth-\labelsep
              \topsep    \z@skip
              \parsep    \z@skip
              \partopsep \z@skip
              \itemsep   \z@skip}
%    \end{macrocode}
%
% 使用 \pkg{enumitem} 命令调整默认列表环境间的距离,
%    \begin{macrocode}
\setlist{nosep}
%    \end{macrocode}
%
%
% \subsubsection{脚注}
% \label{sec:footnote}
%
% 严格禁止脚注跨页,参考 \href{https://gibnub.com/tuna/thuthesis/issues/778}{\#778}
% 和 \url{https://texfaq.org/FAQ-splitfoot}。
%    \begin{macrocode}
\interfootnotelinepenalty=10000
%    \end{macrocode}
%
% 脚注内容采用小五号字,中文用宋体,英文和数字用 Times New Roman 体按两端对齐格式书写,
% 单倍行距,段前段后均空 0 磅。
% 脚注的序号按页编排,不同页的脚注序号不需要连续。
%
% 脚注处序号“1,……,10”的字体是“正文”,不是“上标”,序号与脚注内容文字之间空半个汉字符,
% 脚注的段落格式为:单倍行距,段前空 0 磅,段后空 0 磅,悬挂缩进 1.5 字符;
% 字号为小五号字,汉字用宋体,外文用 Times New Roman 体。
%
% 脚注序号使用带圈数字。
% \begin{macro}{\bnu@circled}
% 生成带圈的脚注数字,最多处理到 10。
%    \begin{macrocode}
\newcommand\bnu@circled[1]{%
  \ifnum#1 >10\relax
    \bnu@warning{%
      Too many footnotes in this page.
      Keep footnote less than 10%
    }%
  \fi
  {\symbol{\the\numexpr#1+"245F\relax}}%
}
\renewcommand{\thefootnote}{\bnu@circled{\c@footnote}}
\renewcommand{\thempfootnote}{\bnu@circled{\c@mpfootnote}}
%    \end{macrocode}
% \end{macro}
%
% 定义脚注分割线,字号(宋体小五),以及悬挂缩进(1.5字符)。
%    \begin{macrocode}
\def\footnoterule{\vskip-3\p@\hrule\@width0.3\textwidth\@height0.4\p@\vskip2.6\p@}
\footnotemargin=13.5bp
%    \end{macrocode}
%
% 修改 \pkg{footmisc} 定义的脚注格式。
%    \begin{macrocode}
\long\def\@makefntext#1{%
  \begingroup
    % 序号取消上标
    \def\@makefnmark{\hbox{\normalfont\@thefnmark}}%
    \xiaowu
    \ifFN@hangfoot
      \bgroup
      \setbox\@tempboxa\hbox{%
        \ifdim\footnotemargin>\z@
          \hb@xt@\footnotemargin{\@makefnmark\hss}%
        \else
          \@makefnmark
        \fi
      }%
      \leftmargin\wd\@tempboxa
      \rightmargin\z@
      \linewidth \columnwidth
      \advance \linewidth -\leftmargin
      \parshape \@ne \leftmargin \linewidth
      % \footnotesize
      \xiaowu
      \@setpar{{\@@par}}%
      \leavevmode
      \llap{\box\@tempboxa}%
      \parskip\hangfootparskip\relax
      \parindent\hangfootparindent\relax
    \else
      \parindent1em%
      \noindent
      \ifdim\footnotemargin>\z@
        \hb@xt@ \footnotemargin{\hss\@makefnmark}%
      \else
        \ifdim\footnotemargin=\z@
          \llap{\@makefnmark}%
        \else
          \llap{\hb@xt@ -\footnotemargin{\@makefnmark\hss}}%
        \fi
      \fi
    \fi
    \footnotelayout#1%
    \ifFN@hangfoot
      \par\egroup
    \fi
  \endgroup
}
%    \end{macrocode}
%
%
% \subsubsection{数学相关}
% \label{sec:equation}
% 允许太长的公式断行、分页等。
%    \begin{macrocode}
\allowdisplaybreaks[4]
%    \end{macrocode}
%
% 公式距前后文的距离由 4 个参数控制,参见 \cs{normalsize} 的定义。
%
% 中文模板的公式编号使用仍保留英文括号。需要修改 \pkg{amsmath} 的 \cs{tagform@}。
%    \begin{macrocode}
\newcommand\bnu@put@parentheses[1]{%
  \ifbnu@language@chinese
    (#1)%
  \else
    (#1)%
  \fi
}
% \def\tagform@#1{\maketag@@@{(\ignorespaces#1\unskip\@@italiccorr)}}
\def\tagform@#1{\maketag@@@{\bnu@put@parentheses{\ignorespaces#1\unskip\@@italiccorr}}}
%    \end{macrocode}
%
% 重定义 \cs{eqref},去掉原来的的 \cs{tagform@} 中 \cs{maketag@@@} 的 \cs{hbox} 和
% \cs{m@th},防止中文左括号与前面文字的距离过窄。
%    \begin{macrocode}
% \newcommand{\eqref}[1]{\textup{\tagform@{\ref{#1}}}}
\renewcommand{\eqref}[1]{%
  \textup{%
    \normalfont\bnu@put@parentheses{%
      \ignorespaces\ref{#1}\unskip\@@italiccorr
    }%
  }%
}
%    \end{macrocode}
%
% \subsubsection{浮动对象:插图和表格}
% \label{sec:float}
%
% 图表浮动体的默认位置设为 |h|。
%    \begin{macrocode}
\def\fps@figure{htbp}
\def\fps@table{htbp}
%    \end{macrocode}
%
% 设置浮动对象和文字之间的距离
%    \begin{macrocode}
\setlength{\floatsep}{12\p@ \@plus 2\p@ \@minus 2\p@}
\setlength{\textfloatsep}{12\p@ \@plus 2\p@ \@minus 2\p@}
\setlength{\intextsep}{12\p@ \@plus 2\p@ \@minus 2\p@}
\setlength{\@fptop}{0bp \@plus1.0fil}
\setlength{\@fpsep}{12bp \@plus2.0fil}
\setlength{\@fpbot}{0bp \@plus1.0fil}
%    \end{macrocode}
%
% 由于 LaTeX2e kernel 的问题,图表等浮动体与文字前后的距离不一致,需要进行 patch。
% 参考 \href{https://gibnub.com/tuna/thuthesis/issues/614}{tuna/thuthesis/issues\#614}、
% \url{https://www.zhihu.com/question/46618031} 和
% \url{https://tex.stackexchange.com/a/40363/82731}。
%    \begin{macrocode}
\patchcmd{\@addtocurcol}%
  {\vskip \intextsep}%
  {\edef\save@first@penalty{\the\lastpenalty}\unpenalty
   \ifnum \lastpenalty = \@M  % hopefully the OR penalty
     \unpenalty
   \else
     \penalty \save@first@penalty \relax % put it back
   \fi
   \ifnum\outputpenalty <-\@Mii
     \addvspace\intextsep
     \vskip\parskip
   \else
     \addvspace\intextsep
   \fi}%
  {}{\bnu@patch@error{\@addtocurcol}}
\patchcmd{\@addtocurcol}%
  {\vskip\intextsep \ifnum\outputpenalty <-\@Mii \vskip -\parskip\fi}%
  {\ifnum\outputpenalty <-\@Mii
     \aftergroup\vskip\aftergroup\intextsep
     \aftergroup\nointerlineskip
   \else
     \vskip\intextsep
   \fi}%
  {}{\bnu@patch@error{\@addtocurcol}}
\patchcmd{\@getpen}{\@M}{\@Mi}
  {}{\bnu@patch@error{\@getpen}}
%    \end{macrocode}
%
% 下面这组命令使浮动对象的缺省值稍微宽松一点,从而防止幅度对象占据过多的文本页面,
% 也可以防止在很大空白的浮动页上放置很小的图形。
%    \begin{macrocode}
\renewcommand{\textfraction}{0.15}
\renewcommand{\topfraction}{0.85}
\renewcommand{\bottomfraction}{0.65}
\renewcommand{\floatpagefraction}{0.60}
%    \end{macrocode}
%
% 允许用户设置图表编号的连接符。
%    \begin{macrocode}

\bnu@define@key{
  figure-number-separator = {
    name    = figure@number@separator,
    default = {.},
  },
  table-number-separator = {
    name    = table@number@separator,
    default = {.},
  },
  equation-number-separator = {
    name    = equation@number@separator,
    default = {-},
  },
  number-separator = {
    name    = number@separator,
    default = {.},
  },
  figurestables-chapternumber = {
    name = figurestables@chapternumber,
    choices = {
      false,
      true,
    },
    default = false,
  },
}

\ifbnu@degree@bachelor
  \counterwithout{figure}{chapter}
  \counterwithout{table}{chapter}
\fi
\bnu@option@hook{figurestables-chapternumber}{
  \ifbnu@figurestables@chapternumber@true
    \counterwithin{figure}{chapter}
    \counterwithin{table}{chapter}
  \else 
    \counterwithout{figure}{chapter}
    \counterwithout{table}{chapter}
  \fi
}

\renewcommand\thefigure{%
  \ifnum\c@chapter>\z@    
    \ifbnu@figurestables@chapternumber@true  
      \thechapter\bnu@figure@number@separator
    \fi  
    \@arabic\c@figure
  \fi
}

\renewcommand\thetable{%
  \ifnum\c@chapter>\z@    
    \ifbnu@figurestables@chapternumber@true
      \thechapter\bnu@table@number@separator\@arabic\c@figure
    \fi  
    \@arabic\c@table
  \fi
}

\renewcommand\theequation{%
  \ifnum\c@chapter>\z@
    \thechapter
    \bnu@equation@number@separator
  \fi
  \@arabic\c@equation
}
\newcommand\bnu@set@number@separator{%
  \let\bnu@figure@number@separator\bnu@number@separator
  \let\bnu@table@number@separator\bnu@number@separator
  \let\bnu@equation@number@separator\bnu@number@separator
}
\bnu@option@hook{number-separator}{\bnu@set@number@separator}
%    \end{macrocode}
%
% 定制浮动图形和表格标题样式:
% \begin{itemize}
%   \item 图表标题字体为 11pt
%   \item 去掉图表号后面的冒号,图序与图名文字之间空一个汉字符宽度
%   \item 图:caption 在下,段前空 6 磅,段后空 12 磅
%   \item 表:caption 在上,段前空 12 磅,段后空 6 磅
% \end{itemize}
%    \begin{macrocode}
\DeclareCaptionFont{bnu}{%
  \ifbnu@degree@bachelor
    \fontsize{11bp}{15bp}\selectfont
  \else
    \ifbnu@language@chinese
      \fontsize{11bp}{14.3bp}\selectfont
    \else
      \fontsize{11bp}{12.65bp}\selectfont
    \fi
  \fi
}
\captionsetup{
  font           = bnu,
  labelsep       = quad,
  skip           = 6bp,
  figureposition = bottom,
  tableposition  = top,
}
\captionsetup[sub]{font=bnu}
\renewcommand{\thesubfigure}{(\alph{subfigure})}
\renewcommand{\thesubtable}{(\alph{subtable})}
% \renewcommand{\p@subfigure}{:}
%    \end{macrocode}
%
% 研究生要求表单元格中的文字采用 11pt 宋体字,单倍行距。段前空 3 磅,段后空 3 磅。
% 对于中文,\cs{arraystretch} 需要调整为 $1 + 6 / (11 \times 1.3) \approx 1.42$。
% 对于英文,\cs{arraystretch} 需要调整为 $1 + 6 / (11 \times 1.15) \approx 1.47$。
%
% 注意不能简单地把行距设为 $\SI{11}{pt} \times 1.3 + \SI{6}{pt} = \SI{20.3}{pt}$,
% 这会导致含有多行文字的单元格中行距有误。
%
% 其他浮动体中(比如 \env{algorithm})的字号默认同表格一致。
%    \begin{macrocode}
\newcommand\bnu@set@table@font{
  \ifbnu@language@chinese
    \def\bnu@table@font{%
      \fontsize{11bp}{14.3bp}\selectfont
      \renewcommand\arraystretch{1.42}%
    }%
  \else
    \def\bnu@table@font{%
      \fontsize{11bp}{12.65bp}\selectfont
      \renewcommand\arraystretch{1.47}%
    }%
  \fi
}
\bnu@set@table@font
\bnu@option@hook{language}{\bnu@set@table@font}
\patchcmd\@floatboxreset{%
  \normalsize
}{%
  \bnu@table@font
}{}{\bnu@patch@error{\@floatboxreset}}
%    \end{macrocode}
%
% 对 \pkg{longtable} 跨页表格进行相同的设置。
%
% 在 Word 模板中按照正确的设置(需要去掉文档网格),
% 中文模板每页能装下 1 行标题、1 行表头、30 行表身,
% 英文模板每页能装下 1 行标题、1 行表头、33 行表身。
%    \begin{macrocode}
\AtEndOfPackageFile*{longtable}{
  \AtBeginEnvironment{longtable}{%
    \bnu@table@font
  }
}
%    \end{macrocode}
%
% 研究生和本科生都推荐使用三线表,并且要求表的上、下边线为单直线,线粗为 1.5 磅;
% 第三条线为单直线,线粗为 1 磅。
% 这里设置 \pkg{booktabs} 线粗的默认值。
%    \begin{macrocode}
\heavyrulewidth=1.5bp
\lightrulewidth=1bp
%    \end{macrocode}
%
%    \begin{macrocode}
\AtEndOfPackageFile*{threeparttable}{
  \g@addto@macro\TPT@defaults{\wuhao}
}
%    \end{macrocode}
%
% \subsubsection{章节标题}
% \label{sec:theor}
%    \begin{macrocode}
\ifbnu@degree@bachelor
  \newcommand{\bnu@abstract@name}{摘\quad 要}
  \newcommand{\bnu@abstract@name@en}{ABSTRACT}
\else
  \ifbnu@degree@graduate
    \newcommand{\bnu@abstract@name@en}{ABSTRACT}
    \newcommand{\bnu@abstract@name}{摘\quad 要}
  \else
    \newcommand{\bnu@abstract@name}{摘\quad 要}
    \newcommand{\bnu@abstract@name@en}{Abstract}
  \fi
\fi
%    \end{macrocode}
%
% 各级标题格式设置。
%    \begin{macrocode}
\ctexset{%
  chapter = {
    nameformat   = {},
    numberformat = {},
    titleformat  = {},
    fixskip      = true,
    afterindent  = true,
    lofskip      = 0pt,
    lotskip      = 0pt,
  },
  section = {
    afterindent  = true,
  },
  subsection = {
    afterindent  = true,
  },
  subsubsection = {
    afterindent  = true,
  },
  paragraph/afterindent = true,
  subparagraph/afterindent = true,
}
%    \end{macrocode}
%
% 本科生要求:
% \begin{center}
%   \begin{tabular}{lclll}
%     \toprule
%     标题        & 中文       & 英文       & 段前/后间距 & 行距 \\
%     \midrule
%     一级标题  & 黑体加粗三号 & Times New Roman 16pt       & 0.5 行   & 20pt \\
%     二级标题  & 黑体加粗小三号   & Times New Roman 15pt & 0.5 行   & 20pt \\
%     三级标题  & 黑体加粗四号 & Times New Roman 14pt     & 0.5 行   & 20pt \\
%     四级标题  & 黑体加粗小四号 & Times New Roman 12pt   & 0.5 行   & 20pt  \\
%     \bottomrule
%   \end{tabular}
% \end{center}
%
% 这里三级标题的“中文黑体加粗四号”和“英文 Times New Roman 14pt”不一致,取 14pt。
%    \begin{macrocode}
\newcommand\bnu@set@section@format{%
  \ifbnu@degree@bachelor
    \ctexset{%
      chapter = {
        format     = \centering\heiti\bfseries\fontsize{16bp}{20bp}\selectfont,
        aftername = \hspace{0.5em},
        beforeskip = 0.5\baselineskip,
        afterskip  = 0.5\baselineskip,
      },
      section = {
        format     = \heiti\bfseries\fontsize{15bp}{20bp}\selectfont,
        beforeskip = 0.5\baselineskip,
        afterskip  = 0.5\baselineskip,
      },
      subsection = {
        format     = \heiti\bfseries\fontsize{14bp}{20bp}\selectfont,
        beforeskip = 0.5\baselineskip,
        afterskip  = 0.5\baselineskip,
      },
      subsubsection = {
        format     = \heiti\bfseries\fontsize{12bp}{20bp}\selectfont,
        beforeskip = 0.5\baselineskip,
        afterskip  = 0.5\baselineskip,
      },
    }%
    \ifbnu@main@language@chinese
      \ctexset{
        chapter = {
          name   = {第,章},
          number = \thechapter,
        },
      }%
    \else
      \ctexset{
        chapter = {
          name   = \chaptername\space,
          number = \bnu@english@number{chapter},
        },
      }%
    \fi
%    \end{macrocode}
%
% 研究生要求:
% \begin{itemize}
%   \item 各章标题,例如:“\textsf{第 1 章 引言}”。
%
%     章序号与标题之间空一个汉字符。
%     采用黑体三号字,居中书写,单倍行距,
%     段前空 24 磅,段后空 18 磅。
%
%   \item 一级节标题,例如:“\textsf{2.1 实验装置与实验方法}”。
%
%     节标题序号与标题名之间空半个汉字符(下同)。
%     采用黑体,小三号,顶头,单倍行距,
%     段前空 24 磅,段后空 6 磅。
%
%   \item 二级节标题,例如:“\textsf{2.1.1 实验装置}”。
%
%     采用黑体,四号,顶头,单倍行距,
%     段前空 12 磅,段后空 6 磅。
%
%   \item 三级节标题,例如:“\textsf{2.1.2.1 归纳法}”。
%
%     采用黑体,小四号,顶头,单倍行距,
%     段前空 12 磅,段后空 6 磅。
% \end{itemize}
%
% 由于 Word 的行距算法不同,这里进行了一些调整使得视觉上更接近。
%    \begin{macrocode}
  \else
    \ctexset{%
      chapter = {
        beforeskip = 27bp,
        afterskip  = 27bp,
        number     = \thechapter,
      },
      section = {
        beforeskip = 24bp,
        afterskip  = 6bp,
      },
      subsection = {
        beforeskip = 12bp,
        afterskip  = 6bp,
      },
      subsubsection = {
        beforeskip = 12bp,
        afterskip  = 6bp,
      },
    }%
    \ifbnu@main@language@chinese
      \ctexset{%
        chapter = {
          format      = \centering\sffamily\sanhao,
          nameformat  = {},
          titleformat = {},
          name        = {第,章},
          aftername = \hspace{0.5em},
        },
        section = {
          format     = \sffamily\fontsize{14bp}{20bp}\selectfont,
          aftername = \hspace{0.25em},
        },
        subsection = {
          format     = \sffamily\fontsize{13bp}{20bp}\selectfont,
          aftername = \hspace{0.25em},
        },
        subsubsection = {
          format     = \sffamily\fontsize{12bp}{20bp}\selectfont,
          aftername = \hspace{0.25em},
        },
      }%
    \else
      \ctexset{%
        chapter = {
          format      = \centering\sffamily\bfseries\fontsize{16bp}{20bp}\selectfont,
          nameformat  = \MakeUppercase,
          titleformat = \MakeUppercase,
          name        = \chaptername\space,
          aftername   = \space,
        },
        section = {
          format     = \sffamily\bfseries\fontsize{14bp}{20bp}\selectfont,
          aftername  = \space,
        },
        subsection = {
          format     = \sffamily\bfseries\fontsize{13bp}{20bp}\selectfont,
          aftername  = \space,
        },
        subsubsection = {
          format     = \sffamily\bfseries\fontsize{12bp}{20bp}\selectfont,
          aftername  = \space,
        },
      }%
    \fi
  \fi
}
\bnu@set@section@format
\bnu@option@hook{degree}{\bnu@set@section@format}
\bnu@option@hook{main-language}{\bnu@set@section@format}
%    \end{macrocode}
%
%    \begin{macrocode}
\newcommand\bnu@english@number[1]{%
  \expandafter\ifcase\csname c@#1\endcsname
    Zero\or
    One\or
    Two\or
    Three\or
    Four\or
    Five\or
    Six\or
    Seven\or
    Eight\or
    Nine\or
    Ten\or
    Eleven\or
    Twelve\or
    Thirteen\or
    Fourteen\or
    Fifteen\or
    Sixteen\or
    Seventeen\or
    Eighteen\or
    Nineteen\or
    Twenty\or
    \bnu@error{You are genius}%
  \fi
}
%    \end{macrocode}
%
% \begin{macro}{\bnu@chapter*}
% 默认的 \cs{chapter*} 很难同时满足研究生院和本科生的论文要求。本科论文要求所有的
% 章都出现在目录里,比如摘要、Abstract、主要符号表等,所以可以简单的扩展默
% 认\cs{chapter*} 实现这个目的。但是研究生又不要这些出现在目录中,而且致谢和声明
% 部分的章名、页眉和目录都不同,所以定义一个灵活的 \cs{bnu@chapter*} 专门处理这些
% 要求。
%
% \cs{bnu@chapter*}\oarg{tocline}\marg{title}\oarg{header}: tocline 是出现在目录
% 中的条目,如果为空则此 chapter 不出现在目录中,如果省略表示目录出现 title;
% title 是章标题;header 是页眉出现的标题,如果忽略则取 title。通过这个宏我才真
% 正体会到 \TeX{} macro 的力量!
%    \begin{macrocode}
\newcommand\bnu@pdfbookmark[2]{}
\newcommand\bnu@phantomsection{}
\NewDocumentCommand\bnu@chapter{s o m o}{%
  \IfBooleanF{#1}{%
    \bnu@error{You have to use the star form: \string\bnu@chapter*}%
  }%
  \if@openright\cleardoublepage\else\clearpage\fi%
  \IfValueTF{#2}{%
    \ifthenelse{\equal{#2}{}}{%
      \bnu@pdfbookmark{0}{#3}%
    }{%
      \bnu@phantomsection
      \addcontentsline{toc}{chapter}{#2}%
    }%
  }{%
    \bnu@phantomsection
    \addcontentsline{toc}{chapter}{#3}%
  }%
  \ifbnu@degree@bachelor\ctexset{chapter/beforeskip=40bp}\fi
  \chapter*{#3}%
  \ifbnu@degree@bachelor\ctexset{chapter/beforeskip=30bp}\fi
  \IfValueTF{#4}{%
    \ifthenelse{\equal{#4}{}}{%
      \@mkboth{}{}%
    }{%
      \@mkboth{#4}{#4}%
    }%
  }{%
    \@mkboth{#3}{#3}%
  }%
}
%    \end{macrocode}
% \end{macro}
%
%
% \subsubsection{目录}
% \label{sec:toc}
% 最多 4 层,即: x.x.x.x,对应的命令和层序号分别是:
% \cs{chapter}(0), \cs{section}(1), \cs{subsection}(2), \cs{subsubsection}(3)。
%    \begin{macrocode}
\setcounter{secnumdepth}{3}
\setcounter{tocdepth}{2}
%    \end{macrocode}
%
% \begin{macro}{\tableofcontents}
% 目录生成命令。
%    \begin{macrocode}
\renewcommand\tableofcontents{%
  \ifbnu@degree@graduate 
    \ctexset{%
      chapter/titleformat = \centering\heiti\xiaoer,
    }%
    \bnu@chapter*[]{\contentsname}%
    \ctexset{%
      chapter/titleformat = {},
    }%
  \else
    \bnu@chapter*[]{\contentsname}%
  \fi
  \@starttoc{toc}%
}
\bnu@define@key{
  toc-chapter-style = {
    name = toc@chapter@style,
    choices = {
      times,
      arial,      
    },
    default = times,
  },
}
\newcommand\bnu@leaders{\titlerule*[4bp]{.}}
\newcommand\bnu@set@toc@format{%
  \contentsmargin{\z@}%
%    \end{macrocode}
%
% 本科生:
% 目录:三号黑体
% 章的标题用黑体三号字,行间距为 20pt,
% 条的标题用小三号黑体字
% 款的标题用四号黑体字
% 项的标题用小四号黑体字
% 行前空 6pt,行后空 0pt。
%
% 注意示例中章标题的字母和数字是衬线体,所以这里用 \cs{heiti}。
% 示例中的一级、二级和三级节标题分别缩进 1, 1.5 和 2 个汉字符。
%    \begin{macrocode}
  \ifbnu@degree@bachelor
    \ifbnu@main@language@chinese
      \titlecontents{chapter}
        [\z@]{\addvspace{6bp}
          \ifbnu@toc@chapter@style@arial
            \sanhao\sffamily
          \else
            \sanhao\heiti
          \fi
        }
        {\contentspush{\thecontentslabel\quad}}{}
        {\rmfamily\bnu@leaders\thecontentspage}%
      \titlecontents{section}
        [1em]{\xiaosan\heiti}
        {\contentspush{\thecontentslabel\quad}}{}
        {\bnu@leaders\thecontentspage}%
      \titlecontents{subsection}
        [1.5em]{\sihao\heiti}
        {\contentspush{\thecontentslabel\quad}}{}
        {\bnu@leaders\thecontentspage}%
      \titlecontents{subsubsection}
        [2em]{\xiaosi\heiti}
        {\contentspush{\thecontentslabel\quad}}{}
        {\bnu@leaders\thecontentspage}%
    \else
%    \end{macrocode}
%
% 本科生英文专业要求“左侧按级依次缩进 0.5cm”。
%    \begin{macrocode}
      \ifbnu@main@language@english
        \titlecontents{chapter}
          [\z@]{\addvspace{6bp}\sffamily}
          {\contentspush{\thecontentslabel\quad}}{}
          {\rmfamily\bnu@leaders\thecontentspage}%
        \titlecontents{section}
          [0.5cm]{}
          {\contentspush{\thecontentslabel\quad}}{}
          {\bnu@leaders\thecontentspage}%
        \titlecontents{subsection}
          [1cm]{}
          {\contentspush{\thecontentslabel\quad}}{}
          {\bnu@leaders\thecontentspage}%
      \fi
    \fi
%    \end{macrocode}
%
% 研究生:
% \begin{enumerate}
%   \item 目录这两个字是小二号,黑体,居中
%   \item 目录中的章标题行为小四号,宋体加粗,固定行距 20 磅,段前空 6 磅,
%     段后 0 磅;其他内容采用宋体小四号字,行距为固定值 20 磅,
%     段前、段后均为 0 磅。
%   \item 目录中的章标题行居左书写,一级节标题行缩进 1 个汉字符,
%     二级节标题行缩进 2 个汉字符。
% \end{enumerate}
%
% 注意示例中章标题的字母和数字是无衬线体,所以用这里用 \cs{sffamily},
% 但是页码仍然用 \cs{rmfamily}。
%    \begin{macrocode}
  \else
    \ifbnu@main@language@chinese
      \titlecontents{chapter}
        [\z@]{\addvspace{6bp}\bfseries\rmfamily}
        {\contentspush{\thecontentslabel\quad}}{}
        {\rmfamily\bnu@leaders\thecontentspage}%
      \titlecontents{section}
        [1em]{}
        {\contentspush{\thecontentslabel\quad}}{}
        {\bnu@leaders\thecontentspage}%
      \titlecontents{subsection}
        [2em]{}
        {\contentspush{\thecontentslabel\quad}}{}
        {\bnu@leaders\thecontentspage}%
    \else
      \titlecontents{chapter}
        [\z@]{\addvspace{6bp}\heiti}
        {\contentspush{\MakeUppercase{\thecontentslabel}\space}\MakeUppercase}{\MakeUppercase}
        {\rmfamily\bnu@leaders\thecontentspage}%
      \titlecontents{section}
        [1em]{}
        {\contentspush{\thecontentslabel\space}}{}
        {\bnu@leaders\thecontentspage}%
      \titlecontents{subsection}
        [2em]{}
        {\contentspush{\thecontentslabel\space}}{}
        {\bnu@leaders\thecontentspage}%
    \fi
  \fi
}
\bnu@set@toc@format
\bnu@option@hook{degree}{\bnu@set@toc@format}
\bnu@option@hook{main-language}{\bnu@set@toc@format}
%    \end{macrocode}
% \end{macro}
%
%
% \subsubsection{封面和封底}
% \label{sec:cover}
% 定义密级参数。
%    \begin{macrocode}
\bnu@define@key{
  secret-level = {
    name = secret@level,
  },
  secret-year = {
    name = secret@year,
  },
%    \end{macrocode}
%
% 论文中英文题目。
%    \begin{macrocode}
  title = {
    default = {标题},
  },
  title* = {
    default = {Title},
    name    = title@en,
  },
%    \end{macrocode}
%
% 作者、导师、副导师、联合指导老师。
%    \begin{macrocode}
  author = {
    default = {姓名},
  },
  author* = {
    default = {Name of author},
    name    = author@en,
  },
  student-id = {
    name = student@id,
    default = {201721130000}
  },
  supervisor = {
    default = {导师姓名},
  },
  supervisor* = {
    default = {Name of supervisor},
    name    = supervisor@en,
  },
  supervisor-department ={
  default = {指导教师单位},
  name = supervisor@department,
  },
  associate-supervisor = {
    name = associate@supervisor,
  },
  associate-supervisor* = {
    name = associate@supervisor@en,
  },
  co-supervisor = {
    name = co@supervisor,
  },
  co-supervisor* = {
    name = co@supervisor@en,
  },
  % Reserved for compatibility
  joint-supervisor = {
    name = co@supervisor,
  },
  joint-supervisor* = {
    name = co@supervisor@en,
  },
%    \end{macrocode}
%
% 学位中英文。
%    \begin{macrocode}
  degree-category = {
    default = {工学博士},
    name    = degree@category,
  },
  degree-category* = {
    default = {Doctor of Philosophy},
    name    = degree@category@en,
  },
  % 为了向后兼容
  degree-name = {
    name    = degree@category,
  },
  degree-name* = {
    name    = degree@category@en,
  },
}
\bnu@option@hook{degree-name}{%
  \bnu@warning{`degree-name' is deprecated. Use `degree-category' instead.}
}
\bnu@option@hook{degree-name*}{%
  \bnu@warning{`degree-name*' is deprecated. Use `degree-category*' instead.}
}
%    \end{macrocode}
%
% 院系中英文名称。
%    \begin{macrocode}
\bnu@define@key{
  department = {
    default = {数学科学学院},
  },
%    \end{macrocode}
%
% 学科中英文名称。
%    \begin{macrocode}
  discipline = {
    % default = {计算数学},
  },
  discipline* = {
    % default = {Computer Science and Technology},
    name = discipline@en,
  },
}
\bnu@option@hook{discipline}{%
  \ifbnu@degree@type@professional
    \bnu@warning{`discipline' for professional degree is deprecated. Use `professional-field' instead.}
    \let\bnu@professional@field\bnu@discipline
    \let\bnu@discipline\@empty
  \fi
}
\bnu@option@hook{discipline*}{%
  \ifbnu@degree@type@professional
    \bnu@warning{`discipline*' for professional degree is deprecated. Use `professional-field*' instead.}
    \let\bnu@professional@field@en\bnu@discipline@en
    \let\bnu@discipline@en\@empty
  \fi
}
%    \end{macrocode}
%
% 专业领域。
%    \begin{macrocode}
\bnu@define@key{
  professional-field = {
    name    = professional@field,
  },
  professional-field* = {
    name    = professional@field@en,
  },
%    \end{macrocode}
%
% 工程领域。
%    \begin{macrocode}
  engineering-field = {
    name    = engineering@field,
  },
  engineering-field* = {
    name    = engineering@field@en,
  },
%    \end{macrocode}
%
% 论文成文日期。
%    \begin{macrocode}
  date = {
    default = {\the\year-\two@digits{\month}-\two@digits{\day}},
  },
%    \end{macrocode}
%
% 博士后专用封面参数。
%    \begin{macrocode}
  clc,
  udc,
  id,
  discipline-level-1 = {
    default = {一级学科名称},
    name    = discipline@level@i,
  },
  discipline-level-2 = {
    default = {二级学科名称},
    name    = discipline@level@ii,
  },
  start-date = {
    name    = start@date,
    default = {\the\year-\two@digits{\month}-\two@digits{\day}},
  },
  end-date = {
    name    = end@date,
    default = {\the\year-\two@digits{\month}-\two@digits{\day}},
  },
%    \end{macrocode}
%
% 中文封面后是否生成书脊页。
%    \begin{macrocode}
  include-spine = {
    name = include@spine,
    choices = {
      false,
      true,
    },
    default = false,
  },
%    \end{macrocode}
%
% 中文封面后是否生成英文封面。
%    \begin{macrocode}
  include-titlepage-en = {
    name = include@titlepage@en,
    choices = {
      false,
      true,
    },
    default = false,
  },
}
%    \end{macrocode}
%
% 输出日期的给定格式:\cs{bnu@format@date}\marg{format}\marg{date},
% 其中格式 \meta{format} 接受三个参数分别对应年、月、日,
% \meta{date} 是 ISO 格式的日期(yyyy-mm-dd)。
%    \begin{macrocode}
\newcommand\bnu@format@date[2]{%
  \edef\bnu@@date{#2}%
  \def\bnu@@process@date##1-##2-##3\@nil{%
    #1{##1}{##2}{##3}%
  }%
  \expandafter\bnu@@process@date\bnu@@date\@nil
}
\newcommand\bnu@date@zh@digit[3]{#1 年 \number#2 月 \number#3 日}
\newcommand\bnu@date@zh@digit@short[3]{#1 年 \number#2 月}
\newcommand\bnu@date@zh@short[3]{\zhdigits{#1}年\zhnumber{#2}月}
\newcommand\bnu@date@month[1]{%
  \ifcase\number#1\or
    January\or February\or March\or April\or May\or June\or
    July\or August\or September\or October\or November\or December%
  \fi
}
\newcommand\bnu@date@en@short[3]{\bnu@date@month{#2}, #1}
%    \end{macrocode}
%
% 下划线命令。
% \pkg{ulem} 的下划线 \cs{uline} 可以控制粗细和深度。
%    \begin{macrocode}
\newcommand\bnu@underline[2][6em]{\hskip1pt\underline{\hb@xt@ #1{\hss#2\hss}}\hskip3pt}
\newcommand\bnu@uline[2][6em]{\uline{\hb@xt@ #1{\hss#2\hss}}}
%    \end{macrocode}
%
% 将内容拉伸或压缩到固定宽度。
%    \begin{macrocode}
\newcommand\bnu@fixed@box[2]{%
  \begingroup
    \ifLuaTeX
      \ltjsetparameter{kanjiskip = {0pt plus 2filll minus 1filll}}%
    \else
      \renewcommand\CJKglue{\hspace{0pt plus 2filll minus 1filll}}%
    \fi
    \makebox[#1][l]{#2}%
  \endgroup
}
%    \end{macrocode}
%
% 如果内容小于给定宽度,则拉伸至该宽度,否则取自然宽度。
%    \begin{macrocode}
\newbox\bnu@stretch@box
\newcommand\bnu@stretch[2]{%
  \sbox\bnu@stretch@box{#2}%
  \ifdim \wd\bnu@stretch@box < #1\relax
    \begingroup
      \ifLuaTeX
        \ltjsetparameter{kanjiskip = {0pt plus 2filll}}%
      \else
        \renewcommand\CJKglue{\hspace{0pt plus 2filll}}%
      \fi
      \makebox[#1][l]{#2}%
    \endgroup
  \else
    \box\bnu@stretch@box
  \fi
}
%    \end{macrocode}
%
% 如果内容小于给定宽度,则在右侧填充空白至该宽度,否则取自然宽度。
%    \begin{macrocode}
\newbox\bnu@pad@box
\newcommand\bnu@pad[2]{%
  \sbox\bnu@pad@box{#2}%
  \ifdim \wd\bnu@pad@box < #1\relax
    \makebox[#1][l]{\box\bnu@pad@box}%
  \else
    \box\bnu@pad@box
  \fi
}
%    \end{macrocode}
%
% 导师的姓名和职称使用“,”分开,所以这里用 \pkg{kvsetkeys} 的 \cs{comma@parse} 来处理。
%    \begin{macrocode}
\newcounter{bnu@csl@count}
\newcommand\bnu@name@title@process[1]{%
  \ifcase\c@bnu@csl@count  % == 0
    \gdef\bnu@@name{#1}%
  \or  % == 1
    \gdef\bnu@@title{#1}%
  \fi
  \stepcounter{bnu@csl@count}%
}
\newcommand\bnu@name@title@format[2]{%
  \bnu@pad{2.3cm}{\bnu@stretch{3.5em}{#1}}%
  \bnu@stretch{3em}{#2}%
}
\newcommand\bnu@name@title[1]{%
  \setcounter{bnu@csl@count}{0}%
  \gdef\bnu@@name{}%
  \gdef\bnu@@title{}%
  \expandafter\comma@parse\expandafter{#1}{\bnu@name@title@process}%
  \bnu@name@title@format{\bnu@@name}{\bnu@@title}%
}
%    \end{macrocode}
%
% \myentry{封面}
% \begin{macro}{\maketitle}
% 生成封面(题名页)总命令。
%    \begin{macrocode}
\renewcommand\maketitle{%
  \cleardoublepage
  \pagenumbering{Alph}%
  \bnu@pdfbookmark{-1}{\bnu@title}%
  \bnu@titlepage
  \ifbnu@include@spine@true
    \spine
  \fi
  \ifbnu@degree@graduate
    \ifbnu@thesis@type@thesis
      \ifbnu@include@titlepage@en@true
        \cleardoublepage
        \bnu@titlepage@en
      \fi
    \fi
  \fi
  \clearpage
}
%    \end{macrocode}
% \end{macro}
%
% \begin{macro}{\bnu@titlepage}
% 中文封面(题名页)
%
%    \begin{macrocode}
\newcommand\bnu@titlepage{%
  \bnusetup{language = chinese}%
  \ifbnu@degree@graduate
    % 研究生
    \ifbnu@thesis@type@thesis
      % 学位论文
      \bnu@titlepage@thesis
    \else
      \ifbnu@thesis@type@proposal
        % 选题报告
        \bnu@titlepage@proposal
      \fi
    \fi
  \else
    \ifbnu@degree@bachelor
      % 本科生
      \bnu@titlepage@bachelor
    \else
      \ifbnu@degree@postdoc
        % 博后
        \bnu@cover@postdoc
        \cleardoublepage
        \bnu@titlepage@postdoc
      \fi
    \fi
  \fi
  \bnu@reset@main@language
}
%    \end{macrocode}
% \end{macro}
%
% \myentry{研究生中文封面}
% 《写作指南》规定中文封面页边距:
% 上—6. 0 厘米,下—5.5 厘米,左—4.0 厘米,右—4.0 厘米,装订线 0 厘米。
% 然而作为事实标准的 Word 模板的页边距是上下 6.0 厘米,左右 4.0 厘米。
% 这里缩小上边距以方便排版保密信息。
%    \begin{macrocode}

\newcommand\bnu@titlepage@thesis{%
  \newgeometry{
    top     = 1.4cm,
    bottom  = 3.4cm,
    hmargin = 3.15cm,
  }%
  \thispagestyle{empty}%
  \null\vskip 8.1pt%
  \begingroup
    \centering
    \parbox[t][2cm][t]{\textwidth}{%
      \hskip -21.5pt%
      \bnu@titlepage@secret
    }  \par
    \includegraphics[width=0.6\textwidth]{BNU_name_blue.png}
    \vskip 3.85bp%
    \bnu@titlepage@degree
    \vskip 70.105bp%
    \begingroup
      \centering
      \bfseries\heiti\xiaoer\bnu@title \par
    \endgroup
    \ifbnu@main@language@english
      \vskip 5.4pt%
      \begingroup
        \sffamily\bfseries\fontsize{20bp}{31.2bp}\selectfont
        Thesis Title: \bnu@title@en\par
      \endgroup
      \vskip -9.2pt%
    \fi  
    \vfill
    \parbox[t][9.776cm][t]{\textwidth}{%
      \centering\bfseries\songti\sihao
      \bnu@titlepage@info
    }
    \par
  \endgroup
  \clearpage
  \restoregeometry
}
%    \end{macrocode}
%
% 涉密信息
%    \begin{macrocode}
\newcommand\bnu@titlepage@secret{%
  \sffamily\sanhao
  \ifx\bnu@secret@level\@empty
    \phantom{秘密}%
  \else
    \bnu@secret@level\symbol{"2605}\makebox[3em][c]{\bnu@secret@year}年%
  \fi\par
}
%    \end{macrocode}
%
% 申请学位的学科门类或专业学位类别: 三号(16bp)宋体字,字距延伸 0.5bp,
% 所以 \cs{CJKglue} 应该设为 1 bp。
%    \begin{macrocode}
\newcommand\bnu@title@page@degree@category{%
  \begingroup
    \fontsize{16bp}{22bp}\selectfont
    \ifLuaTeX
      \fontspec{\CJK@family}%
      \ltjsetparameter{kanjiskip = {1bp}}%
    \else
      \CJKfamily+{}%
      \renewcommand\CJKglue{\hspace{1bp}}%
    \fi
    \ifbnu@thesis@type@thesis
      (申请--大学\bnu@degree@category
      \ifbnu@degree@type@professional
        专业%
      \fi
      学位论文)%
    \else
      \ifbnu@thesis@type@proposal
        (--大学%
        \ifbnu@degree@doctor
          博士%
        \else
          \ifbnu@degree@master
            硕士%
          \fi
        \fi
        学位论文选题报告)%
      \fi
    \fi
    \par
  \endgroup
}
%    \end{macrocode}
%
% 论文类型
%    \begin{macrocode}
\newcommand\bnu@titlepage@degree{%
  \vskip 14bp
  \begingroup
    \CJKfamily+{}\bfseries\heiti\fontsize{42bp}{60.53bp}\selectfont%
    \def\CJKglue{\hskip 1bp}%
    \ifbnu@thesis@type@thesis
      \bnu@stretch{7.5550em}{%
        \ifbnu@degree@type@professional
          专业%
        \else
          学术%
        \fi
        \ifbnu@degree@doctor
          博士%
        \else
          \ifbnu@degree@master
            硕士%
          \fi
        \fi
        研究生%
      }\\
      \bnu@stretch{4.33em}{学位论文}%
    \else
      \ifbnu@thesis@type@proposal
        (北京师范大学%
        \ifbnu@degree@doctor
          博士%
        \else
          \ifbnu@degree@master
            硕士%
          \fi
        \fi
        学位论文选题报告)%
      \fi
    \fi
    \par
  \endgroup
}
%    \end{macrocode}

% 作者及指导教师信息
%    \begin{macrocode}
\newcommand\bnu@titlepage@info{%
  \bnu@titlepage@info@tabular{1.3cm}{2.45cm}{2.35cm}{0.77cm}{%
  \bnu@info@item{作者}{}{\bnu@pad{2.3cm}{\bnu@stretch{3.5em}{\bnu@author}}}%
  \bnu@info@item{导师}{\bnu@name@title}{\bnu@supervisor}%
  \bnu@info@item{副导师}{\bnu@name@title}{\bnu@associate@supervisor}%
  \bnu@info@item{联合导师}{\bnu@name@title}{\bnu@co@supervisor}%
  \bnu@info@item{培养单位}{}{\bnu@department}%
  \bnu@info@item{学号}{}{\bnu@student@id}%  
  \ifbnu@degree@type@academic
    \bnu@info@item{学科专业}{}{\bnu@discipline}%    
  \else
    \bnu@info@item{学科领域}{}{\bnu@professional@field}%
  \fi
  \bnu@info@item{完成日期}{}{\bnu@titlepage@date}%
 }\par
}
%    \end{macrocode}
%
% 标题页作者信息表
% \texttt{\#1}: 表格左侧至版心的距离;\\
% \texttt{\#2}: “培养单位”的边框宽度;\\
% \texttt{\#3}: “培养单位”的文字宽度;\\
% \texttt{\#4}: 冒号的边框;\\
% \texttt{\#5}: 表格内容。
%    \begin{macrocode}
\newcommand\bnu@titlepage@info@tabular[5]{%
  \def\bnu@info@item##1##2##3{%
    \ifx##3\@empty\else
      \bnu@pad{#2}{\bnu@fixed@box{#3}{##1}}%
      \bnu@pad{#4}{:}%
      & ##2{##3}\\
    \fi
  }%
  \begingroup
    \renewcommand\arraystretch{1.7477}%
    \setlength{\tabcolsep}{0pt}
    \begin{tabular}{lc}%
      #5%
    \end{tabular}%
  \endgroup
}
%    \end{macrocode}
%
% 论文成文打印的日期,用三号宋体汉字,字距延伸 0.5bp,
% 所以 \cs{CJKglue} 应该设为 1 bp。
%    \begin{macrocode}
\newcommand\bnu@titlepage@date{%
  \begingroup
    \sanhao
    \ifLuaTeX
      \ltjsetparameter{kanjiskip = {1bp}}%
    \else
      \renewcommand\CJKglue{\hspace{1bp}}%
    \fi
    \bnu@format@date{\bnu@date@zh@short}{\bnu@date}\par
  \endgroup
}
%    \end{macrocode}
%
% \myentry{研究生英文封面}
% 未修改,默认不显示。
% \begin{macro}{\bnu@titlepage@en}
%    \begin{macrocode}
\newcommand{\bnu@titlepage@en}{%
  \newgeometry{
    top     = 5.5cm,
    bottom  = 5cm,
    hmargin = 3.4cm,
  }%
  \thispagestyle{empty}%
  \bnusetup{language = english}%
  \ifbnu@degree@type@academic
    \bnu@titlepage@en@graduate@academic
  \else
    \bnu@titlepage@en@graduate@professional
  \fi
  \bnu@reset@main@language
  \clearpage
  \restoregeometry
}
\newcommand\bnu@titlepage@en@graduate@academic{%
  \begingroup
    \centering
    \null\vskip -0.31cm%
    \parbox[t][143bp][t]{\textwidth}{%
      \centering\bnu@titlepage@en@title
    }\par
    \sanhao[1.725]%
    \bnu@titlepage@en@degree
    \vskip 3bp%
    in\par
    \vskip 3.5bp%
    {\bfseries\sffamily\bnu@discipline@en\par}
    \vfill
    {\sffamily by\par}
    \vskip 0.24cm%
    {\sffamily\bfseries\bnu@author@en\par}%
    \vskip 0.18cm%
    \parbox[t][3.0cm][t]{\textwidth}{%
      \centering
      \xiaosan[2.1]%
      \bnu@titlepage@en@supervisor
    }\par
    \bnu@titlepage@en@date
    \vskip 0.7cm%
  \endgroup
}
\newcommand\bnu@titlepage@en@graduate@professional{%
  \begingroup
    \centering
    \null\vskip -0.31cm%
    \parbox[t][143bp][t]{\textwidth}{%
      \centering\bnu@titlepage@en@title
    }\par
    \sanhao[1.725]%
    \bnu@titlepage@en@degree
    \vfill
    {\sffamily by\par}
    \vskip 0.24cm%
    {\sffamily\bfseries\bnu@author@en\par}%
    \ifx\bnu@professional@field@en\empty
      \vskip 1.95cm%
    \else
      \vskip -0.1cm%
      {\sffamily\bfseries(\bnu@professional@field@en)\par}%
      \vskip 1.1cm%
    \fi
    \parbox[t][3.37cm][t]{\textwidth}{%
      \centering
      \xiaosan[1.82]%
      \bnu@titlepage@en@supervisor
    }\par
    \bnu@titlepage@en@date
    \vskip 0.3cm%
  \endgroup
}
\newcommand\bnu@titlepage@en@title{%
  \begingroup
    % 对齐到网格,每行 15.6bp
    \sffamily\bfseries\fontsize{20bp}{31.2bp}\selectfont
    \bnu@title@en\par
  \endgroup
}
\newcommand\bnu@thesis@name@en{%
  \ifbnu@degree@master
    Thesis%
  \else
    Dissertation%
  \fi
}
\newcommand\bnu@titlepage@en@degree{%
  \bnu@thesis@name@en{} submitted to\par
  {\bfseries Tsinghua University\par}%
  in partial fulfillment of the requirement\par
  for the
  \ifbnu@degree@type@professional
    professional
  \fi
  degree of\par
  {\sffamily\bfseries\bnu@degree@category@en\par}%
}
\newcommand\bnu@titlepage@en@supervisor{%
  \begin{tabular}{r@{\makebox[20.5bp][l]{\hspace{2bp}:}}l}%
    \renewcommand\arraystretch{1}%
    \bnu@thesis@name@en{} Supervisor & \bnu@supervisor@en \\
    \ifx\bnu@associate@supervisor@en\@empty\else
      Associate Supervisor           & \bnu@associate@supervisor@en \\
    \fi
    \ifx\bnu@co@supervisor@en\@empty\else
      Co-supervisor                  & \bnu@co@supervisor@en \\
    \fi
  \end{tabular}%
}
\newcommand\bnu@titlepage@en@date{%
  \begingroup
    \sffamily\bfseries\sanhao
    \bnu@format@date{\bnu@date@en@short}{\bnu@date}\par
  \endgroup
}
%    \end{macrocode}
% \end{macro}
%
% \myentry{本科生封面}
% 本科生封面要求:
% \begin{itemize}
%   \item 论文题目为三号宋体加粗。
%   \item 院(系)、专业、学号、姓名、指导教师等为四号宋体加粗。
%   \item 日期为小四号宋体。
% \end{itemize}
%
% 外文系英语专业要求题目先写中文标题,再写英文标题,字号 26pt,32 磅行距。
%    \begin{macrocode}
\newcommand\bnu@titlepage@bachelor{%
  \newgeometry{
    left = 2.5cm,
    top = 2.5cm,
    right = 2.0cm,
    bottom = 2.0cm,
  }%
  \thispagestyle{empty}%
  \begingroup
    \centering
    \parbox[t][0cm][t]{\textwidth}{%
      \hfill
      \xiaosi
      \ifx\bnu@secret@level\@empty\else
        \bnu@secret@level\space\bnu@secret@year 年\par
      \fi
    }%
  \endgroup
  \vfill
  \begingroup
    \centering
    \includegraphics[scale=1.13]{bnu-name-bachelor.pdf}%
    \vskip 0.4cm%
    {\bfseries\xiaochu\songti\ziju{0.13} 本科生毕业论文(设计)\par}%
  \endgroup
  \vskip 2.2cm%
  \begingroup
    \heiti
    \noindent\hspace{3em}\makebox[16em]{\sanhao[1.1]\textbf{毕业论文(设计)题目:}}\par
    \vskip 0.3cm%
    \parbox[t]{15cm}{%
      \ifbnu@main@language@chinese
        \sanhao[1.8]\songti%
      \else
        \fontsize{14bp}{21.84bp}\heiti\selectfont
      \fi
      \newlength{\titlelength}
      \settowidth{\titlelength}{\bnu@title}
      \ifdim\titlelength < 405pt
        \CJKunderline*[textformat=\bfseries, skip=false, thickness=0.05em, depth=0.12em]{\hfill\bnu@title\hfill}\par%
        \CJKunderline*[textformat=\bfseries, skip=false, thickness=0.05em, depth=0.12em]{\hfill}%
      \else
        \CJKunderline*[textformat=\bfseries, skip=false, thickness=0.05em, depth=0.12em]{\hfill\bnu@title\hfill}\par% 
      \fi 
      \ifbnu@main@language@english
        \bnusetup{language=english}%
        \expandafter\uline\expandafter{\bnu@title@en}\par
        \bnusetup{language=chinese}%
      \fi
    }\par
  \endgroup
  \vskip 1.9cm%
  \begingroup
    \ifx\bnu@co@supervisor\@empty
      \bfseries\songti\sihao[2.32]%
    \else
      \bfseries\songti\sihao[1.62]%
    \fi
    \leftskip=4.8cm%
    \parindent=\z@
    \def\bnu@info@item##1##2##3{%
      \ifx##3\@empty\else
        \bnu@fixed@box{%
          \ifx\bnu@co@supervisor\@empty
            6em%
          \else
            7.5em%
          \fi
        }{##1}:##2{##3}\par
      \fi
    }%
    \def\bnu@name@title@format##1##2{%
      \bnu@stretch{3em}{##1}\quad ##2%
    }%
    \bnu@info@item{部院系}{}{\bnu@department}%
    \bnu@info@item{专业}{}{\bnu@discipline}%
    \bnu@info@item{学号}{}{\bnu@student@id}%
    \bnu@info@item{学生姓名}{\bnu@name@title}{\bnu@author}%
    \def\bnu@name@title@format##1##2{%
      \bnu@stretch{3em}{##1}%
    }%
    \bnu@info@item{指导教师}{\bnu@name@title}{\bnu@supervisor}%
    \def\bnu@name@title@format##1##2{%
      \bnu@stretch{3em}{##2}%
    }%
    \bnu@info@item{指导教师职称}{\bnu@name@title}{\bnu@supervisor}%
    \def\bnu@name@title@format##1##2{%
      \bnu@stretch{3em}{##1}%
    }%
    \bnu@info@item{联合指导教师}{\bnu@name@title}{\bnu@co@supervisor}%
    \def\bnu@name@title@format##1##2{%
      \bnu@stretch{3em}{##2}%
    }%
    \bnu@info@item{联合指导教师职称}{\bnu@name@title}{\bnu@co@supervisor}%
    \bnu@info@item{指导教师单位}{}{\bnu@supervisor@department}%
  \endgroup
  \vskip 2.6cm%
  \begingroup
    \centering
    \ifLuaTeX
      \fontspec{\CJK@family}%
    \else
      \CJKfamily+{}%
    \fi
    \xiaosi\bnu@format@date{\bnu@date@zh@digit}{\bnu@date}\par
  \endgroup
  \vfill
  \clearpage
  \restoregeometry
}
%    \end{macrocode}
%
% \myentry{博士后封面}
% 无上述内容
%    \begin{macrocode}
\newcommand\bnu@cover@postdoc{%
  \thispagestyle{empty}%
  \begin{center}%
    \renewcommand\ULthickness{0.7pt}%
    \vspace*{0.35cm}%
    {\sihao[2.6]%
      \bnu@stretch{3.1em}{分类号}\bnu@underline[3.7cm]{\bnu@clc}\hfill
      密级\bnu@underline[3.7cm]{\bnu@secret@level}\par
      \bnu@stretch{3.1em}{U D C}\bnu@underline[3.7cm]{\bnu@udc}\hfill
      编号\bnu@underline[3.7cm]{\bnu@id}\par
    }%
    \vskip 3.15cm%
    {\sffamily\bfseries\xiaoer[2.6]%
      {\ziju{1.5}--大学\par}%
      {\ziju{0.5}博士后研究工作报告\par}%
    }%
    \vskip 0.2cm%
    \parbox[t][4.0cm][c]{\textwidth}{%
      \centering\sihao[3.46]%
      \renewcommand\ULdepth{1em}%
      \expandafter\uline\expandafter{\bnu@title}\par
    }\par
    \vskip 0.4cm%
    {\xiaosi\bnu@author\par}%
    \vskip 1.4cm%
    {\xiaosi[1.58]%
      \renewcommand\ULdepth{0.9em}%
      工作完成日期\quad
      \bnu@uline[5.9cm]{%
        \bnu@format@date{\bnu@date@zh@digit@short}{\bnu@start@date}—%
        \bnu@format@date{\bnu@date@zh@digit@short}{\bnu@end@date}
      }\par
      \vskip 0.55cm%
      报告提交日期\quad
      \bnu@uline[5.9cm]{\bnu@format@date{\bnu@date@zh@digit@short}{\bnu@date}}\par
    }%
    \vskip 0.45cm%
    {\xiaosi[2]{\ziju{1}--大学}\quad (北京)\par}%
    \vskip 0.25cm%
    {\xiaosi[2]\bnu@format@date{\bnu@date@zh@digit@short}{\bnu@date}\par}%
  \end{center}%
}
%    \end{macrocode}
%
% \myentry{博士后题名页}
%    \begin{macrocode}
\newcommand\bnu@titlepage@postdoc{%
  \thispagestyle{empty}%
  \begin{center}%
    \vspace*{1.5cm}%
    \parbox[t][3cm][c]{\textwidth}{%
      \centering\sanhao[1.95]\bnu@title\par
    }\par
    \vskip 0.15cm%
    \parbox[t][3cm][c]{\textwidth}{%
      \centering\sihao[1.36]\bnu@title@en\par
    }\par
    \vskip 0.4cm%
    {\xiaosi[2.6]%
      \begin{tabular}{l@{\quad}l}%
        \renewcommand\arraystretch{1}%
        \bnu@stretch{11em}{博士后姓名}                  & \bnu@author           \\
        \bnu@stretch{11em}{流动站(一级学科)名称}      & \bnu@discipline@level@i  \\
        \bnu@stretch{11em}{专\quad{}业(二级学科)名称} & \bnu@discipline@level@ii \\
      \end{tabular}\par
    }%
    \vskip 2.7cm%
    {\xiaosi[2.6]%
      研究工作起始时间\quad\bnu@format@date{\bnu@date@zh@digit}{\bnu@start@date}\par
      \vskip 0.1cm%
      研究工作期满时间\quad\bnu@format@date{\bnu@date@zh@digit}{\bnu@end@date}\par
    }%
    \vskip 2.1cm%
    {\xiaosi[2.6]--大学人事处(北京)\par}%
    \vskip 0.6cm%
    {\wuhao\bnu@format@date{\bnu@date@zh@digit@short}{\bnu@date}\par}%
  \end{center}%
}
%    \end{macrocode}
%
% \subsubsection{答辩委员会名单}
% 无上述部分
% \begin{environment}{committee}
% 学位论文指导小组、公开评阅人和答辩委员会名单。
%    \begin{macrocode}
\def\bnu@committee@name{学位论文指导小组、公开评阅人和答辩委员会名单}
\NewEnviron{committee}[1][]{%
  \ifbnu@degree@graduate
    \cleardoublepage
    \let\bnu@committee@file\@empty
    \kv@define@key{bnu@committee}{name}{\let\bnu@committee@name\kv@value}%
    \kv@define@key{bnu@committee}{file}{\let\bnu@committee@file\kv@value}%
    \kv@set@family@handler{bnu@committee}{%
      \ifx\kv@value\relax
        \let\bnu@committee@file\kv@key
      \else
        \kv@handled@false
      \fi
    }%
    \kvsetkeys{bnu@committee}{#1}%
    \ifx\bnu@committee@file\@empty
      \begingroup
        \ctexset{
          chapter = {
            format    = \centering\sffamily\fontsize{16bp}{20bp}\selectfont,
            afterskip = 49bp,
          },
          section = {
            beforeskip  =  26bp,
            afterskip   =  9.5bp,
            format      += \centering,
            numbering   =  false,
            afterindent =  false,
          },
        }%
        \bnu@chapter*[]{\bnu@committee@name}%
        \thispagestyle{empty}%
        \bnusetup{language=chinese}%
        \BODY\clearpage
        \bnu@reset@main@language
      \endgroup
    \else
      \bnu@pdfbookmark{0}{\bnu@committee@name}%
      \includepdf{\bnu@committee@file}%
    \fi
  \fi
}
%    \end{macrocode}
% \end{environment}
%
% \subsubsection{授权说明和原创性说明}
% \begin{macro}{\copyrightpage}
% 关于学位论文使用授权的说明,学位论文原创性声明。
%    \begin{macrocode}
\newcommand\copyrightpage[1][]{%
  \cleardoublepage
  \ifbnu@degree@postdoc\relax\else
    \def\bnu@@tmp{#1}
    \ifx\bnu@@tmp\@empty
      \bnusetup{language=chinese}%
      \ifbnu@degree@bachelor
        \bnu@copyright@page@bachelor
      \else
        \bnu@copyright@page@graduate
      \fi
      \bnu@reset@main@language
    \else
      \thispagestyle{empty}%
      \bnu@pdfbookmark{0}{关于学位论文使用授权的说明}%
      \bnu@phantomsection
      \kv@define@key{bnu@copyright}{file}{\includepdf{\kv@value}}%
      \kv@set@family@handler{bnu@copyright}{%
        \ifx\kv@value\relax
          \includepdf{\kv@key}%
        \else
          \kv@handled@false
        \fi
      }%
      \kvsetkeys{bnu@copyright}{#1}%
    \fi
  \fi
}
%    \end{macrocode}
%
% 支持扫描文件替换。
%    \begin{macrocode}
\newcommand{\bnu@authorization@frontdate}{%
  日\ifbnu@degree@bachelor\hspace{1em}\else\hspace{2em}\fi 期:}
\newcommand\bnu@copyright@page@graduate{%
  \begingroup
    \ctexset{
      chapter = {
        format     = {\centering\sffamily\erhao},
        beforeskip = 0bp,
        afterskip  = 0bp,
      },
    }%
    \bnu@chapter*[]{}%
    \thispagestyle{empty}%
  \endgroup
  \vskip 23bp%
  \begingroup
    \fontsize{16bp}{26bp}\selectfont
    \centering
    \textbf{北京师范大学学位论文原创性声明}\par
  \endgroup
  \vskip 16bp%
  \begingroup
    \fontsize{12bp}{23.4bp}\selectfont
    本人郑重声明: 所呈交的学位论文,是本人在导师的指导下,独立进行研究工作所取得的成果。
    除文中已经注明引用的内容外,本论文不含任何其他个人或集体已经发表或撰写过的作品成果。
    对本文的研究做出重要贡献的个人和集体,均已在文中以明确方式标明。
    本人完全意识到本声明的法律结果由本人承担。 \par
  \endgroup
  \vskip 24bp%
  \vskip 24bp%
  \begingroup
    \fontsize{12bp}{23.4bp}\selectfont
    \parindent\z@
    \leftskip 24bp%
    学位论文作者签名:\hspace{7em}%
    日期:\hspace{4em}年\hspace{2em}月\hspace{2em}日%
    \vskip 6bp%
    \par
    \endgroup
    \vskip 100bp%
  \begingroup
    \centering
    \fontsize{16bp}{26bp}\selectfont
    \textbf{学位论文使用授权书}\par
  \endgroup
  \vskip 16bp%
  \begingroup
    \fontsize{12bp}{23.4bp}\selectfont
    学位论文作者完全了解北京师范大学有关保留和使用学位论文的规定,即:
    学校有权保留并向国家有关部门或机构送交论文的复印件和电子版,
    允许学位论文被查阅和借阅;
    学校可以公布学位论文的全部或部分内容,可以允许采用影印、缩印或其它复制手段保存、
    汇编学位论文。保密的学位论文在解密后适用于本授权书。 \par
  \endgroup
  \vskip 33bp%
  \begingroup
    \fontsize{12bp}{23.4bp}\selectfont
    \parindent\z@
    \leftskip 43bp%
    本人签字:\hspace{4bp}\bnu@underline[7em]{}\hspace{47bp}%
    日\hspace{2em}期:\hspace{4bp}\bnu@underline[7em]{}\hspace{47bp}\par%
    \vskip 8bp%
    导师签名:\hspace{4bp}\bnu@underline[7em]{}\hspace{47bp}%
    日\hspace{2em}期:\hspace{4bp}\bnu@underline[7em]{}\par
  \endgroup
}
\newcommand\bnu@copyright@page@bachelor{%
  \begingroup
    \ctexset{
      chapter = {
        format     = {\centering\sffamily\erhao[1]},
        beforeskip = 1bp,
        afterskip  = 1bp,
      },
    }%
    \bnu@chapter*[]{}%
    \thispagestyle{empty}%
  \endgroup
  \vskip 4bp%
  \begingroup
    \centering \xiaosi
    \textbf{北京师范大学本科毕业论文(设计)诚信承诺书}\par
  \endgroup
  \vskip 12bp%
  \begingroup
    \songti\xiaosi[1.5]
    本人郑重声明: 所呈交的毕业论文(设计),是本人在导师的指导下,独立进行研究工作所取得的成果。
    除文中已经注明引用的内容外,本论文不含任何其他个人或集体已经发表或撰写过的作品成果。
    对本文的研究做出重要贡献的个人和集体,均已在文中以明确方式标明。
    本人完全意识到本声明的法律结果由本人承担。\par
  \endgroup
  \vskip 48bp%
  \begingroup
    \fontsize{12bp}{23.4bp}\selectfont
    \parindent\z@
    \leftskip 24bp%
    本人签名:\hspace{10em}%
    \hspace{4em}年\hspace{2em}月\hspace{2em}日%
    \vskip 6bp%
    \par
  \endgroup
  \vskip 120bp%
  \begingroup
    \centering
    \xiaosi
    \textbf{北京师范大学本科毕业论文(设计)使用授权书}\par
  \endgroup
  \vskip 12bp%
  \begingroup
    \songti\xiaosi[1.5]
    本人完全了解北京师范大学有关收集、保留和使用毕业论文(设计)的规定,
    即:本科生毕业论文(设计)工作的知识产权单位属北京师范大学。
    学校有权保留并向国家有关部门或机构送交论文的复印件和电子版,
    允许毕业论文(设计)被查阅和借阅;学校可以公布毕业论文(设计)的全部或部分内容,
    可以采用影印、缩印或扫描等复制手段保存、汇编毕业论文(设计)。
    保密的毕业论文(设计)在解密后遵守此规定。\par
  \endgroup
  \vskip 24bp%
  \begingroup
    \songti\xiaosi
    本论文(是、否)保密论文。\par
    保密论文在\bnu@underline[4em]{}年\bnu@underline[2em]{}月解密后适用本授权书。\par
    \endgroup
    \vskip 36bp%
  \begingroup
    \fontsize{12bp}{23.4bp}\selectfont
    \parindent\z@
    \leftskip 24bp%
    本人签名:\hspace{10em}%
    \hspace{4em}年\hspace{2em}月\hspace{2em}日%
    \vskip 12bp%
    \par
  \endgroup
  \begingroup
    \fontsize{12bp}{23.4bp}\selectfont
    \parindent\z@
    \leftskip 24bp%
    导师签名:\hspace{10em}%
    \hspace{4em}年\hspace{2em}月\hspace{2em}日%
    \par
  \endgroup  
}
%    \end{macrocode}
% \end{macro}
%
% \subsubsection{摘要}
% \label{sec:abstractformat}
%
% \begin{macro}{\bnu@clist@use}
% 不同论文格式关键词之间的分割不太相同,我们用 \option{keywords} 和
% \option{keywords*} 来收集关键词列表,然后用本命令来生成符合要求的格式,
% 类似于 \LaTeX3 的 \cs{clist\_use:Nn}。
%    \begin{macrocode}
\bnu@define@key{
  keywords,
  keywords* = {
    name = keywords@en,
  },
}
\newcommand\bnu@clist@use[2]{%
  \def\bnu@@tmp{}%
  \def\bnu@clist@processor##1{%
    \ifx\bnu@@tmp\@empty
      \def\bnu@@tmp{#2}%
    \else
      #2%
    \fi
    ##1%
  }%
  \expandafter\comma@parse\expandafter{#1}{\bnu@clist@processor}%
}
%    \end{macrocode}
% \end{macro}
%
% \begin{environment}{abstract}
% 研究生中文摘要的标题为小三号黑体,居中。
% 摘要内容,为小四号宋体,段落按照“首行缩进”格式,每段开头左缩进2个汉字符,标点符号占一格。
% 本科生毕业论文中文摘要:小三号黑体,摘要内容:四号宋体,行距20 磅;
% 英文摘要:小三号,摘要内容:四号,Times New Roman字体,单倍行距
%    \begin{macrocode}
\newenvironment{abstract}{%
  \bnusetup{language = chinese}%
  \renewcommand\headrulewidth{0pt}%
  \fancyhead[C]{}%
  \ifbnu@degree@graduate
    \clearpage
    \begingroup      
      \ctexset{%
        chapter = {
          break   = {},
          format  = {\centering\heiti\sanhao},
          afterskip  = 20bp,
        }
      }%
      \ifbnu@main@language@english
        \ctexset{%
          chapter/format = \centering\sffamily\fontsize{16bp}{20bp}\selectfont,
        }%
      \fi
      \bnu@chapter*[]{\bnu@title}[\bnu@abstract@name]%
    \endgroup
    \begingroup
      \centering
      \xiaosan\heiti \bnu@abstract@name \par%
      \vskip 15bp
    \endgroup
    \xiaosi[1.667]
  \else
    \ifbnu@degree@bachelor
      \clearpage
      \begingroup
        \ctexset{%
          chapter = {
            break   = {},
            format  = {\centering\bfseries\heiti\sanhao},
            afterskip  = 35.6bp,
          }
        }%
        \bnu@chapter*[\bnu@abstract@name]{\bnu@title}[\bnu@abstract@name]%
      \endgroup
      \begingroup
        \centering
        \bfseries\xiaosan\heiti \bnu@abstract@name \par%
        \vskip 15bp
      \endgroup
      \sihao
    \else
      \bnu@chapter*[]{\bnu@abstract@name}%
    \fi
  \fi
}{%
%    \end{macrocode}
%
% 研究生
% 关键词(四号黑体):3~8个关键词,逗号间隔。(小四号宋体)
% 本科生
% 关键词:四号黑体,关键词内容:四号宋体;
%    \begin{macrocode}
  \par
  \null\par
  \ifbnu@degree@graduate
    \noindent
    \textsf{\sihao 关键词:}%
  \else
    \textbf{关键词:}%
  \fi
  \bnu@clist@use{\bnu@keywords}{,}%
  \gdef\bnu@keywords{}%
  \ifbnu@degree@bachelor
    \cleardoublepage
  \fi
  \bnu@reset@main@language % switch back to main language
}
%    \end{macrocode}
% \end{environment}
%
% \begin{environment}{abstract*}
% 研究生英文摘要
% 页眉:ABSTRACT:Times New Roman,五号,居中;
% 英文题目:三号,全部大写字母,居中;
% ABSTRACT:四号加粗,居中;
% 英文题目下空三行,打“ABSTRACT”,再下来空两行打印英文摘要内容,
% 英文一律采用“Times New Roman”字体,小四号。
% 本科生英文摘要
% 英文摘要(小三号),摘要内容:四号,Times New Roman 字体,单倍行距。
%
%    \begin{macrocode}
\newenvironment{abstract*}{%
  \ifbnu@degree@bachelor
    \cleardoublepage
  \fi
  \bnusetup{language = english}%
  \ifbnu@degree@graduate
    \begingroup
      \ctexset{%
          chapter = {
          break   = {},
          format  = {\centering\bfseries\sanhao},
          afterskip  = 54bp,
        }
      }%
      \bnu@chapter*[]{\bnu@title@en}[\bnu@abstract@name@en]%
    \endgroup
    \begingroup
      \centering
      \xiaosan\bfseries \bnu@abstract@name@en \par%
      \vskip 30bp
    \endgroup
  \else
    \begingroup
      \ctexset{%
          chapter = {
          break   = {},
          format  = {\centering\bfseries\sanhao},
          afterskip  = 35.6bp,
        }
      }%
      \bnu@chapter*[\bnu@abstract@name@en]{\MakeUppercase{\bnu@title@en}}[\bnu@abstract@name@en]%
    \endgroup
    \begingroup
      \centering
      \sihao\bfseries \bnu@abstract@name@en \par%
      \vskip 24bp
    \endgroup
  \fi
}{%
  \par
  \null\par
  \ifbnu@degree@graduate
    \vskip 24bp
    \noindent
    \textbf{KEY WORDS:}\space
    \bnu@clist@use{\bnu@keywords@en}{, }%
    \vspace*{\stretch{1}}%
  \fi
  \ifbnu@degree@bachelor
    \textbf{KEY WORDS:}\space
    \bnu@clist@use{\bnu@keywords@en}{, }%  
    \cleardoublepage
  \fi
  \bnu@reset@main@language % switch back to main language
}
%    \end{macrocode}
% \end{environment}
%
% \subsubsection{主要符号表}
% \label{sec:denotationfmt}
% \begin{environment}{denotation}
% 主要符号表。
%    \begin{macrocode}
\newenvironment{denotation}[1][2.5cm]{%
  \ifbnu@degree@bachelor
    \cleardoublepage
  \fi
  \ifbnu@degree@graduate
    \bnu@chapter*{\bnu@denotation@name}%
  \else
    \bnu@chapter*[]{\bnu@denotation@name}%
  \fi
  \vskip-30bp\xiaosi[1.6]\begin{bnu@denotation}[labelwidth=#1]
}{%
  \end{bnu@denotation}
}
\newlist{bnu@denotation}{description}{1}
\setlist[bnu@denotation]{%
  nosep,
  font=\normalfont,
  align=left,
  leftmargin=!, % sum of the following 3 lengths
  labelindent=0pt,
  labelwidth=2.5cm,
  labelsep*=0.5cm,
  itemindent=0pt,
}
%    \end{macrocode}
% \end{environment}
%
%
% \subsubsection{致谢以及声明}
% \label{sec:ackanddeclare}
%
% \begin{environment}{acknowledgements}
% 定义致谢环境
%    \begin{macrocode}
\newcommand{\bnu@statement@text}{本人郑重声明:所呈交的学位论文,是本人在导师指导下
  ,独立进行研究工作所取得的成果。尽我所知,除文中已经注明引用的内容外,本学位论
  文的研究成果不包含任何他人享有著作权的内容。对本论文所涉及的研究工作做出贡献的
  其他个人和集体,均已在文中以明确方式标明。}
\newcommand{\bnu@signature}{签\hspace{1em}名:}
\newcommand{\bnu@backdate}{日\hspace{1em}期:}
%    \end{macrocode}
%
% 定义致谢与声明环境。
%    \begin{macrocode}
\newenvironment{acknowledgements}{%
  \@mainmatterfalse
  \bnu@end@appendix@ref@section
  \ifbnu@degree@bachelor
    \cleardoublepage
  \fi
  \bnu@chapter*{\bnu@acknowledgements@name}%
}{%
  \ifbnu@degree@bachelor
    \par
    \hfill
    \begin{minipage}[t]{8em}
      \centering
      \sanhao\sffamily
      \bnu@author\newline
      \bnu@format@date{\bnu@date@zh@digit@short}{\bnu@date}
    \end{minipage}
    \clearpage
  \else
    \ifbnu@degree@graduate
      \par
      \hfill
      {%
        \sihao
        \begin{minipage}[t]{6.5em}
          \centering
          \sihao
          % \heiti
          \quad\bnu@author\newline
          \mbox{\bnu@format@date{\bnu@date@zh@digit@short}{\bnu@date}}
        \end{minipage}
      }
      \clearpage
    \fi
  \fi
}
%    \end{macrocode}
% \end{environment}
%
% \begin{macro}{statement}
% 声明部分(支持扫描文件替换)
%    \begin{macrocode}
\bnu@define@key{
  statement-page-style = {
    name = statement@page@style,
    choices = {
      auto,
      empty,
      plain,
    },
    default = auto,
  },
  statement-page-number = {
    name = statement@page@number,
    choices = {
      false,
      true,
    },
    default = false,
  },
}
\bnu@option@hook{statement-page-number}{%
  \ifbnu@statement@page@number@false
    \bnusetup{statement-page-style=empty}%
  \else
    \bnusetup{statement-page-style=plain}%
  \fi
  \bnu@warning{%
    The "statement-page-number" option is deprecated.
    Use "page-style" option of \protect\statement command instead%
  }%
}
\newif\ifbnu@statement@exists
\newcommand\statement[1][]{%
  \@mainmatterfalse
  \bnu@end@appendix@ref@section
  \bnu@statement@existstrue
  \ifbnu@degree@bachelor
    \cleardoublepage
    \def\bnu@statement@name{声\hspace{2em}明}%
  \else
    \def\bnu@statement@name{声\hspace{1em}明}%
  \fi
  \let\bnu@statement@file\@empty
  \kv@define@key{bnu@statement}{page-style}{\bnusetup{statement-page-style=##1}}%
  \kv@define@key{bnu@statement}{file}{\let\bnu@statement@file\kv@value}%
  \kv@set@family@handler{bnu@statement}{%
    \ifx\kv@value\relax
      \let\bnu@statement@file\kv@key
    \else
      \kv@handled@false
    \fi
  }%
  \kvsetkeys{bnu@statement}{#1}%
  \ifbnu@statement@page@style@auto
    \ifx\bnu@statement@file\@empty
      \ifbnu@degree@bachelor
        \bnusetup{statement-page-style = empty}%
      \else
        \bnusetup{statement-page-style = plain}%
      \fi
    \else
      \ifbnu@degree@bachelor
        \bnusetup{statement-page-style = plain}%
      \else
        \bnusetup{statement-page-style = empty}%
      \fi
    \fi
  \fi
  \ifx\bnu@statement@file\@empty
    \bnusetup{language=chinese}%
    \begingroup
      \ifbnu@degree@graduate
        \ifbnu@main@language@english
          \ctexset{%
            chapter/format = \centering\sffamily\fontsize{16bp}{20bp}\selectfont,
          }%
        \fi
      \fi
      \bnu@chapter*{\bnu@statement@name}%
    \endgroup
    \thispagestyle{\bnu@statement@page@style}%
    \bnu@statement@text\par
    \ifbnu@degree@graduate
      \vskip 2cm%
    \else
      \null\par
    \fi
    {\hfill\bnu@signature\bnu@underline[2.5cm]\relax
      \bnu@backdate\bnu@underline[2.5cm]\relax}%
    \bnu@reset@main@language
  \else
    \includepdf[pagecommand={%
      \markboth{\bnu@statement@name}{}%
      \bnu@phantomsection
      \addcontentsline{toc}{chapter}{\bnu@statement@name}%
      \thispagestyle{\bnu@statement@page@style}%
    }]{\bnu@statement@file}%
  \fi
  \ifbnu@degree@bachelor
    \cleardoublepage
  \fi
}
%    \end{macrocode}
% \end{macro}
%
% 兼容旧版本保留 \env{acknowledgement}。
%    \begin{macrocode}
\let\acknowledgement\acknowledgements
\let\endacknowledgement\endacknowledgements
%    \end{macrocode}
%
% \subsubsection{插图和附表清单}
% \label{sec:threelists}
% 定义图表以及公式目录样式。
%    \begin{macrocode}
\def\bnu@listof#1{% #1: float type
  \setcounter{tocdepth}{2} % restore tocdepth in case being modified
  \@ifstar
    {\bnu@chapter*[]{\csname list#1name\endcsname}\@starttoc{\csname ext@#1\endcsname}}
    {\bnu@chapter*{\csname list#1name\endcsname}\@starttoc{\csname ext@#1\endcsname}}%
}
%    \end{macrocode}
%
% \begin{macro}{\listoffigures}
% \begin{macro}{\listoffigures*}
% 插图清单。
%    \begin{macrocode}
\renewcommand\listoffigures{%
  \ifbnu@degree@bachelor
    \ifbnu@backmatter\else
      \bnu@warning{The list of figures should be placed in back matter}%
    \fi
  \fi
  \bnu@listof{figure}%
}
\titlecontents{figure}
  [\z@]{}
  {\contentspush{\figurename~\thecontentslabel\quad}}{}
  {\nobreak\bnu@leaders\nobreak\hfil\thecontentspage}
%    \end{macrocode}
% \end{macro}
% \end{macro}
%
% \begin{macro}{\listoftables}
% \begin{macro}{\listoftables*}
% 附表清单。
%    \begin{macrocode}
\renewcommand\listoftables{%
  \ifbnu@degree@bachelor
    \ifbnu@backmatter\else
      \bnu@warning{The list of tables should be placed in back matter}%
    \fi
  \fi
  \bnu@listof{table}%
}
\titlecontents{table}
  [\z@]{}
  {\contentspush{\tablename~\thecontentslabel\quad}}{}
  {\bnu@leaders\thecontentspage}
%    \end{macrocode}
% \end{macro}
% \end{macro}
%
% \begin{macro}{\listoffiguresandtables}
% 将插图和附表合在一起列出“插图和附表清单”。
%    \begin{macrocode}
\newcommand\listoffiguresandtables{%
  \bnu@chapter*{\bnu@list@figure@table@name}%
  \@starttoc{lof}%
  \par
  \null\par
  \@starttoc{lot}%
}
%    \end{macrocode}
% \end{macro}
%
% \begin{macro}{\equcaption}
% 本命令只是为了生成公式列表,所以这个 caption 是假的。如果要编号最好用
% \env{equation} 环境,如果是其它编号环境,请手动添加 \cs{equcaption}。
% 用法如下:
%
% \cs{equcaption}\marg{counter}
%
% \marg{counter} 指定出现在索引中的编号,一般取 \cs{theequation},如果你是用
% \pkg{amsmath} 的 \cs{tag},那么默认是 \cs{tag} 的参数;除此之外可能需要你
% 手工指定。
%
%    \begin{macrocode}
\def\ext@equation{loe}
\def\equcaption#1{%
  \addcontentsline{\ext@equation}{equation}%
                  {\protect\numberline{#1}}}
%    \end{macrocode}
% \end{macro}
%
% \begin{macro}{\listofequations}
% \begin{macro}{\listofequations*}
% \LaTeX{} 默认没有公式索引,此处定义自己的 \cs{listofequations}。
% 公式索引没有名称,所以不设置固定的 label 宽度。
%    \begin{macrocode}
\newcommand\listofequations{\bnu@listof{equation}}
\titlecontents{equation}
  [0pt]{\addvspace{6bp}}
  {\bnu@equation@name~\thecontentslabel}{}
  {\nobreak\bnu@leaders\nobreak\thecontentspage}
\contentsuse{equation}{loe}
%    \end{macrocode}
% \end{macro}
% \end{macro}
%
%
% \subsection{参考文献}
% \label{sec:ref}
%
% 参考文献的格式根据用户选择的 \BibTeX{}/BibLaTeX 分别进行配置,
% 所以使用 \pkg{filehook} 的方式。
%
% 设置 \option{cite-style} 的接口,只对 \BibTeX{} 的编译方式有效。
%    \begin{macrocode}
\bnu@define@key{
  cite-style = {
    name = cite@style,
    choices = {
      super,
      inline,
      author-year,
    }
  }
}
%    \end{macrocode}
%
% \subsubsection{BibTeX + \pkg{natbib} 宏包}
%
%    \begin{macrocode}
\def\bibliographystyle#1{%
  \gdef\bu@bibstyle{#1}%
  \ifx\@begindocumenthook\@undefined\else
    \expandafter\AtBeginDocument
  \fi
    {\if@filesw
       \immediate\write\@auxout{\string\bibstyle{#1}}%
       \immediate\write\@auxout{\string\gdef\string\bu@bibstyle{#1}}%
     \fi}%
}
\def\bibliography#1{%
  \if@filesw
    \immediate\write\@auxout{\string\bibdata{\zap@space#1 \@empty}}%
    \immediate\write\@auxout{\string\gdef\string\bu@bibdata{#1}}%
  \fi
  \gdef\bu@bibdata{#1}%
  \@input@{\jobname.bbl}}
%    \end{macrocode}
%
% \BibTeX{} 和 \pkg{natbib} 宏包的配置。
%    \begin{macrocode}
\PassOptionsToPackage{compress}{natbib}
\AtEndOfPackageFile*{natbib}{
%    \end{macrocode}
% \begin{macro}{\inlinecite}
% 依赖于 \pkg{natbib} 宏包,修改其中的命令。 旧命令 \cs{onlinecite} 依然可用。
%    \begin{macrocode}
  \DeclareRobustCommand\inlinecite{\@inlinecite}
  \def\@inlinecite#1{\begingroup\let\@cite\NAT@citenum\citep{#1}\endgroup}
  \let\onlinecite\inlinecite
%    \end{macrocode}
% \end{macro}
%
% 几种种引用样式,与 \file{bst} 文件名保持一致,
% 这样在使用 \cs{bibliographystyle} 选择参考文献表的样式时也会设置对应的引用样式。
%    \begin{macrocode}
  \newcommand\bibstyle@super{%
    \bibpunct{[}{]}{,}{s}{,}{\textsuperscript{,}}}
  \newcommand\bibstyle@inline{%
    \bibpunct{[}{]}{,}{n}{,}{,}}
  \@namedef{bibstyle@author-year}{%
    \bibpunct{(}{)}{;}{a}{,}{,}}
%    \end{macrocode}
%
%    \begin{macrocode}
  \bnu@option@hook{cite-style}{\@nameuse{bibstyle@\bnu@cite@style}}
%    \end{macrocode}
%
% 几种种引用样式,与 \file{bst} 文件名保持一致,
% 这样在使用 \cs{bibliographystyle} 选择参考文献表的样式时也会设置对应的引用样式。
%    \begin{macrocode}
  \@namedef{bibstyle@bnuthesis-numeric}{\citestyle{super}}
  \@namedef{bibstyle@bnuthesis-author-year}{\citestyle{author-year}}
  \@namedef{bibstyle@bnuthesis-bachelor}{\citestyle{super}}
%    \end{macrocode}
%
% 修改引用的样式。
% 这里在 filehook 中无法使用 \cs{patchcmd},所以只能手动重定义。
%
% 将 \cs{citep} super 式引用的页码改为上标。
%    \begin{macrocode}
  \renewcommand\NAT@citesuper[3]{%
    \ifNAT@swa
      \if*#2*\else
        #2\NAT@spacechar
      \fi
      % \unskip\kern\p@\textsuperscript{\NAT@@open#1\NAT@@close}%
      %  \if*#3*\else\NAT@spacechar#3\fi\else #1\fi\endgroup}
      \unskip\kern\p@
      \textsuperscript{%
        \NAT@@open#1\NAT@@close
        \if*#3*\else#3\fi
      }%
      \kern\p@
    \else
      #1%
    \fi
    \endgroup
  }
%    \end{macrocode}
%
% 将 \cs{citep} numbers 式引用的页码改为上标并置于括号外。
%    \begin{macrocode}
  \renewcommand\NAT@citenum[3]{%
    \ifNAT@swa
      \NAT@@open
      \if*#2*\else
        #2\NAT@spacechar
      \fi
      % #1\if*#3*\else\NAT@cmt#3\fi\NAT@@close
      #1\NAT@@close
      \if*#3*\else
        \textsuperscript{#3}%
      \fi
    \else
      #1%
    \fi
    \endgroup
  }
%    \end{macrocode}
%
% 修改 \cs{citet} 引用的样式。
%    \begin{macrocode}
  \def\NAT@citexnum[#1][#2]#3{%
    \NAT@reset@parser
    \NAT@sort@cites{#3}%
    \NAT@reset@citea
    \@cite{\def\NAT@num{-1}\let\NAT@last@yr\relax\let\NAT@nm\@empty
      \@for\@citeb:=\NAT@cite@list\do
      {\@safe@activestrue
      \edef\@citeb{\expandafter\@firstofone\@citeb\@empty}%
      \@safe@activesfalse
      \@ifundefined{b@\@citeb\@extra@b@citeb}{%
        {\reset@font\bfseries?}
          \NAT@citeundefined\PackageWarning{natbib}%
        {Citation `\@citeb' on page \thepage \space undefined}}%
      {\let\NAT@last@num\NAT@num\let\NAT@last@nm\NAT@nm
        \NAT@parse{\@citeb}%
        \ifNAT@longnames\@ifundefined{bv@\@citeb\@extra@b@citeb}{%
          \let\NAT@name=\NAT@all@names
          \global\@namedef{bv@\@citeb\@extra@b@citeb}{}}{}%
        \fi
        \ifNAT@full\let\NAT@nm\NAT@all@names\else
          \let\NAT@nm\NAT@name\fi
        \ifNAT@swa
        \@ifnum{\NAT@ctype>\@ne}{%
          \@citea
          \NAT@hyper@{\@ifnum{\NAT@ctype=\tw@}{\NAT@test{\NAT@ctype}}{\NAT@alias}}%
        }{%
          \@ifnum{\NAT@cmprs>\z@}{%
          \NAT@ifcat@num\NAT@num
            {\let\NAT@nm=\NAT@num}%
            {\def\NAT@nm{-2}}%
          \NAT@ifcat@num\NAT@last@num
            {\@tempcnta=\NAT@last@num\relax}%
            {\@tempcnta\m@ne}%
          \@ifnum{\NAT@nm=\@tempcnta}{%
            \@ifnum{\NAT@merge>\@ne}{}{\NAT@last@yr@mbox}%
          }{%
            \advance\@tempcnta by\@ne
            \@ifnum{\NAT@nm=\@tempcnta}{%
%    \end{macrocode}
%
% 在顺序编码制下,\pkg{natbib} 只有在三个以上连续文献引用才会使用连接号,
% 这里修改为允许两个引用使用连接号。
%    \begin{macrocode}
              % \ifx\NAT@last@yr\relax
              %   \def@NAT@last@yr{\@citea}%
              % \else
              %   \def@NAT@last@yr{--\NAT@penalty}%
              % \fi
              \def@NAT@last@yr{-\NAT@penalty}%
            }{%
              \NAT@last@yr@mbox
            }%
          }%
          }{%
          \@tempswatrue
          \@ifnum{\NAT@merge>\@ne}{\@ifnum{\NAT@last@num=\NAT@num\relax}{\@tempswafalse}{}}{}%
          \if@tempswa\NAT@citea@mbox\fi
          }%
        }%
        \NAT@def@citea
        \else
          \ifcase\NAT@ctype
            \ifx\NAT@last@nm\NAT@nm \NAT@yrsep\NAT@penalty\NAT@space\else
              \@citea \NAT@test{\@ne}\NAT@spacechar\NAT@mbox{\NAT@super@kern\NAT@@open}%
            \fi
            \if*#1*\else#1\NAT@spacechar\fi
            \NAT@mbox{\NAT@hyper@{{\citenumfont{\NAT@num}}}}%
            \NAT@def@citea@box
          \or
            \NAT@hyper@citea@space{\NAT@test{\NAT@ctype}}%
          \or
            \NAT@hyper@citea@space{\NAT@test{\NAT@ctype}}%
          \or
            \NAT@hyper@citea@space\NAT@alias
          \fi
        \fi
      }%
      }%
        \@ifnum{\NAT@cmprs>\z@}{\NAT@last@yr}{}%
        \ifNAT@swa\else
%    \end{macrocode}
%
% 将页码放在括号外边,并且置于上标。
%    \begin{macrocode}
          % \@ifnum{\NAT@ctype=\z@}{%
          %   \if*#2*\else\NAT@cmt#2\fi
          % }{}%
          \NAT@mbox{\NAT@@close}%
          \@ifnum{\NAT@ctype=\z@}{%
            \if*#2*\else
              \textsuperscript{#2}%
            \fi
          }{}%
          \NAT@super@kern
        \fi
    }{#1}{#2}%
  }%
%    \end{macrocode}
%
% 修改 \cs{citep} author-year 式的页码:
%    \begin{macrocode}
  \renewcommand\NAT@cite%
      [3]{\ifNAT@swa\NAT@@open\if*#2*\else#2\NAT@spacechar\fi
          % #1\if*#3*\else\NAT@cmt#3\fi\NAT@@close\else#1\fi\endgroup}
          #1\NAT@@close\if*#3*\else\textsuperscript{#3}\fi\else#1\fi\endgroup}
%    \end{macrocode}
%
% 修改 \cs{citet} author-year 式的页码:
%    \begin{macrocode}
  \def\NAT@citex%
    [#1][#2]#3{%
    \NAT@reset@parser
    \NAT@sort@cites{#3}%
    \NAT@reset@citea
    \@cite{\let\NAT@nm\@empty\let\NAT@year\@empty
      \@for\@citeb:=\NAT@cite@list\do
      {\@safe@activestrue
      \edef\@citeb{\expandafter\@firstofone\@citeb\@empty}%
      \@safe@activesfalse
      \@ifundefined{b@\@citeb\@extra@b@citeb}{\@citea%
        {\reset@font\bfseries ?}\NAT@citeundefined
                  \PackageWarning{natbib}%
        {Citation `\@citeb' on page \thepage \space undefined}\def\NAT@date{}}%
      {\let\NAT@last@nm=\NAT@nm\let\NAT@last@yr=\NAT@year
        \NAT@parse{\@citeb}%
        \ifNAT@longnames\@ifundefined{bv@\@citeb\@extra@b@citeb}{%
          \let\NAT@name=\NAT@all@names
          \global\@namedef{bv@\@citeb\@extra@b@citeb}{}}{}%
        \fi
      \ifNAT@full\let\NAT@nm\NAT@all@names\else
        \let\NAT@nm\NAT@name\fi
      \ifNAT@swa\ifcase\NAT@ctype
        \if\relax\NAT@date\relax
          \@citea\NAT@hyper@{\NAT@nmfmt{\NAT@nm}\NAT@date}%
        \else
          \ifx\NAT@last@nm\NAT@nm\NAT@yrsep
              \ifx\NAT@last@yr\NAT@year
                \def\NAT@temp{{?}}%
                \ifx\NAT@temp\NAT@exlab\PackageWarningNoLine{natbib}%
                {Multiple citation on page \thepage: same authors and
                year\MessageBreak without distinguishing extra
                letter,\MessageBreak appears as question mark}\fi
                \NAT@hyper@{\NAT@exlab}%
              \else\unskip\NAT@spacechar
                \NAT@hyper@{\NAT@date}%
              \fi
          \else
            \@citea\NAT@hyper@{%
              \NAT@nmfmt{\NAT@nm}%
              \hyper@natlinkbreak{%
                \NAT@aysep\NAT@spacechar}{\@citeb\@extra@b@citeb
              }%
              \NAT@date
            }%
          \fi
        \fi
      \or\@citea\NAT@hyper@{\NAT@nmfmt{\NAT@nm}}%
      \or\@citea\NAT@hyper@{\NAT@date}%
      \or\@citea\NAT@hyper@{\NAT@alias}%
      \fi \NAT@def@citea
      \else
        \ifcase\NAT@ctype
          \if\relax\NAT@date\relax
            \@citea\NAT@hyper@{\NAT@nmfmt{\NAT@nm}}%
          \else
          \ifx\NAT@last@nm\NAT@nm\NAT@yrsep
              \ifx\NAT@last@yr\NAT@year
                \def\NAT@temp{{?}}%
                \ifx\NAT@temp\NAT@exlab\PackageWarningNoLine{natbib}%
                {Multiple citation on page \thepage: same authors and
                year\MessageBreak without distinguishing extra
                letter,\MessageBreak appears as question mark}\fi
                \NAT@hyper@{\NAT@exlab}%
              \else
                \unskip\NAT@spacechar
                \NAT@hyper@{\NAT@date}%
              \fi
          \else
            \@citea\NAT@hyper@{%
              \NAT@nmfmt{\NAT@nm}%
              \hyper@natlinkbreak{\NAT@spacechar\NAT@@open\if*#1*\else#1\NAT@spacechar\fi}%
                {\@citeb\@extra@b@citeb}%
              \NAT@date
            }%
          \fi
          \fi
        \or\@citea\NAT@hyper@{\NAT@nmfmt{\NAT@nm}}%
        \or\@citea\NAT@hyper@{\NAT@date}%
        \or\@citea\NAT@hyper@{\NAT@alias}%
        \fi
        \if\relax\NAT@date\relax
          \NAT@def@citea
        \else
          \NAT@def@citea@close
        \fi
      \fi
      }}\ifNAT@swa\else
%    \end{macrocode}
%
% 将页码放在括号外边,并且置于上标。
%    \begin{macrocode}
        % \if*#2*\else\NAT@cmt#2\fi
        \if\relax\NAT@date\relax\else\NAT@@close\fi
        \if*#2*\else\textsuperscript{#2}\fi
      \fi}{#1}{#2}}
%    \end{macrocode}
%
% 参考文献表的正文部分用五号字。
% 行距采用固定值 16 磅,段前空 3 磅,段后空 0 磅。
%
% 本科生要求宋体五号/Times New Roman 10.5 pt,固定行距 17pt,段前后间距 3pt;
% 英文专业要求悬挂缩进 0.5inch(1.27 厘米)。
%
% 复用 \pkg{natbib} 的 \texttt{thebibliography} 环境,调整距离。
%    \begin{macrocode}
  \renewcommand\bibsection{\bnu@chapter*{\bibname}}
  \newcommand\bnu@set@bibliography@format{%
    \ifbnu@degree@bachelor
      \renewcommand\bibfont{\fontsize{10.5bp}{17bp}\selectfont}%
      \setlength{\bibsep}{6bp \@plus 3bp \@minus 3bp}%
      \ifbnu@main@language@chinese
        \setlength{\bibhang}{21bp}%
      \else
        \setlength{\bibhang}{0.5in}%
      \fi
    \else
      \renewcommand\bibfont{\fontsize{10.5bp}{16bp}\selectfont}%
      \setlength{\bibsep}{3bp \@plus 3bp \@minus 3bp}%
      \setlength{\bibhang}{21bp}%
    \fi
  }
  \bnu@set@bibliography@format
  \bnu@option@hook{degree}{\bnu@set@bibliography@format}
  \bnu@option@hook{main-language}{\bnu@set@bibliography@format}
%    \end{macrocode}
%
% 研究生要求每一条文献的内容要尽量写在同一页内。
% 遇有被迫分页的情况,可通过“留白”或微调本页行距的方式尽量将同一条文献内容放在一页。
% 所以上述 \cs{bibsep} 的设置允许 1pt 的伸缩,
% 同时增加同一条文献内分页的惩罚,
% 这里参考 \href{https://gibnub.com/plk/biblatex/blob/e5d6e69e61613cc33ab1fcc2083a8277eb9cfce5/tex/latex/biblatex/biblatex.def}{BibLaTeX 的设置}。
%    \begin{macrocode}
  \patchcmd\thebibliography{%
    \clubpenalty4000%
  }{%
    \interlinepenalty=5000\relax
    \clubpenalty=10000\relax
  }{}{\bnu@patch@error{\thebibliography}}
  \patchcmd\thebibliography{%
    \widowpenalty4000%
  }{%
    \widowpenalty=10000\relax
  }{}{\bnu@patch@error{\thebibliography}}
%    \end{macrocode}
%
% 参考文献表的编号居左,宽度 1 cm。
%    \begin{macrocode}
  \def\@biblabel#1{[#1]\hfill}
  \renewcommand\NAT@bibsetnum[1]{%
    % \settowidth\labelwidth{\@biblabel{#1}}%
    % \setlength{\leftmargin}{\labelwidth}%
    % \addtolength{\leftmargin}{\labelsep}%
    \setlength{\leftmargin}{1cm}%
    \setlength{\itemindent}{\z@}%
    \setlength{\labelsep}{0.1cm}%
    \setlength{\labelwidth}{0.9cm}%
    \setlength{\itemsep}{\bibsep}
    \setlength{\parsep}{\z@}%
    \ifNAT@openbib
      \addtolength{\leftmargin}{\bibindent}%
      \setlength{\itemindent}{-\bibindent}%
      \setlength{\listparindent}{\itemindent}%
      \setlength{\parsep}{0pt}%
    \fi
  }
}
%    \end{macrocode}
%
% \subsubsection{\pkg{biblatex} 宏包}
%
%    \begin{macrocode}
\AtEndOfPackageFile*{biblatex}{
  \AtBeginDocument{
    \ifthenelse{\equal{\blx@bbxfile}{apa}}{\def\bibname{REFERENCES}}{}
    \ifthenelse{\equal{\blx@bbxfile}{apa6}}{\def\bibname{REFERENCES}}{}
    \ifthenelse{\equal{\blx@bbxfile}{mla}}{\def\bibname{WORKS CITED}}{}
    \ifthenelse{\equal{\blx@bbxfile}{mla-new}}{\def\bibname{WORKS CITED}}{}
  }
  \DeclareRobustCommand\inlinecite{\parencite}
  \defbibheading{bibliography}[\bibname]{\bnu@chapter*{\bibname}}
  \newcommand\bnu@set@bibliography@format{%
    \ifbnu@degree@bachelor
      \renewcommand\bibfont{\fontsize{10.5bp}{17bp}\selectfont}%
      \setlength{\bibitemsep}{6bp \@plus 3bp \@minus 3bp}%
      \ifbnu@main@language@chinese
        \setlength{\bibhang}{21bp}%
      \else
        \setlength{\bibhang}{0.5in}%
      \fi
    \else
      \renewcommand\bibfont{\fontsize{10.5bp}{16bp}\selectfont}%
      \setlength{\bibitemsep}{3bp \@plus 3bp \@minus 3bp}%
      \setlength{\biblabelsep}{0.1cm}%
      \setlength{\bibhang}{21bp}%
    \fi
  }
  \bnu@set@bibliography@format
  \bnu@option@hook{degree}{\bnu@set@bibliography@format}
  \bnu@option@hook{main-language}{\bnu@set@bibliography@format}
}
%    \end{macrocode}
%
% \subsubsection{\pkg{apacite} 宏包}
%
% \pkg{apacite} 在 \cs{begindocument} 处载入的 \file{english.apc}
% 会覆盖掉 \cs{bibname} 的定义,所以需要重新 \cs{bnu@set@chapter@names}。
%    \begin{macrocode}
\AtEndOfPackageFile*{apacite}{
  \AtBeginDocument{
    \bnu@set@chapter@names
  }
  \renewcommand\bibliographytypesize{\fontsize{10.5bp}{17bp}\selectfont}
  \setlength{\bibitemsep}{6bp}
  \ifbnu@main@language@chinese
    \setlength{\bibleftmargin}{21bp}
    \setlength{\bibindent}{-\bibleftmargin}
  \else
    \setlength{\bibleftmargin}{0.5in}
    \setlength{\bibindent}{-\bibleftmargin}
  \fi
  \def\st@rtbibchapter{%
    \if@numberedbib%
      \chapter{\bibname}%   e.g.,   6. References
    \else%
      \bnu@chapter*{\bibname}%   e.g.,   References
    \fi%
  }%
}
%    \end{macrocode}
%
% \subsection{附录}
% \label{sec:appendix}
%
%    \begin{macrocode}
\g@addto@macro\appendix{%
  \@mainmattertrue
  \ifbnu@degree@bachelor
    \ifbnu@statement@exists\else
      \bnu@warning{The appendices should be placed after statement}%
    \fi
  \fi
}
%    \end{macrocode}
%
% 研究生和本科生的写作指南均未规定附录的节标题是否加入目录,
% 但是从示例来看,目录中只出现附录的 chapter 标题,
% 不出现附录中的 section 及 subsection 的标题。
% 部分院系(例如自动化系)的格式审查的老师甚至一致口头如此要求。
% (\href{https://gibnub.com/tuna/thuthesis/pull/425}{\#425})
%    \begin{macrocode}
\bnu@define@key{
  toc-depth = {
    name = toc@depth,
  },
}
%    \end{macrocode}
%
% 这里不要使用 \cs{addcontentsline},
% 避免写入 \pkg{titletoc} 的 \file{.ptc} 文件中,
% 造成 \env{survey} 的子目录中 |tocdepth| 为 0。
%    \begin{macrocode}
\bnu@option@hook{toc-depth}{%
  \ifx\@begindocumenthook\@undefined
    \protected@write\@auxout{}{%
      \string\ttl@writefile{toc}{%
        \protect\setcounter{tocdepth}{\bnu@toc@depth}%
      }%
    }%
  \else
    \setcounter{tocdepth}{\bnu@toc@depth}%
  \fi
}
\g@addto@macro\appendix{%
  \bnusetup{
    toc-depth = 0,
  }%
}
%    \end{macrocode}
%
% 附录中的图、表不列入插图清单/附表清单。
%    \begin{macrocode}
\bnu@define@key{
  appendix-figure-in-lof = {
    name = appendix@figure@in@lof,
    choices = {
      true,
      false,
    },
    default = false,
  },
}
\bnu@option@hook{appendix-figure-in-lof}{%
  \ifbnu@appendix@figure@in@lof@true
    \addtocontents{lof}{\string\let\string\contentsline\string\ttl@contentsline}%
    \addtocontents{lot}{\string\let\string\contentsline\string\ttl@contentsline}%
    \addtocontents{loe}{\string\let\string\contentsline\string\ttl@contentsline}%
  \else
    \addtocontents{lof}{\string\let\string\contentsline\string\ttl@gobblecontents}%
    \addtocontents{lot}{\string\let\string\contentsline\string\ttl@gobblecontents}%
    \addtocontents{loe}{\string\let\string\contentsline\string\ttl@gobblecontents}%
  \fi
}
\g@addto@macro\appendix{%
  \bnusetup{
    appendix-figure-in-lof = false,
  }%
}
%    \end{macrocode}
%
% 附录中的参考文献等另行编序号。
%    \begin{macrocode}
\newcommand\bnu@end@appendix@ref@section{}
%    \end{macrocode}
%
%    \begin{macrocode}
%    \end{macrocode}
%
% \pkg{bibunits} 在载入时会保存 \cs{bibliography} 和 \cs{bibliographystyle},
% 所以在载入宏包前修改定义。
%    \begin{macrocode}
\AtBeginOfPackageFile*{bibunits}{
  \def\bibliography#1{%
    \if@filesw
      \immediate\write\@auxout{\string\bibdata{\zap@space#1 \@empty}}%
%    \end{macrocode}
%
% 正文的 \cs{bibliography} 同时设置附录参考文献的默认 \file{.bib} 数据库。
%    \begin{macrocode}
      \immediate\write\@auxout{\string\gdef\string\bu@bibdata{#1}}%
    \fi
    \@input@{\jobname.bbl}%
    \gdef\bu@bibdata{#1}%
  }
  \def\bibliographystyle#1{%
    \ifx\@begindocumenthook\@undefined\else
      \expandafter\AtBeginDocument
    \fi
      {\if@filesw
        \immediate\write\@auxout{\string\bibstyle{#1}}%
%    \end{macrocode}
%
% 正文的 \cs{bibliographystyle} 同时设置附录参考文献的默认 \file{.bst} 样式。
%    \begin{macrocode}
        \immediate\write\@auxout{\string\gdef\string\bu@bibstyle{#1}}%
      \fi}%
      \gdef\bu@bibstyle{#1}%
  }
}
\AtEndOfPackageFile*{bibunits}{
  \def\@startbibunit{%
    \global\let\@startbibunitorrelax\relax
    \global\let\@finishbibunit\@finishstartedbibunit
    \global\advance\@bibunitauxcnt 1
    \if@filesw
      {\endlinechar-1
%    \end{macrocode}
%
% 使附录 aux 文件的 \cs{gdef}\cs{@localbibstyle} 能够生效。
%    \begin{macrocode}
      \makeatletter
      \@input{\@bibunitname.aux}}%
      \immediate\openout\@bibunitaux\@bibunitname.aux
      \immediate\write\@bibunitaux{\string\bibstyle{\@localbibstyle}}%
    \fi
  }
  \def\bu@bibliography#1{%
    \putbib[#1]%
  }
  \def\bu@bibliographystyle#1{%
    \if@filesw
      \immediate\write\@bibunitaux{\string\gdef\string\@localbibstyle{#1}}%
    \fi
    \gdef\@localbibstyle{#1}%
  }
  \providecommand\printbibliography{\putbib\relax}%
  \g@addto@macro\appendix{%
    \renewcommand\@bibunitname{\jobname-appendix-\@alph\c@chapter}%
    \bibliographyunit[\chapter]%
    \renewcommand\bibsection{%
      \ctexset{section/numbering = false}%
      \section{\bibname}%
      \ctexset{section/numbering = true}%
    }%
%    \end{macrocode}
%
% 研究生附录的引用编号加前缀,如附录 A 的引用 [1] 为 [A.1]。
%    \begin{macrocode}
    \ifbnu@degree@graduate
      \renewcommand\citenumfont{\@Alph\c@chapter.}%
      \renewcommand\@extra@binfo{@-\@alph\c@chapter}%
      \renewcommand\@extra@b@citeb{@-\@alph\c@chapter}%
      \renewcommand\bibnumfmt[1]{[\@Alph\c@chapter.#1]\hfill}%
    \fi
  }
  \renewcommand\bnu@end@appendix@ref@section{%
    \bibliographyunit\relax
  }
  \AtEndDocument{\bnu@end@appendix@ref@section}
  \renewcommand\bnu@set@survey@bibheading{%
    \renewcommand\bibsection{%
      \par
      \vskip 20bp%
      \bnu@pdfbookmark{1}{\bibname}%
      \begingroup
        \centering
        \xiaosi[1.667]\bibname\par
      \endgroup
      \vskip 6bp%
    }%
  }%
%    \end{macrocode}
%
% 如果正文和附录引用了同一文献,\pkg{bibunits} 会给出无意义的警告,这里消除警告。
%    \begin{macrocode}
  % \let\@xtestdef\@gobbletwo  % This doesn't work
  \def\bibunits@rerun@warning{\relax}
}
\PassOptionsToPackage{defernumbers = true}{biblatex}
\AtEndOfPackageFile*{biblatex}{
  \DeclareRefcontext{appendix}{}
  \g@addto@macro\appendix{%
    \pretocmd\chapter{%
      \newrefsection
      \ifbnu@degree@bachelor\else
        \@tempcnta=\c@chapter
        \advance\@tempcnta\@ne
        \newrefcontext[labelprefix = {\@Alph\@tempcnta.}]{appendix}%
      \fi
    }{}{\bnu@patch@error{\chapter}}%
    \defbibheading{bibliography}[\bibname]{%
      \ctexset{section/numbering = false}%
      \section{#1}%
      \ctexset{section/numbering = true}%
    }%
  }
  \renewcommand\bnu@set@survey@bibheading{%
    \defbibheading{bibliography}[\bibname]{%
      \par
      \vskip 20bp%
      \bnu@pdfbookmark{1}{\bibname}%
      \begingroup
        \centering
        \xiaosi[1.667]##1\par
      \endgroup
      \vskip 6bp%
    }%
  }%
  \def\bibliographystyle#1{%
    \bnu@warning{'bibliographystyle' invalid for 'biblatex'.}%
  }
}
%    \end{macrocode}
%
% 本科生《写作指南》有独特的要求:附录 A 为外文资料的调研阅读报告或书面翻译,
% 并且要分别附上独立的参考文献和外文资料的原文索引。
% 所以这里定义 \env{survey} 和 \env{translation} 专门处理这两种情况,
% 其中参考文献使用了 \pkg{bibunits} 宏包的功能。
%
% 注意 \pkg{titletoc} 在 2019/07/14 v2.11.1702 修改了 \cs{printcontents} 接口,
% 而且 \cs{@ifpackagelater} 只能用在导言区中,所以需要定义辅助宏。
%    \begin{macrocode}
\@ifpackagelater{titletoc}{2019/07/14}{
  \newcommand\bnu@print@contents[5]{%
    \printcontents[#1]{#2}{#3}[#4]{}%
  }
}{
  \newcommand\bnu@print@contents[5]{%
    \printcontents[#1]{#2}{#3}{\setcounter{tocdepth}{#4}#5}%
  }
}
%    \end{macrocode}
%
% \begin{environment}{survey}
% 外文资料的调研阅读报告。
%    \begin{macrocode}
\newenvironment{survey}{%
  \chapter{外文资料的调研阅读报告}%
  \bnusetup{language = english}%
  \let\title\bnu@appendix@title
  \let\maketitle\bnu@appendix@maketitle
  \bnu@set@partial@toc@format
  \renewcommand\tableofcontents{%
    \section*{Contents}%
    \bnu@pdfbookmark{1}{Contents}%
    \bnu@print@contents{survey}{l}{1}{2}{}%
    \vskip 20bp%
  }%
  \let\appendix\bnu@appendix@appendix
  \bnu@set@survey@bibheading
  \renewcommand\bibname{参考文献}%
  \startcontents[survey]%
}{%
  \stopcontents[survey]%
  \bnu@reset@main@language % restore language
}
\newcommand\bnu@set@appendix@bib@heading{}
%    \end{macrocode}
% \end{environment}
%
% \begin{environment}{translation}
% 外文资料的书面翻译。
%    \begin{macrocode}
\newenvironment{translation}{%
  \chapter{外文资料的书面翻译}%
  \bnusetup{language = chinese}%
  \let\title\bnu@appendix@title
  \let\maketitle\bnu@appendix@maketitle
  \renewenvironment{abstract}{%
    \ctexset{
      section = {
        format    += \centering,
        numbering  = false,
      },
    }%
    \section{摘要}%
  }{%
    \par
    \ifx\bnu@keywords\@empty\else
      \textbf{关键词:}\bnu@clist@use{\bnu@keywords}{;}\par
    \fi
  }%
  \bnu@set@partial@toc@format
  \renewcommand\tableofcontents{%
    \section*{目录}%
    \bnu@pdfbookmark{1}{目录}%
    \bnu@print@contents{translation}{l}{1}{2}{}%
    \vskip 20bp%
  }%
  \let\appendix\bnu@appendix@appendix
  \def\bibsection{%
    \begingroup
      \ctexset{section/numbering=false}%
      \section{\bibname}%
    \endgroup
  }%
  \startcontents[translation]%
}{%
  \stopcontents[translation]%
  \bnu@reset@main@language % restore language
}
%    \end{macrocode}
% \end{environment}
%
% \begin{environment}{translation}
% 书面翻译对应的原文索引,区别于译文的参考文献。
%    \begin{macrocode}
\newenvironment{translation-index}{}{}
\AtEndOfPackageFile*{bibunits}{
  \renewenvironment{translation-index}{%
    \begin{bibunit}%
      \renewcommand\@bibunitname{\jobname-index}%
      \renewcommand\bibname{书面翻译对应的原文索引}%
      \bnu@set@survey@bibheading
  }{%
    \end{bibunit}%
  }
}
\AtEndOfPackageFile*{biblatex}{
  \renewenvironment{translation-index}{%
    \endrefsection
    \begin{refsection}%
      \renewcommand\bibname{书面翻译对应的原文索引}%
      \bnu@set@survey@bibheading
  }{%
    \end{refsection}%
  }
}
%    \end{macrocode}
% \end{environment}
%
% 调研阅读报告需要独立的标题,这里仿照了标准文档类的用法 \cs{title}, \cs{maketitle}。
%    \begin{macrocode}
\DeclareRobustCommand\bnu@appendix@title[1]{\gdef\bnu@appendix@@title{#1}}
\newcommand\bnu@appendix@maketitle{%
  \par
  \begin{center}%
    \xiaosi[1.667]\bnu@appendix@@title
  \end{center}%
  \par
}
%    \end{macrocode}
%
%    \begin{macrocode}
\newcommand\bnu@set@partial@toc@format{%
  \titlecontents{section}
    [\z@]{}
    {\contentspush{\thecontentslabel\quad}}{}
    {\bnu@leaders\thecontentspage}%
  \titlecontents{subsection}
    [1em]{}
    {\contentspush{\thecontentslabel\quad}}{}
    {\bnu@leaders\thecontentspage}%
  \titlecontents{subsubsection}
    [2em]{}
    {\contentspush{\thecontentslabel\quad}}{}
    {\bnu@leaders\thecontentspage}%
}
%    \end{macrocode}
%
% 书面翻译的附录。
%    \begin{macrocode}
\newcommand\bnu@appendix@appendix{%
  \def\theHsection{\Hy@AlphNoErr{section}}%
  \setcounter{section}{0}%
  \setcounter{subsection}{0}%
  \renewcommand\thesection{\thechapter.\@Alph\c@section}%
}%
%    \end{macrocode}
%
% 调研阅读报告的参考文献(或书面翻译对应的外文资料的原文索引)标题用宋体小四号字,段前 20pt,段后 6pt,行距 20pt。
%    \begin{macrocode}
\newcommand\bnu@set@survey@bibheading{}
%    \end{macrocode}
%
%
% \subsection{个人简历}
%
% \begin{environment}{resume}
% 个人简历发表文章等。
%    \begin{macrocode}
\newenvironment{resume}{%
  \@mainmatterfalse
  \bnu@end@appendix@ref@section
  \bnu@chapter*{\bnu@resume@name}%
  \ctexset{
    section = {
      format    += \centering,
      numbering = false,
    },
  }%
  \ifbnu@language@chinese
    \ctexset{
      subsection = {
        format     = \sffamily\fontsize{14bp}{20bp}\selectfont,
        numbering  = false,
        aftertitle = :,
      },
    }%
    \setlist[achievements]{
      topsep     = 6bp,
      itemsep    = 6bp,
      leftmargin = 1cm,
      labelwidth = 1cm,
      labelsep   = 0pt,
      first      = {
        \ifbnu@degree@graduate
          \fontsize{12bp}{16bp}\selectfont
        \fi
      },
      align      = left,
      label      = [\arabic*],
      resume     = achievements,
    }%
  \else
    \ctexset{
      subsection = {
        beforeskip = 0pt,
        afterskip  = 0pt,
        format     = \bfseries\normalsize,
        indent     = \parindent,
        numbering  = false,
      },
    }%
    \ifbnu@degree@bachelor
      % 内容部分用Arial字体,字号15pt,行距采用固定值20pt, 段前后 0pt。
      \sffamily\fontsize{15bp}{20bp}\selectfont
    \fi
    \setlist[achievements]{
      topsep     = 0bp,
      itemsep    = 0bp,
      leftmargin = 1.75cm,
      labelsep   = 0.5cm,
      align      = right,
      label      = [\arabic*],
      resume     = achievements,
    }%
  \fi
}{}
%    \end{macrocode}
% \end{environment}
%
% 旧的 \cs{resumeitem} 和 \cs{researchitem} 已经过时。
%    \begin{macrocode}
\newcommand\resumeitem[1]{%
  \bnu@error{The "\protect\resumeitem" is obsolete. Please update to the new format}%
}
\newcommand\researchitem[1]{%
  \bnu@error{The "\protect\researchitem" is obsolete. Please update to the new format}%
}
%    \end{macrocode}
%
% \begin{environment}{achievements}
% 学术成果由 \env{achievements} 环境罗列。
%    \begin{macrocode}
\newlist{achievements}{enumerate}{1}
\setlist[achievements]{
  topsep     = 6bp,
  partopsep  = 0bp,
  itemsep    = 6bp,
  parsep     = 0bp,
  leftmargin = 10mm,
  itemindent = 0pt,
  align      = left,
  label      = [\arabic*],
  resume     = achievements,
}
%    \end{macrocode}
% \end{environment}
%
%    \begin{macrocode}
\newenvironment{publications}{%
  \bnu@deprecate{"publications" environment}{"achievements"}%
  \begin{achievements}%
}{%
  \end{achievements}%
}
\newcommand\publicationskip{%
  \bnu@error{The "\protect\publicationskip" is obsolete. Do not use it}%
}
%    \end{macrocode}
%
% \subsection{指导教师/小组学术评语}
% \begin{environment}{comments}
%    \begin{macrocode}
\NewEnviron{comments}[1][]{%
  \bnu@end@appendix@ref@section
  \ifbnu@degree@graduate
    \@mainmatterfalse
    \kv@define@key{bnu@comments}{name}{\let\bnu@comments@name\kv@value}%
    \kv@set@family@handler{bnu@comments}{%
      \ifx\kv@value\relax
        \let\bnu@comments@name\kv@key
      \else
        \kv@handled@false
      \fi
    }%
    \kvsetkeys{bnu@comments}{#1}%
    \chapter{\bnu@comments@name}%
    \BODY\clearpage
  \fi
}
%    \end{macrocode}
% \end{environment}
%
% \subsection{答辩委员会决议书}
% \begin{environment}{resolution}
%    \begin{macrocode}
\NewEnviron{resolution}{%
  \bnu@end@appendix@ref@section
  \ifbnu@degree@graduate
    \@mainmatterfalse
    \chapter{\bnu@resolution@name}%
    \BODY\clearpage
  \fi
}
%    \end{macrocode}
% \end{environment}
%
% \subsection{综合论文训练记录表}
%
% \begin{macro}{\record}
% (本科生专用)插入综合论文训练记录表的 PDF 版本,并加入书签。
%
%    \begin{macrocode}
\newcommand{\record}[1]{%
  \let\bnu@record@file\@empty
  \kv@define@key{bnu@record}{file}{\let\bnu@record@file\kv@value}%
  \kv@set@family@handler{bnu@record}{%
    \ifx\kv@value\relax
      \let\bnu@record@file\kv@key
    \else
      \kv@handled@false
    \fi
  }%
  \kvsetkeys{bnu@record}{#1}%
  \ifx\bnu@record@file\@empty
    \bnu@error{File path of \protect\record\space is required}
  \fi
  \cleardoublepage
  \bnu@pdfbookmark{0}{综合论文训练记录表}%
  \includepdf[pages=-]{\bnu@record@file}%
}
%    \end{macrocode}
%
% \end{macro}
%
% \subsection{其他宏包的设置}
%
% 这些宏包并非格式要求,但是为了方便同学们使用,在这里进行简单设置。
%
% \subsubsection{\pkg{hyperref} 宏包}
%
% 使用 \cs{PassOptionsToPackage} 的方式进行配置,允许用户在 \cs{usepackage}
% 覆盖配置(\href{https://github.com/tuna/thuthesis/issues/863}{tuna/thuthesis\#863})。
%
%    \begin{macrocode}
\PassOptionsToPackage{
  linktoc            = all,
  bookmarksdepth     = 2,
  bookmarksnumbered  = true,
  bookmarksopen      = true,
  bookmarksopenlevel = 1,
  bookmarksdepth     = 3,
  unicode            = true,
  psdextra           = true,
  breaklinks         = true,
  plainpages         = false,
  pdfdisplaydoctitle = true,
  hidelinks,
}{hyperref}
\AtEndOfPackageFile*{hyperref}{
  \newcounter{bnu@bookmark}
  \renewcommand\bnu@pdfbookmark[2]{%
    \phantomsection
    \stepcounter{bnu@bookmark}%
    \pdfbookmark[#1]{#2}{bnuchapter.\thebnu@bookmark}%
  }
  \renewcommand\bnu@phantomsection{%
    \phantomsection
  }
  \pdfstringdefDisableCommands{
    \let\\\relax
    \let\quad\relax
    \let\qquad\relax
    \let\hspace\@gobble
  }%
%    \end{macrocode}
%
% \pkg{hyperref} 与 \pkg{unicode-math} 存在一些兼容性问题,见
% \href{https://github.com/ustctug/ustcthesis/issues/223}{%
%   ustctug/ustcthesis\#223},
% \href{https://github.com/ho-tex/hyperref/pull/90}{ho-tex/hyperref\#90} 和
% \href{https://github.com/ustctug/ustcthesis/issues/235}{%
%   ustctug/ustcthesis/\#235}。
%    \begin{macrocode}
  \@ifpackagelater{hyperref}{2019/04/27}{}{%
    \g@addto@macro\psdmapshortnames{\let\mu\textmu}
  }%
  \ifbnu@main@language@chinese
    \hypersetup{
      pdflang = zh-CN,
    }%
  \else
    \hypersetup{
      pdflang = en-US,
    }%
  \fi
  \AtBeginDocument{%
    \ifbnu@main@language@chinese
      \hypersetup{
        pdftitle    = \bnu@title,
        pdfauthor   = \bnu@author,
        pdfsubject  = \bnu@degree@category,
        pdfkeywords = \bnu@keywords,
      }%
    \else
      \hypersetup{
        pdftitle    = \bnu@title@en,
        pdfauthor   = \bnu@author@en,
        pdfsubject  = \bnu@degree@category@en,
        pdfkeywords = \bnu@keywords@en,
      }%
    \fi
    \hypersetup{
      pdfcreator={\bnuthesis-v\version}}
  }%
}
%    \end{macrocode}
%
% \subsubsection{\pkg{mathtools} 宏包}
%
% \pkg{mathtools} 会修改 \pkg{unicode-math} 的 \cs{underbrace} 和 \cs{overbrace},
% 需要还原为 \cs{LaTeXunderbrace} 和 \cs{LaTeXoverbrace},
% 参考 \url{https://tex.stackexchange.com/q/521394/82731}。
%    \begin{macrocode}
\AtEndOfPackageFile*{mathtools}{
  \@ifpackageloaded{unicode-math}{
    \let\underbrace\LaTeXunderbrace
    \let\overbrace\LaTeXoverbrace
  }{}
}
%    \end{macrocode}
%
% \subsubsection{\pkg{nomencl} 宏包}
%
%    \begin{macrocode}
\AtEndOfPackageFile*{nomencl}{
  \let\nomname\bnu@denotation@name
  \def\thenomenclature{\begin{denotation}[\nom@tempdim]}
  \def\endthenomenclature{\end{denotation}}
}
%    \end{macrocode}
%
% \subsubsection{\pkg{siunitx} 宏包}
%
%    \begin{macrocode}
\AtEndOfPackageFile*{siunitx}{%
  \newcommand\bnu@set@siunitx@language{%
    \ifbnu@language@chinese
      \sisetup{
        list-final-separator = { 和 },
        list-pair-separator  = { 和 },
        range-phrase         = {~},
      }%
    \else
      \ifbnu@language@english
        \sisetup{
          list-final-separator = {, and },
          list-pair-separator  = { and },
          range-phrase         = { to },
        }%
      \fi
    \fi
  }
  \bnu@set@siunitx@language
  \bnu@option@hook{language}{\bnu@set@siunitx@language}
}
%    \end{macrocode}
%
% \subsubsection{\pkg{amsthm} 宏包}
%
% 定理标题使用黑体,正文使用宋体,冒号隔开。
%    \begin{macrocode}
\AtEndOfPackageFile*{amsthm}{%
  \newtheoremstyle{bnu}
    {\z@}{\z@}
    {\normalfont}{\z@}
    {\normalfont\heiti}{\bnu@theorem@separator}
    {0.5em}{}
  \theoremstyle{bnu}
  \newtheorem{assumption}{\bnu@assumption@name}[chapter]%
  \newtheorem{definition}{\bnu@definition@name}[chapter]%
  \newtheorem{proposition}{\bnu@proposition@name}[chapter]%
  \newtheorem{lemma}{\bnu@lemma@name}[chapter]%
  \newtheorem{theorem}{\bnu@theorem@name}[chapter]%
  \newtheorem{axiom}{\bnu@axiom@name}[chapter]%
  \newtheorem{corollary}{\bnu@corollary@name}[chapter]%
  \newtheorem{exercise}{\bnu@exercise@name}[chapter]%
  \newtheorem{example}{\bnu@example@name}[chapter]%
  \newtheorem{remark}{\bnu@remark@name}[chapter]%
  \newtheorem{problem}{\bnu@problem@name}[chapter]%
  \newtheorem{conjecture}{\bnu@conjecture@name}[chapter]%
  \renewenvironment{proof}[1][\bnu@proof@name]{\par
    \pushQED{\qed}%
    % \normalfont \topsep6\p@\@plus6\p@\relax
    \normalfont \topsep\z@\relax
    \trivlist
    \item[\hskip\labelsep
      %     \itshape
      % #1\@addpunct{.}]\ignorespaces
      \sffamily
      #1]\ignorespaces
  }{%
    \popQED\endtrivlist\@endpefalse
  }
  \renewcommand\qedsymbol{\bnu@qed}
}
%    \end{macrocode}
%
% \subsubsection{\pkg{ntheorem} 宏包}
%
% 定理标题使用黑体,正文使用宋体,冒号隔开。
%    \begin{macrocode}
\AtEndOfPackageFile*{ntheorem}{%
  \theorembodyfont{\normalfont}%
  \theoremheaderfont{\normalfont\sffamily}%
  \theoremsymbol{\bnu@qed}%
  \newtheorem*{proof}{\bnu@proof@name}%
  \theoremstyle{plain}%
  \theoremsymbol{}%
  \theoremseparator{\bnu@theorem@separator}%
  \newtheorem{assumption}{\bnu@assumption@name}[chapter]%
  \newtheorem{definition}{\bnu@definition@name}[chapter]%
  \newtheorem{proposition}{\bnu@proposition@name}[chapter]%
  \newtheorem{lemma}{\bnu@lemma@name}[chapter]%
  \newtheorem{theorem}{\bnu@theorem@name}[chapter]%
  \newtheorem{axiom}{\bnu@axiom@name}[chapter]%
  \newtheorem{corollary}{\bnu@corollary@name}[chapter]%
  \newtheorem{exercise}{\bnu@exercise@name}[chapter]%
  \newtheorem{example}{\bnu@example@name}[chapter]%
  \newtheorem{remark}{\bnu@remark@name}[chapter]%
  \newtheorem{problem}{\bnu@problem@name}[chapter]%
  \newtheorem{conjecture}{\bnu@conjecture@name}[chapter]%
}
%    \end{macrocode}
%
% \subsubsection{\pkg{algorithm} 宏包}
%
% 使 \env{algorithm} 和 \env{listing} 环境的名称随语言设置而改变,
% 并使其在附录中的编号规则与图、表等一致。
% 对算法环境编号是否带有章节进行了判断,使得与图、表一致。
%
% \begin{macro}{\listofalgorithm}
% \begin{macro}{\listofalgorithm*}
%    \begin{macrocode}
\bnu@option@hook{figurestables-chapternumber}{
  \ifbnu@figurestables@chapternumber@true
    \PassOptionsToPackage{chapter}{algorithm}
  \else 
    \relax
  \fi
}
\AtEndOfPackageFile*{algorithm}{
  \floatname{algorithm}{\bnu@algorithm@name}
  \renewcommand\listofalgorithms{%
    \bnu@listof{algorithm}%
  }
  \renewcommand\listalgorithmname{\bnu@list@algorithm@name}
  \def\ext@algorithm{loa}
  \contentsuse{algorithm}{loa}
  \titlecontents{algorithm}
    [\z@]{}
    {\contentspush{\fname@algorithm~\thecontentslabel\quad}}{}
    {\bnu@leaders\thecontentspage}
}
%    \end{macrocode}
% \end{macro}
% \end{macro}
%
% \subsubsection{\pkg{algorithm2e} 宏包}
%
%    \begin{macrocode}
\bnu@option@hook{figurestables-chapternumber}{
  \ifbnu@figurestables@chapternumber@true
    \PassOptionsToPackage{algochapter}{algorithm2e}
  \else 
    \relax
  \fi
}
\AtEndOfPackageFile*{algorithm2e}{
  \renewcommand\algorithmcfname{\bnu@algorithm@name}
  \SetAlgoCaptionLayout{bnu@caption@font}
  \SetAlCapSty{relax}
  \SetAlgoCaptionSeparator{\hspace*{1em}}
  \SetAlFnt{\fontsize{11bp}{14.3bp}\selectfont}
  \renewcommand\listofalgorithms{%
    \bnu@listof{algorithmcf}%
  }
  \renewcommand\listalgorithmcfname{\bnu@list@algorithm@name}
  \def\ext@algorithmcf{loa}
  \contentsuse{algocf}{loa}
  \titlecontents{algocf}
    [\z@]{}
    {\contentspush{\algorithmcfname~\thecontentslabel\quad}}{}
    {\bnu@leaders\thecontentspage}
}
%    \end{macrocode}
%
% \subsubsection{\pkg{minted} 宏包}
%
%    \begin{macrocode}
\AtEndOfPackageFile*{minted}{
  \newcommand\bnu@set@listing@language{%
    \ifbnu@language@chinese
      \floatname{listing}{代码}%
    \else
      \floatname{listing}{Listing}%
    \fi
  }
  \bnu@set@listing@language
  \bnu@option@hook{language}{\bnu@set@listing@language}
}
%    \end{macrocode}
%
% \subsection{书脊}
% \label{sec:spine}
% \begin{macro}{\spine}
% 单独使用书脊命令会在新的一页产生竖排书脊,
% 参考 \url{https://tex.stackexchange.com/a/38585}。
%
% 本科生:
%   书脊的书写要求:用仿宋\_GB2312 字书写,字体大小根据论文的薄厚而定。
%   书脊上方写论文题目,下方写本科生姓名,距上下页边均为 3cm。
%
% 研究生:
%   硕士学位论文可不打印书脊,博士学位论文要求打印书脊,字体为黑体小四号。
%   示例中上下页边距为 5.5 cm,左右边距为 1 cm。
%    \begin{macrocode}
\bnu@define@key{
  spine-font = {
    name = spine@font,
  },
  spine-title = {
    name = spine@title,
  },
  spine-author = {
    name = spine@author,
  },
  spine-school = {
    name = spine@school,
  },
}
\renewcommand\bnu@spine@font{%
  \ifbnu@degree@graduate
    \xiaosi
  \fi
}
\newcommand*\CJKmovesymbol[1]{\raise.3em\hbox{#1}}
\newcommand*\CJKmove{%
  \punctstyle{plain}%
  \let\CJKsymbol\CJKmovesymbol
  \let\CJKpunctsymbol\CJKsymbol
}
\NewDocumentCommand{\spine}{
    O{
      \ifx\bnu@spine@title\@empty
        \bnu@title
      \else
        \bnu@spine@title
      \fi
    }
    O{
      \ifx\bnu@spine@author\@empty
        \bnu@author
      \else
        \bnu@spine@author
      \fi
    }
    O{
      \bnu@spine@school
    }}{%
  \clearpage
  \ifbnu@degree@bachelor
    \newgeometry{
      vmargin = 3cm,
      hmargin = 1cm,
    }%
  \else
    \newgeometry{
      vmargin = 5.5cm,
      hmargin = 1cm,
    }%
  \fi
  \thispagestyle{empty}%
  \ifbnu@main@language@chinese
    \bnu@pdfbookmark{0}{书脊}%
  \else
    \bnu@pdfbookmark{0}{Spine}%
  \fi
  \begingroup
    \noindent\hfill
    \rotatebox[origin=lt]{-90}{%
      \makebox[\textheight]{%
        \heiti
        \addCJKfontfeatures*{RawFeature={vertical}}%
        \bnu@spine@font
        \CJKmove
        #1\hfill
        \bnu@stretch{3em}{#2}
        \hfill
        \bnu@stretch{4.5em}{#3}%
      }%
    }%
  \endgroup
  \clearpage
  \restoregeometry
}
%    \end{macrocode}
% \end{macro}
%
%
% \subsection{其它}
% \label{sec:other}
%
% 借用 \cls{ltxdoc} 和 \cls{l3doc} 里面的几个命令方便写文档。
%    \begin{macrocode}
\DeclareRobustCommand\cs[1]{\texttt{\char`\\#1}}
\DeclareRobustCommand\file{\nolinkurl}
\DeclareRobustCommand\env{\textsf}
\DeclareRobustCommand\pkg{\textsf}
\DeclareRobustCommand\cls{\textsf}
%    \end{macrocode}
%
%    \begin{macrocode}
\sloppy
%</cls>
%    \end{macrocode}
%
%
% \iffalse
%    \begin{macrocode}
%<*dtx-style>
\ProvidesPackage{dtx-style}
\RequirePackage{hypdoc}
\RequirePackage{ifthen}
\RequirePackage{fontspec}[2017/01/20]
\RequirePackage{amsmath}
\RequirePackage{unicode-math}
\RequirePackage{siunitx}
\RequirePackage[UTF8,scheme=chinese]{ctex}
\RequirePackage[
  top=2.5cm, bottom=2.5cm,
  left=4cm, right=2cm,
  headsep=3mm]{geometry}
\RequirePackage{hologo}
\RequirePackage{array,longtable,booktabs}
\RequirePackage{listings}
\RequirePackage{fancyhdr}
\RequirePackage{xcolor}
\RequirePackage{enumitem}
\RequirePackage{etoolbox}
\RequirePackage{metalogo}

% Pretending the `amssymb` has been loaded.
% This should be removed after `markdown` v2.13.0.
\expandafter\def\csname ver@amssymb.sty\endcsname{0000/00/00}
\RequirePackage[tightLists=false]{markdown}
\expandafter\let\csname ver@amssymb.sty\endcsname\relax

\markdownSetup{
  renderers = {
    link = {\href{#2}{#1}},
  }
}

\hypersetup{
  pdflang     = zh-CN,
  pdftitle    = {BNUThesis:北京师范大学学位论文模板},
  pdfauthor   = {北京师范大学图书馆},
  pdfsubject  = {北京师范大学学位论文模板使用说明},
  pdfkeywords = {论文模板; 北京师范大学; 使用说明},
  pdfdisplaydoctitle = true
}%

\setmainfont[
  Extension      = .otf,
  UprightFont    = *-regular,
  BoldFont       = *-bold,
  ItalicFont     = *-italic,
  BoldItalicFont = *-bolditalic,
]{texgyrepagella}
\setsansfont[
  Extension      = .otf,
  UprightFont    = *-regular,
  BoldFont       = *-bold,
  ItalicFont     = *-italic,
  BoldItalicFont = *-bolditalic,
]{texgyreheros}
\setmonofont[
  Extension      = .otf,
  UprightFont    = *-regular,
  BoldFont       = *-bold,
  ItalicFont     = *-italic,
  BoldItalicFont = *-bolditalic,
  Scale          = MatchLowercase,
  Ligatures      = CommonOff,
]{texgyrecursor}

\unimathsetup{
  math-style=ISO,
  bold-style=ISO,
}
\DeclareRobustCommand\mathellipsis{\mathinner{\unicodecdots}}
\IfFontExistsTF{XITSMath-Regular.otf}{
  \setmathfont[
    Extension    = .otf,
    BoldFont     = XITSMath-Bold,
    StylisticSet = 8,
  ]{XITSMath-Regular}
  \setmathfont[range={cal,bfcal},StylisticSet=1]{XITSMath-Regular.otf}
}{
  \setmathfont[
    Extension    = .otf,
    BoldFont     = *bold,
    StylisticSet = 8,
  ]{xits-math}
  \setmathfont[range={cal,bfcal},StylisticSet=1]{xits-math.otf}
}

\colorlet{bnu@macro}{blue!60!black}
\colorlet{bnu@env}{blue!70!black}
\colorlet{bnu@option}{purple}
\patchcmd{\PrintMacroName}{\MacroFont}{\MacroFont\bfseries\color{bnu@macro}}{}{}
\patchcmd{\PrintDescribeMacro}{\MacroFont}{\MacroFont\bfseries\color{bnu@macro}}{}{}
\patchcmd{\PrintDescribeEnv}{\MacroFont}{\MacroFont\bfseries\color{bnu@env}}{}{}
\patchcmd{\PrintEnvName}{\MacroFont}{\MacroFont\bfseries\color{bnu@env}}{}{}

\def\DescribeOption{%
  \leavevmode\@bsphack\begingroup\MakePrivateLetters%
  \Describe@Option}
\def\Describe@Option#1{\endgroup
  \marginpar{\raggedleft\PrintDescribeOption{#1}}%
  \bnu@special@index{option}{#1}\@esphack\ignorespaces}
\def\PrintDescribeOption#1{\strut \MacroFont\bfseries\sffamily\color{bnu@option} #1\ }
\def\bnu@special@index#1#2{\@bsphack
  \begingroup
    \HD@target
    \let\HDorg@encapchar\encapchar
    \edef\encapchar usage{%
      \HDorg@encapchar hdclindex{\the\c@HD@hypercount}{usage}%
    }%
    \index{#2\actualchar{\string\ttfamily\space#2}
           (#1)\encapchar usage}%
    \index{#1:\levelchar#2\actualchar
           {\string\ttfamily\space#2}\encapchar usage}%
  \endgroup
  \@esphack}

\lstdefinestyle{lstStyleBase}{%
   basicstyle=\small\ttfamily,
   aboveskip=\medskipamount,
   belowskip=\medskipamount,
   lineskip=0pt,
   boxpos=c,
   showlines=false,
   extendedchars=true,
   upquote=true,
   tabsize=2,
   showtabs=false,
   showspaces=false,
   showstringspaces=false,
   numbers=none,
   linewidth=\linewidth,
   xleftmargin=4pt,
   xrightmargin=0pt,
   resetmargins=false,
   breaklines=true,
   breakatwhitespace=false,
   breakindent=0pt,
   breakautoindent=true,
   columns=flexible,
   keepspaces=true,
   gobble=4,
   framesep=3pt,
   rulesep=1pt,
   framerule=1pt,
   backgroundcolor=\color{gray!5},
   stringstyle=\color{green!40!black!100},
   keywordstyle=\bfseries\color{blue!50!black},
   commentstyle=\slshape\color{black!60}}

\lstdefinestyle{lstStyleShell}{%
   style=lstStyleBase,
   frame=l,
   rulecolor=\color{purple},
   language=bash}

\lstdefinestyle{lstStyleLaTeX}{%
   style=lstStyleBase,
   frame=l,
   rulecolor=\color{violet},
   language=[LaTeX]TeX}

\lstnewenvironment{latex}{\lstset{style=lstStyleLaTeX}}{}
\lstnewenvironment{shell}{\lstset{style=lstStyleShell}}{}

\setlist{nosep}

\DeclareDocumentCommand{\option}{m}{\textsf{#1}}
\DeclareDocumentCommand{\env}{m}{\texttt{#1}}
\DeclareDocumentCommand{\pkg}{s m}{%
  \textsf{#2}\IfBooleanF#1{\bnu@special@index{package}{#2}}}
\DeclareDocumentCommand{\cls}{s m}{%
  \textsf{#2}\IfBooleanF#1{\bnu@special@index{package}{#2}}}
\DeclareDocumentCommand{\file}{s m}{%
  \nolinkurl{#2}\IfBooleanF#1{\bnu@special@index{file}{#2}}}
\newcommand{\myentry}[1]{%
  \marginpar{\raggedleft\color{purple}\bfseries\strut #1}}
\newcommand{\note}[2][Note]{{%
  \color{magenta}{\bfseries #1}\emph{#2}}}

\g@addto@macro\UrlBreaks{%
  \do0\do1\do2\do3\do4\do5\do6\do7\do8\do9%
  \do\A\do\B\do\C\do\D\do\E\do\F\do\G\do\H\do\I\do\J\do\K\do\L\do\M
  \do\N\do\O\do\P\do\Q\do\R\do\S\do\T\do\U\do\V\do\W\do\X\do\Y\do\Z
  \do\a\do\b\do\c\do\d\do\e\do\f\do\g\do\h\do\i\do\j\do\k\do\l\do\m
  \do\n\do\o\do\p\do\q\do\r\do\s\do\t\do\u\do\v\do\w\do\x\do\y\do\z
}
\Urlmuskip=0mu plus 0.1mu

\DeclareDocumentCommand{\githubuser}{m}{\href{https://github.com/#1}{@#1}}

\def\bnuthesis{\textsc{BNU}\-\textsc{Thesis}}
\def\thuthesis{\textsc{Thu}\-\textsc{Thesis}}
%</dtx-style>
%    \end{macrocode}
% \fi
%
% \Finale
%
\endinput
% \iffalse
%  Local Variables:
%  mode: doctex
%  TeX-master: t
%  End:
% \fi
